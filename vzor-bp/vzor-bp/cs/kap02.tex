\chapter{Programátorská dokumentace}
V této kapitole popíšeme naší implementaci platformy MHUrho a ukázkové hry. K implementaci platformy bylo použito Visual Studio 2017 Education a .NET Framework 4.7.2, který je zároveň cílovým frameworkem naší platformy. Celá implementace je obsažena v jediném \uv{solution}, MHUrho. Toto solution se skládá z několika projektů, představujících implementaci vlastní platformy, instalátoru platformy a ukázkové hry.

\section{Skupina projektů MHUrho}
Implementace platformy je tvořena třemi projekty. Hlavním z těchto projektů je projekt \textit{MHUrho}, jehož výstupem je \uv{knihovna} MHUrho, implementující jádro funkcionality našeho projektu. Tato knihovna je poskytována tvůrcům her pro tvorbu pluginů. Dále tato knihovna obsahuje přenositelné části implementací požadované funkcionality 

Zbylé dva projekty vytvářejí aplikaci cílenou na jediný systém. Výstupem \textit{MHUrho.Desktop} je implementace platformy pro desktopové systémy, pro nás zatím pouze systém Windows. Výstupem \textit{MHUrho.Android} je verze platformy pro mobilní systém Android. Tato verze obsahuje pouze kostry řešení a slouží zde pouze jako bod budoucí rozšiřitelnosti platformy. Obě tyto aplikace využívají knihovnu MHUrho, která, jak bylo zmíněno výše, implementuje přenositelnou část funkcionality. Dále knihovna definuje rozhraní a třídy, které nelze implementovat přenositelně a je tedy nutné pro každý z cílových systému vytvořit zvláštní implementaci, která je následně při startu aplikace poskytnuta knihovně.

Drtivou většinu funkcionality jsme byli schopni implementovat přenositelně, tato sekce se tedy bude zabývat především implementací knihovny MHUrho a až na závěr zmíníme funkcionalitu, kterou jsme byli nuceni vysunout do projektů pro jednotlivé systémy.

Knihovna MHUrho je rozdělena do několika \textit{namespace}, tedy jmenných prostorů, které spojují třídy zaobírající se společným aspektem funkcionality. Těmito jmennými prostory jsou:
\begin{enumerate}
	\item \texttt{Logic},
	\item \texttt{WorldMap},
	\item \texttt{Packaging},
	\item \texttt{Plugins},
	\item \texttt{UserInterface},
	\item \texttt{Input},
	\item \texttt{Pathfinding},
	\item \texttt{DefaultComponents},
	\item \texttt{CameraMovement},
	\item \texttt{Helpers},
	\item \texttt{EditorTools},
	\item \texttt{Storage},
	\item ostatní méně důležité.
\end{enumerate}

V následujících sekcích postupně projdeme jednotlivé jmenné prostory, popíšeme obsažené třídy, jejich účel a interakci s třídami z jiných jmenných prostorů.

\section{Logika hry}
Logika hry, obsažená ve jmenném prostoru \texttt{Logic}, je tvořena skupinou tříd reprezentujících herní svět z pohledu umělé inteligence hráčů, jednotek a budov. 
Z pohledu herního enginu je každá úroveň reprezentována scénou, tedy instancí třídy \texttt{Scene}. Tato scéna tvoří kořen grafu, ve kterém se nacházejí všechny \texttt{Node} reprezentující herní mapu, kameru a entity přítomné ve hře. Struktura tohoto grafu je vždy probrána v odpovídající části dokumentace. Na každou z těchto \texttt{Node} lze přidat instance potomků třídy \texttt{Component}, které následně implementují chování objektu reprezentovaného danou \texttt{Node}.

\subsection{LevelManager}
Hlavním třídou, reprezentující aktuálně spuštěnou úroveň, je třída \texttt{LevelManager}. Jak můžeme vidět na grafu referencí \ref{fig:levelmanagergraph}, slouží tato třída jako centrální přístupový bod ke službám platformy, jak pro pluginy, tak pro samotné části platformy a dále slouží jako repozitář všech entit, přítomných ve hře. Každá součást aktuální úrovně dostává referenci na instanci třídy LevelManager a pomocí ní přistupuje ke zbytku platformy.

Tato třída je potomkem třídy \texttt{Component}, která umožňuje přidat ji do grafu scény a obsluhovat události odehrávající se ve scéně. I když by logicky tato komponenta, řídící chování celé úrovně, měla být přiřazena přímo instanci Scény, vytvořili jsme separátní \texttt{Node}, reprezentující logiku úrovně, na které je přidána právě komponenta LevelManager. Účelem této změny bylo umožnění selektivního vypínání logiky úrovně bez vypnutí vykreslování, které je jako jakékoli jiné chování reprezentováno komponentou a při vypnutí scény přestane fungovat.

Jako centrální přístupový bod umožňuje LevelManager vytváření nových Entit, získávání referencí na existující a jejich případné odstraňování. 

Dále slouží jako centrální bod pro spuštění načítání úrovně, které bude přiblíženo v části \ref{sec:loading}.
\subsection{Entity}
\texttt{Entity} je abstraktní třída, sloužící jako předek všech jednotek, budov a projektilů ve hře. Tato třída deklaruje či přímo implementuje funkce a \uv{properties} společné všem entitám. Mezi tyto property patří hráč vlastnící entitu, reference na LevelManager aktuální úrovně, pozice v herním světě a v neposlední řadě reference na plugin ovládající entitu.

Třída Entity je sama potomkem třídy \texttt{Component}, která je poskytována enginem UrhoSharp. Tento vztah nám umožňuje připojit třídu Entity a všechny její potomky k instanci \texttt{Node} odpovídající entitě v grafu scény.

\section{Mapa}

\subsection{Načítání hry}
Načítání velké části tříd je implementováno za použití \texttt{Builder} design patternu. Třídy jako Map, Tile, Player, Unit, Building, Projectile a DefaultComponent obsahují implementaci rozhraní \texttt{ILoader}, které slouží pro načítání objektů. 
\section{Pluginy}

\section{Manipulace s balíčky}



