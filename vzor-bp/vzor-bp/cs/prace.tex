%%% Hlavní soubor. Zde se definují základní parametry a odkazuje se na ostatní části. %%%

%% Verze pro jednostranný tisk:
% Okraje: levý 40mm, pravý 25mm, horní a dolní 25mm
% (ale pozor, LaTeX si sám přidává 1in)
\documentclass[12pt,a4paper,usenames]{report}
\setlength\textwidth{145mm}
\setlength\textheight{247mm}
\setlength\oddsidemargin{15mm}
\setlength\evensidemargin{15mm}
\setlength\topmargin{0mm}
\setlength\headsep{0mm}
\setlength\headheight{0mm}
% \openright zařídí, aby následující text začínal na pravé straně knihy
\let\openright=\clearpage

%% Pokud tiskneme oboustranně:
%\documentclass[12pt,a4paper,twoside,openright,usenames]{report}
%\setlength\textwidth{145mm}
%\setlength\textheight{247mm}
%\setlength\oddsidemargin{14.2mm}
%\setlength\evensidemargin{0mm}
%\setlength\topmargin{0mm}
%\setlength\headsep{0mm}
%\setlength\headheight{0mm}
%\let\openright=\cleardoublepage

%% Vytváříme PDF/A-2u
\usepackage[a-2u]{pdfx}

%% Přepneme na českou sazbu a fonty Latin Modern
\usepackage[czech]{babel}
\usepackage{lmodern}
\usepackage[T1]{fontenc}
\usepackage{textcomp}

%% Použité kódování znaků: obvykle latin2, cp1250 nebo utf8:
\usepackage[utf8]{inputenc}

%%% Další užitečné balíčky (jsou součástí běžných distribucí LaTeXu)
\usepackage{amsmath}        % rozšíření pro sazbu matematiky
\usepackage{amsfonts}       % matematické fonty
\usepackage{amsthm}         % sazba vět, definic apod.
\usepackage{bbding}         % balíček s nejrůznějšími symboly
			    % (čtverečky, hvězdičky, tužtičky, nůžtičky, ...)
\usepackage{bm}             % tučné symboly (příkaz \bm)
\usepackage{graphicx}       % vkládání obrázků
\usepackage{fancyvrb}       % vylepšené prostředí pro strojové písmo
\usepackage{indentfirst}    % zavede odsazení 1. odstavce kapitoly
\usepackage[numbers]{natbib}         % zajištuje možnost odkazovat na literaturu
			    % stylem AUTOR (ROK), resp. AUTOR [ČÍSLO]
\usepackage[nottoc]{tocbibind} % zajistí přidání seznamu literatury,
                            % obrázků a tabulek do obsahu
\usepackage{icomma}         % inteligetní čárka v matematickém módu
\usepackage{dcolumn}        % lepší zarovnání sloupců v tabulkách
\usepackage{booktabs}       % lepší vodorovné linky v tabulkách
\usepackage{paralist}       % lepší enumerate a itemize
\usepackage{xcolor}  % barevná sazba

% Moje balíčky

\usepackage[marginpar]{todo}
\usepackage{subfig}
\usepackage{enumitem}
\usepackage{listings}

\PassOptionsToPackage{hyphens}{url}\usepackage{hyperref}



%%nastaveni C# syntax highlight

\usepackage{color}
\definecolor{bluekeywords}{rgb}{0,0,1}
\definecolor{classgreen}{rgb}{0.17,0.57,0.68}
\definecolor{greencomments}{rgb}{0,0.5,0}
\definecolor{redstrings}{rgb}{0.64,0.08,0.08}
\definecolor{black}{rgb}{0,0,0}
\definecolor{blueattributes}{rgb}{0.37,0.52,0.62}
\definecolor{interfaceyellow}{rgb}{0.60,0.66,0.11}

%\lstdefinelanguage{XML}
%{
%	morestring=[b]",
%	morestring=[s]{>}{<},
%	morecomment=[s]{<?}{?>},
%	morecomment=[s]{<!--}{-->}
%	stringstyle=\color{black},
%	identifierstyle=\color{bluekeywords},
%	keywordstyle=\color{blueattributes},
%	morekeywords={xmlns,version,type}% list your attributes here
%}
\lstdefinestyle{csharp} {
	language=[Sharp]C,
	commentstyle=\color{greencomments},
	stringstyle=\color{redstrings}\ttfamily, 
	keywordstyle=\color{bluekeywords},
	morekeywords={ await, new, async, var },
	emphstyle=[1]{\color{classgreen}},
	emphstyle=[2]{\color{interfaceyellow}}
}

\lstdefinestyle{xml} {
	language=XML,
	stringstyle=\color{black}\ttfamily,
	keywordstyle=\color{bluekeywords},
	emphstyle=[1]{\color{blueattributes}},
}

\lstset{	
	captionpos=b,
	%numbers=left,
	%numberstyle=\tiny,
	columns=flexible,
	frame=single, 
	showspaces=false,
	showtabs=false,
	breaklines=true,
	showstringspaces=false,
	breakatwhitespace=true,
	escapeinside={(*@}{@*)},
	basicstyle=\ttfamily,
	tabsize=2
}

%% Nastavení cesty k adresáři s obrázky pro balíček graphicx
\graphicspath{ {./img/} }

%% Nastaveni enumerate a itemize
\setlist[enumerate,1]{label={\arabic*)}}
\setlist[enumerate,2]{label={\alph*)}}


%%% Údaje o práci

% Název práce v jazyce práce (přesně podle zadání)
\def\NazevPrace{UrhoRTS - platforma pro tvorbu realtimových strategických her}

% Název práce v angličtině
\def\NazevPraceEN{UrhoRTS - Platform for Real-time Strategy Game Creation}

% Jméno autora
\def\AutorPrace{Karel Maděra}

% Rok odevzdání
\def\RokOdevzdani{2019}

% Název katedry nebo ústavu, kde byla práce oficiálně zadána
% (dle Organizační struktury MFF UK, případně plný název pracoviště mimo MFF)
\def\Katedra{Katedra distribuovaných a spolehlivých systémů}
\def\KatedraEN{Department of Distributed and Dependable Systems}

% Jedná se o katedru (department) nebo o ústav (institute)?
\def\TypPracoviste{Katedra}
\def\TypPracovisteEN{Department}

% Vedoucí práce: Jméno a příjmení s~tituly
\def\Vedouci{Mgr. Pavel Ježek, Ph.D.}

% Pracoviště vedoucího (opět dle Organizační struktury MFF)
\def\KatedraVedouciho{Katedra distribuovaných a spolehlivých systémů}
\def\KatedraVedoucihoEN{Department of Distributed and Dependable Systems}

% Studijní program a obor
\def\StudijniProgram{Informatika (B1801)}
\def\StudijniObor{Programování a softwarové systémy (IPSS)}

% Nepovinné poděkování (vedoucímu práce, konzultantovi, tomu, kdo
% zapůjčil software, literaturu apod.)
\def\Podekovani{%
Chtěl bych poděkovat především vedoucímu práce, Mgr. Pavlu Ježkovi, Ph.D., za jeho trpělivost a nedocenitelné rady a postřehy při tvorbě této práce. Dále bych chtěl poděkovat svým rodičům, jejichž podpora byla nedílnou součástí tvorby tohoto díla.
}

% Abstrakt (doporučený rozsah cca 80-200 slov; nejedná se o zadání práce)
\def\Abstrakt{%
Vývoj Realtimových strategických her (RTS) je složitým procesem spojujícím mnoho oborů. Cílem naší práce je vytvoření platformy zjednodušující proces tvorby 3D RTS her pro jednoho hráče a umožňující tvorbu logiky her jako pluginů v jazyce C\#. 

Platforma umožňuje tvorbu her jako balíčků, definovaných pomocí XML souborů, obsahujících modely, textury, animace, definice grafického uživatelského rozhraní a pluginy. Pomocí těchto pluginů, vytvořených pomocí jazyka C\#, umožňuje platforma definici umělé inteligence hráčů, jednotek, budov či projektilů obsažených v balíčcích. Platforma dále poskytuje funkce použitelné při implementaci pluginů. 

Součástí práce je ukázkový balíček, obsahující implementaci hry demonstrující schopnosti platformy.
}


\def\AbstraktEN{%
The development of Real-time strategy (RTS) games is a difficult process spanning many fields. The goal of this thesis is to create a platform to ease the development of 3D single player RTS games and to enable the use of C\# language for plugin creation. 

Our platform enables users to create games as packages for the platform. Each package is defined by a single XML file, describing the contents of the package, which include 3D models, textures, animations, graphical user interface definitions and plugins. These plugins, created using the C\# language, enable the game creator to create artificial intelligence for players, units, buildings and projectiles defined in the package. The platform also provides functions that can be used for creation of plugins.

As a part of this thesis, we will create a showcase package to demonstrate the abilities of our platform.
}

% 3 až 5 klíčových slov (doporučeno), každé uzavřeno ve složených závorkách
\def\KlicovaSlova{%
{RTS hra} {tvorba RTS her} {editace úrovní} {MHUrho}
}
\def\KlicovaSlovaEN{%
{RTS game} {RTS game creation} {level editing} {MHUrho}
}

%% Balíček hyperref, kterým jdou vyrábět klikací odkazy v PDF,
%% ale hlavně ho používáme k uložení metadat do PDF (včetně obsahu).
%% Většinu nastavítek přednastaví balíček pdfx.
\hypersetup{unicode}
\hypersetup{breaklinks=true}

%% Definice různých užitečných maker (viz popis uvnitř souboru)
\include{makra}

%% Titulní strana a různé povinné informační strany
\begin{document}
\include{titulka}

%%% Strana s automaticky generovaným obsahem bakalářské práce

\tableofcontents

%%% Jednotlivé kapitoly práce jsou pro přehlednost uloženy v samostatných souborech
\chapter{Úvod}
%\addcontentsline{toc}{chapter}{Úvod}

Strategické hry jsou žánrem, ve kterém hráči využívají svých mentálních schopností, především taktického a strategického myšlení, pro dosažení cíle. To je jedním z důvodů, proč je tento druh her oblíben mezi studenty naší fakulty. Naše práce se zaměří na ulehčení tvorby konkrétního poddruhu strategických her, a to \emph{Real-time strategy} (RTS) her. K tomuto účelu bude vytvořena platforma rozšiřující herní engine UrhoSharp pro tvorbu RTS her. Oproti předešlým pracím, které se většinou zabývaly 2D RTS hrami, se my pokusíme vytvořit platformu umožňující tvorbu 3D RTS her.

\section{RTS hry}
Real-time strategy hry, v překladu strategické hry probíhající v reálném čase, jsou poddruhem strategických her ve kterém se změny stavu odehrávají v přímé závislosti na změně času v reálném světě. Tato vlastnost je odlišuje od tahových strategických her, ve~kterých má hráč neomezenou dobu na přemýšlení o svém tahu, což mu umožňuje vymyslet optimální strategii. Reakce v reálném čase jsou náročnější jak pro hráče, který je často nucen použít suboptimální strategii, tak pro hru samotnou, která musí provádět výpočet dalšího stavu v omezeném čase a a s omezeným výkonem, o který bojuje s vykreslováním hry. Stejně tak umělá inteligence hráčů musí reagovat na aktivity zbylých hráčů s omezeným časem a výkonem, což limituje množství dat a složitost výpočtu, které může umělá inteligence použít. Z tohoto důvodu je vývoj RTS her složitým procesem, spojujícím mnoho oborů. 

Již od svého vzniku na konci osmdesátých let a začátku devadesátých let minulého století \todo{jiný slovo než obsahovaly} obsahovaly RTS hry několik konceptů, které lze nalézt v drtivé většině her tohoto žánru i dnes. 

Těmito koncepty jsou:
\begin{itemize}
	\item Výroba a ovládání jednotek s cílem ovládnutí části mapy a zničení nepřátelských jednotek a budov
	\item Stavba budov pro umožnění stavby nových druhů budov, jednotek či získání surovin
	\item Získávání surovin pro stavbu jednotek a budov
	\item Výzkum nových technologií
\end{itemize}

Základním stavebním kamenem RTS her je pečlivé vyvážení kombinace strategie/macromanagementu a taktiky/micromanagementu. Tyto pojmy bývají často špatně chápány a někdy zaměňovány, pokusíme se je proto konkrétně definovat. Strategií míníme \todo{citace} rozhodnutí týkající se globálního průběhu hry. Taktika \todo{citace} naopak zahrnuje konkrétní pozice konkrétních jednotek, jejich pohyb po mapě a spolupráci v jedné bitvě.  

Mezi strategická rozhodnutí v RTS hrách patří kupříkladu které budovy hráč postaví, v jakém pořadí dané budovy postaví, které suroviny bude produkovat, které suroviny vynechá, které jednotky bude rekrutovat a které vynechá. \todo{mozna graf nasledujiciho, jak se ovlivnujou} Tato 3 rozhodnutí jsou úzce propojena, protože výběr surovin určuje budovy, které bude hráč schopen postavit a typy jednotek, které bude moci zrekrutovat. Stejně tak výběr budov určuje suroviny, které hráč může produkovat a typy jednotek, které může zrekrutovat. \todo{pojem ekonomie}. 

Taktika/micromanagement je v RTS hrách reprezentován ovládáním jednotek, jejich přesné pozice, pohybu, směru útoku, používání schopností atd. Micromanagement lze ale vidět i v ekonomické stránce hry, kdy hráč 

Nejlepším příkladem pro rozlišení pojmu strategie a taktiky je série her Total War, které kombinuje mód tahové strategie s módem real-time tactics (RTT). V jedné části hry hráč přebírá kontrolu nad celým svým národem, rozhoduje, které budovy budou ve kterých městech postaveny, které jednotky budou rekrutovány a kde budou které armády umístěny. V druhé části, při souboji nepřátelských armád, hráč přebírá kontrolu nad konkrétní armádou a ovládá jednotlivé bataliony, jejich umístění, pohyb a útoky v reálném čase.

Tento příklad nám také ukazuje sousední žánry her na spektru důležitosti strategie a taktiky. Na jedné straně máme tahové strategie, které oproti RTS minimalizují důležitost micromanagementu a soustředí se především na globální strategii. Tahové strategie se vyvinuly ze stolních her, od kterých přebraly velkou část svých herních mechanik. \cite{https://web.archive.org/web/20081201113349/http://archive.gamespy.com/articles/february02/strat04/index.shtm}. Na druhé straně spektra máme RTT a turn-based tactics hry, které se zbavují strategických prvků v podobě surovin a budov, výměnou za větší taktické možnosti a detailnější ovládání jednotek. Příkladem čistě RTT her může být série Blitzkrieg \todo{odkaz nebo citace}. Nejznámějším zástupcem turn-based tactics her je série her X-COM.  

Hranice mezi RTT a RTS není přesně definována, jedná se spíše o plynulý přechod mezi těmito dvěma žánry. Dnešní RTS hry bývají často kritizovány pro zvyšující se důležitost taktiky na úkor strategie, což je posouvá stále blíže k žánru RTT. Tento posun také mění schopnosti potřebné pro hraní těchto her. Postupný přechod k micromanagementu dává velkou výhodu hráčům s rychlejšími reakcemi místo hráčů schopných strategického myšlení.

Jako příklad RTS můžeme vzít hru \emph{Dune II}, která je považována za první hru obsahující všechny prvky RTS.\cite{} Samotný název "real-time strategy", poprvé použitý při propagaci této hry, je připisován prezidentu a spoluzakladateli Westwood Studios, \emph{Brettu Sperrymu}. Toto studio následně využilo zkušenosti získané při tvorbě Dune II pro vývoj jedné z nejznámějších sérií RTS her, Command \& Conquer.


\todo{popsat Dune II}

Hlavní inspirací pro tvorbu platformy, a tím i pro typ her, které bude platforma nejjednodušeji podporovat, byla hra Stronghold. Na této hře, vydané v roce 2002, ukážeme blíže základní principy RTS her.

\subsection{Jednotky}
Jednotky jsou základním nástrojem hráče pro boj s nepřítelem. Pohybem po mapě, poškozováním ostatních jednotek a ničením budov jednotky umožňují hráči vést souboj s protivníkem, získat strategickou výhodu a následně vyhrát hru. 

Hlavním odlišujícím prvkem od budov je možnost pohybu. Tento pohyb je nejčastěji řízen hráčem, ať už na úrovni příkazů jednotlivým jednotkám, tak na úrovni slučování jednotek do skupin a ovládání těchto skupin. Ve velké části RTS her jsou jednotky schopny do určité míry autonomního rozhodování bez zásahu hráče, od střelby na cíl, který se ocitne v jejich dostřelu, po vyhledání krytu, pokud jsou pod palbou. Dobrým příkladem jednotek s vysokou autonomií je série Company of Heroes, kde jednotky automaticky vyhledávají krytí, rozutečou se, pokud jsou pod palbou dělostřelectva, a v případě příliš velkých ztrát utečou z boje. 

Tato autonomie má ale svou cenu, a to v nepředvídatelnosti chování jednotek. Při jednoduché umělé inteligenci jednotek je hráč schopen předvídat jejich chování a využít ho pro svůj prospěch. Naopak při složité umělé inteligenci, jako právě v případě Company of Heroes, je hráč často nucen provést více pokusů při vydávání rozkazu , protože není schopen jednoduše odhadnout chování jednotky. Tato skutečnost činí hry často realističtější, protože simuluje chování reálných vojáků, kteří rozkaz interpretují a implementují podle svého, není ale vhodná pro souboje více hráčů, a už vůbec né více hráčů na profesionální úrovni.

Jako příklad jednoduché umělé inteligence jednotek můžeme vzít hry Starcraft a Warcraft od společnosti Blizzard. Zde se jednotky chovají velice předvídatelně, splňují přesně hráčovi rozkazy a nedělají nic navíc, což umožnilo hře Starcraft II vytořit jednu z prvních masivních e-sport scén na světě. \cite{http://www.gamasutra.com/php-bin/news_index.php?story=18326}

Pro RTS hry je možnost produkce nových jednotek jednou z definujících vlastností. Existuje několik systémů produkce jednotek, úzce svázaných se systémem surovin v dané hře. \todo{see Suroviny} Od kontinuální produkce, kde hráč zvolí produkované jednotky a suroviny jsou spotřebovávány v průběhu produkce, po diskrétní produkci, kde hráč musí vlastnit všechny suroviny potřebné pro výrobu dané jednotky při začátku produkce a všechny suroviny jsou odečteny v jeden okamžik. Kontinuální systém umožňuje hráči naplánovat produkci armády v předstihu, i když v daném okamžiku nevlastní dostatečné suroviny. Naopak při diskrétní produkci je hráč nucen čekat do chvíle, kdy má všechny suroviny, a až poté může začít s produkcí.

Možnosti produkce jednotek také mění taktické volby hráče. Jednotky nemají hodnotu pouze podle své výkonosti v boji, ale také podle doby a ceny produkce. Hráč je mnohem ochotnější obětovat velké množství levných, rychle obnovitelných jednotek než drahých a vzácných jednotek. Hodnota jednotky také může být pro každého hráče různá, podle jeho aktuální ekonomické situace, dostupnosti surovin a podle strategie zvolené nepřítelem. Čas pro produkci jednotek omezuje rychlost změny strategie v reakci na nepřítelovu strategii.

Poměr ceny produkce jednotek a ceny vyzkoumání nových jednotek je jedním z hlavních faktorů určujících strategické možnosti hráče. Při levých jednotkách a drahém postupu je nejlepší strategií výroba velkého počtu dostupného typu jednotek místo odemykání nových silnějších jednotek. To poté vede k tomu, že ve velké většině her se hráči nikdy nedostanou k vývoji nových jednotek a hra skončí porážkou jednoho z hráčů pomocí starších jednotek. Oproti tomu při velké ceně produkce oproti levnějšímu odemykání nových silnějších typů probíhá hra často bez větší konfrontace hráčů až do vyzkoumání posledního stupně jednotek, po kterém pak hráči mohou dedikovat všechny suroviny pro výrobu.

Počet jednotek je často limitován, jak pro účely vyvážení hry, tak pro omezení zátěže hardwaru. Z hlediska vyvážení síly jednotek umožňuje limit na počet jednotek předejít tzv. "Zergu", kdy hráč vytvoří obrovské množství levných jednotek, které následně převálcují jakýkoli odpor. Z hlediska hardwarové náročnosti je účel limitu vcelku zřejmý, protože každá jednotka zabírá určité množství paměti a výpočetního výkonu.

Hráč často začíná s malým počtem jednotek, jejichž účelem je zamezit tzv. Rush strategii, ve které je cílem vytvořit co nejrychleji co nejvíce levných jednotek a zničit nepřítele ještě před tím, než je schopen začít produkovat své jednotky.

Ve hře Stronghold začíná hráč hru s malým počtem jednotek z výše uvedeného důvodu. Ve hře Stronghold Crusader, s kterou je autor nejvíce seznámen, jsou jednotky dvojího typu. Prvním typem jsou arabské jednotky, produkované v arabských kasárnách, které stojí pouze zlaťáky. Druhým typem jednotek jsou křižácké jednotky, produkované v křižáckých kasárnách, které jsou o řád levnější ve zlaťácích než jejich arabské protějšky, ale navíc stojí zbraně a brnění, které je potřeba vyrábět. Arabské jednotky je bohatý hráč schopný rekrutovat velice rychle, a tím vytvořit velkou armádu ve velice krátkém čase. Křižácké jednotky jsou o něco silnější, ale jejich produkce je omezena výrobou zbraní a brnění, není tedy možné bez předchozí přípravy rychle vytvořit armádu křižáckých jednotek. Toto je jeden z příkladů strategického rozhodnutí, které musí hráč učinit, a to zda bude produkovat ve zlatě levnější, silnější, ale časově a surovinově náročnější křižácké jednotky, nebo se bude soustředit na vydělávání zlata a rychlou produkci slabších arabských jednotek. 

Jednotky mají jako ve většině RTS her určitý počet hit pointů, které s každým obdrženým zraněním ubývají, dokud není jednotka zabita. Poškození je určeno typem jednotky, kde každá jednotka má skrytý modifikátor poškození, které obdrží z různých zdrojů.
\todo{popsat concrete a abstract balancing výše} Toto je příklad abstraktního balancingu.

Počet jednotek je omezen na 1000 pro každého hráče, především pro omezení hardwarové náročnosti. 

\subsection{Budovy}
Stavba budov představuje jednu z hlavních prezentací hráčovi strategie. Podle postavených budov lze často vcelku přesně odhadnout, jakou strategii hráč zvolil, čímž je umožněno nepřátelům reagovat a adaptovat svou strategii odpovídajícím způsobem. 

Budovy jsou často uspořádány do postupně se zlepšujících úrovní, které jsou zpřístupňovány výzkumem nových technologií. \todo{see Výzkum technologií} Každá z úrovní obsahuje řadu rozdílných budov, umožňujících zvolit různé strategie. 

Při této volbě lze ale narazit na problém, kdy je hráč nucen zvolit svou strategii před tím, než nalezne protivníky a tedy před tím, než může vidět jejich strategii. Tento problém je velmi výrazný při tzv.  ``rock-paper-scisors'' strategiích, kde strategie 1 poráží strategii 2, strategie 2 poráží strategii 3 a strategie 3 poráží strategii 1. V tuto chvíli hra degeneruje v loterii, zda hráč náhodně vybere správnou strategii porážející tu vybranou nepřítelem. Pro prevenci tohoto problému mají často  počáteční úrovně strategií pouze malé rozdíly v síle, což umožní hráčům reagovat a změnit svoji strategii před tím, než je rozdíl mezi jejich silami neúnosně velký. \cite{http://www.oxeyegames.com/rts-game-play-part-3-build-options/}

Funkce budov:
\begin{itemize}
	\item Produkce jednotek
	\item Vylepšování jednotek
	\item Produkce surovin
	\item Uskladnění surovin
	\item Obrana
	\item Stavba budov
	\item Zkoumání technologií
\end{itemize}

\todo{rozepsat funkce}
\todo{restrikce na umístění}
\todo{zničitelnost}
\todo{různé druhy poškození}

Ve hře Stronghold jsou všechny budovy dostupné od počátku hry. Budova může mít restrikce na místa, kde je dovoleno ji postavit. Tyto restrikce mohou být na typ terénu, sousedící budovy či vzdálenost od hráčovi tvrze. Pro stavbu většiny budov jsou potřeba suroviny, uskladněné v hráčově skladišti, zbrojírně či sýpce. Až na výjimky lze budovy poškodit a následně zničit, některé všemi druhy jednotek, jiné mají omezení na druh jednotek, který je schopen je poškodit. Například hradby a věže je možné ničit pouze pomocí obléhacích zařízení. 

\subsection{Suroviny}
Resource management je přítomný ve všech hrách tohoto žánru již od jeho vzniku. Od koření v Dune II, přes zlato a dřevo ve Warcraft 3, po všechny typy surovin ve hře Stronghold, získávání surovin je jednou z hlavních motivací konfliktu v RTS hrách. 

Systémy surovin lze rozdělit podle způsobu získávání a počtu typů surovin.

Podle způsobu získávání můžeme systém surovin rozdělit na
\begin{enumerate}
	\item Aktivní získávání surovin
	\item Pasivní získávání surovin
\end{enumerate}

Při aktivním získávání surovin existuje ovladatelná herní entita, která svým pohybem mezi pozicemi na mapě přináší suroviny. Tento pohyb může být ovládán hráčem, ale nejčastěji dokáže pracovat jednotka samostatně. Příkladem může být Warcraft, kde speciální jednotky získávají dřevo a zlato přenášením z lesů/dolů do hráčovy hlavní budovy.

Při pasivním získávání přibývají suroviny bez akcí entit, pouze díky vlastnictví určité části mapy nebo druhu budovy. Zdroj bývá nekonečný nebo skoro nekonečný, poskytující suroviny do obsazení nebo zničení zdroje.

Každý z těchto stylů podporuje jinou strategii kontroly mapy. 


Při aktivní získávání surovin je celkový zisk nejčastěji omezen počtem jednotek pracujících na dané surovině, což umožňuje hráči bránit pouze několik málo pozic, mezi kterými se tyto jednotky pohybují, čímž umožňuje hráči postavit svoji ekonomiku z velké části bez ohledu na akce nepřítele.

Pasivní získávání surovin naopak nutí hráče obsazovat a kontrolovat mnoho částí mapy, které slouží jako zdroje suroviny, jak pro svoje vlastní potřeby surovin, tak pro odepření daného zdroje nepříteli. Toto vede k velice hektickému průběhu hry, kdy hráč paralelně vede souboj na mnoha frontách pro obsazení co největší části mapy a získání výhody v počtu surovin.

Tyto dva styly systému získávání surovin lze také kombinovat. Získání pasivního zdroje může sloužit pro uvolnění zdrojů dedikovaných pro aktivní získávání surovin, naopak ztráta pasivního zdroje může být nahrazena zvýšením výkonu aktivního zdroje.


Větší počet druhů surovin by měl hráči umožnit strategicky se rozhodnout, které suroviny bude získávat a které bude ignorovat. Toto rozhodnutí následně ovlivní typy jednotek a budov, které hráč bude stavět a tím i celou strategii dané hry. Zde může nastat problém, kdy hra nutí hráče získávat všechny druhy surovin a neumožňuje hráči vybírat si. Tím je ztracena výhoda více surovin a strategie při hraní této hry budou velice podobné strategiím pro hry s jedním druhem surovin.
\subsection{Vývoj technologií}
Volba vyzkoumaných technologií důležitou součástí strategické části RTS her. 

Technologie jsou často uspořádány ve stromové struktuře, kde vyzkoumání technologie v rodičovském uzlu odemyká technologie ve svých synech. Každé větvení představuje možné rozhodnutí hráče, kterou z větví bude dále zkoumat. V některých hrách bývají dokonce větve výlučné, tedy vyzkoumání jedné větve zamkne přístup k vyzkoumání jiné větve.

Vyzkoumání technologie může mít mnoho různých efektů. Nejčastějším efektem bývá odemknutí nového typu jednotek nebo budov. Další možností je změna vlastností již vlastněných jednotek nebo budov. V neposlední řadě pak může vyzkoumání technologie odemknout nové schopnosti nebo kouzla, které hráč může následně použít při taktických soubojích. 

Explicitní strom technologií, přítomný např. ve hrách série Civilisation, se v RTS hrách vyskytuje spíše výjimečně. Nejčastěji je odemykání nových typů jednotek a budov umožněno stavbou určitého typu budovy nebo dosažení určitého stupně vylepšení již existující budovy. Jako příklad můžeme vzít Warcraft 3, kde závislosti budov na stupních vylepšení a existenci jiných budov tvoří strom technologií. 

\begin{figure}[h]
	\caption{Výřez stromu technologií ze hry Civilisation V}
	\centering
	\includegraphics{civ5_tech_tree}
\end{figure}


\begin{figure}[h]
\caption{Budovy a závislosti mezi nimi tvořící obdobu stromu technologií ve hře Warcraft 3}
\centering
\includegraphics{warcraft_tech_tree}
\end{figure}

Ve hře Stronghold vývoj technologií také není explicitně přítomen, přístup k novým typům jednotek a budov je místo toho znemožněn dostupností surovin a nutností výstavby řetězce pro produkci určitých druhů surovin. 

\section{Požadavky}
Žánr RTS her, jak jsme jej popsaly v předešlé sekci, je pro tvorbu jednoho frameworku podporujícího vývoj všech RTS her příliš rozsáhlý a zahrnuje hry s příliš rozdílnými požadavky na vývoj. Proto jsme se rozhodli v naší práci implementovat framework umožňující vývoj podmnožiny žánru RTS, a to 3D RTS her pro jednoho hráče s mapou rozdělenou na homogenní čtvercové dlaždice. 

Náš framework bude tedy poskytovat prvky společné hrám tohoto druhu, přičemž bude umožňovat vývojáři hry co největší možnosti pro rozšíření poskytovaných prostředků s 
co možná nejmenším omezením na mechaniky her vyvíjených za pomoci našeho frameworku.
 
Hlavním cílem našeho framework je možnost využití .NET Frameworku pro vývoj pluginů, které poté spolu s assety a metadaty vytvoří balíček. Tento balíček poté hráč bude moci přidat do své hry, čímž získá přístup k novým jednotkám, budovám, mapám, typům dlaždic, surovin a oponentů. 

Pro implementaci jsme so rozhodli použít UrhoSharp engine, nad kterým postavíme náš framework a pokusíme se zjednodušit využití tohoto enginu pro tvorbu našeho poddruhu RTS her. 
\begin{enumerate}
	\item Z pohledu tvůrce balíčku bude náš framework sloužit jako knihovna, spouštějící tvůrcovi funkce v reakci na události nastávající ve~hře. Dále bude poskytovat implementaci herní mapy a centrálního registru všech entit, do kterých se logiky jednotek, budov a hráčů budou schopny dotazovat na aktuální stav hry. Pro předejití opakované implementace každým tvůrcem her bude dále poskytovat základní komponenty, umožňující pohyb jednotek po mapě, pohyb projektilů po balistických křivkách a výpočty cílů podle pohybu jednotek, automatickou střelbu na cíle, \todo{popsat všechny DefaultComponenty}. V neposlední řadě bude umožňovat ukládání a zpětné načítání aktuálního stavu hry, spolu s rozhraním pro ukládání a zpětné načítání stavu logiky definované tvůrcem hry. 
	
	Dále framework z pohledu tvůrce balíčku bude sloužit jako spustitelný software, poskytující editor mapy, spolu se základní implementací nástrojů pro její editaci, umožňující změnu výšky jednotlivých rohů dlaždic, dlaždic jako celku nebo i skupin dlaždic. Dále bude umožňovat takto vzniklý hrubý terén zarovnat. Spolu s těmito nástroji pro změnu tvaru terénu bude umožňovat změnu typu na všechny typy dlaždic definované v aktuálně používaném balíčku. Tvůrcům balíčků bude umožněno odebírat tyto základní implementace a přidávat odvozené nebo úplně nové editační nástroje, které pak budou schopni použít při tvorbě map v našem editoru.
	
	Formát uložení hry bude veřejně známý, bude tedy možné vytvářet nezávislé nástroje pro tvorbu a editaci map. 
	\item
	Z pohledu hráče bude framework jako samostatný spustitelný software umožňovat načítání balíčků, editaci map připravených tvůrcem balíčku, vytváření úplně nových map používajících jednotky, budovy, oponenty a typy dlaždic poskytnuté tvůrcem balíčku a následné spuštění všech takto získaných map. V průběhu hry potom bude umožňovat uložení aktuálního stavu hry a následné načtení tohoto stavu.
\end{enumerate}




Pro ukázku bude vytvořena jednoduchá hra, demonstrující možnosti našeho frameworku. \todo{Popsat hru}

\section{Cíle práce}
Cílem této práce je vytvořit framework pro vývoj 3D RTS her pro jednoho hráče, umožňující vývojářům vytvářet hry jako separátně 
distribuované balíčky, které bude poté koncový uživatel schopen připojit do našeho frameworku nainstalovaném na uživatelově počítači.

Při tvorbě balíčků bude umožněno tvůrci použít .NET Framework pro vytvoření Umělé inteligence jednotek, budov a AI hráčů, dále pro přidání nástrojů do editoru map a \todo{ Nespecifické } kontrolu průběhu hry.

Požadované vlastnosti frameworku:
\begin{enumerate}
	\item Vlastnosti pro tvůrce balíčků:
		\begin{enumerate}
			\item Framework musí umožňovat přidávání balíčků za běhu, obsahujících nové typy jednotek, budov,  dlaždic, projektilů a hráčů spolu s jejich modely, texturami a AI.
			\item Framework musí umožňovat použití přidaných balíčků pro tvorbu map a uložení vytvořených map do balíčku použitého pro jejich vytvoření
			\item Editor map musí být rozšiřitelný pomocí pluginů z balíčku.
			\item Herní user interface musí umožňovat rozšiřitelnost pro tlačítka, okna a další grafické prvky přidávané tvůrcem.
		\end{enumerate}

	\item Vlastnosti pro koncového hráče:
		\begin{enumerate}
			\item User interface pro stolní počítače, umožňující vybírání balíčků, map a oponentů, dále načítání a ukládání her, a nastavování zobrazení hry.
			\item Herní user interface musí obsahovat minimapu, poskytující hráči přehled o větší části mapy než kterou vidí vlastní kamerou.
			\item Ovládání kamery umožňující klasický top-down pohled, volné poletování kamery po mapě a následování jednotky
			\item Ukládání a načítání hry
		\end{enumerate}
\end{enumerate}

\chapter{Analýza}
V~první kapitole jsme specifikovali cíl naší práce, tedy implementaci platformy založené na herním enginu UrhoSharp umožňující tvorbu RTS her a~jejich distribuci. V~této kapitole popíšeme problémy při implementaci této platformy, námi zvolená řešení a~jejich alternativy.

\section{Herní engine}
Jak jsme napsali v~sekci \ref{sec:cileprace} Cíle práce, naším cílem je vytvořit platformu pro tvorbu RTS her za použití herního enginu UrhoSharp. \textit{\uv{UrhoSharp je multiplatformní 3D a~2D engine který může být použit pro tvorbu animovaných 3D a~2D scén za použití modelů, materiálů, světel a~kamer} } \citep{site:urhosharp}, jak říká úvodní stránka dokumentace enginu. Jak už název napovídá, UrhoSharp je .NET binding pro Urho3D engine \citep{site:urho3D}, což je opensource herní engine implementovaný v~C++.

Tento engine a~jeho binding do jazyka C\# jsme si vybrali především kvůli tvůrcům .NET bindingu, společnosti Xamarin, která je také autorem implementace .NET Frameworku Mono. V~rámci tohoto vztahu očekáváme největší podporu práce s~managed kódem, například jeho načítáním za běhu, na čemž je postaven náš systém pluginů. 

Jak uvidíme v~následující části, přestože je engine UrhoSharp navržen pro platformu Mono, nebylo na všech systémech možné překonat jejich omezení.

\section{Rozdíly systémů}
\label{sec:system_dif}
Hlavním cílovým systémem naší práce jsme se rozhodli zvolit systém Windows, především kvůli nejrozsáhlejší podpoře frameworku .NET, dále kvůli zřejmým výhodám ovládání pomocí klávesnice a~myši, a~v~neposlední řadě kvůli naší zkušenosti s~tímto systémem. Pro podporu dalších platforem je naším cílem vytvořit návrh platformy umožňující co nejjednodušší rozšíření na tyto systémy. Herní engine a~platforma .NET sice mnohé rozdíly systémů abstrahují a~umožňují multiplatformní řešení, existují ovšem oblasti, které i~při využití těchto abstrakcí vyžadují pro každý systém specifické řešení.  

Implementace pro mobilní systémy (iOS, Android) naráží oproti PC systémům na několik problémů, vycházejících především z~rozdílných operačních systémů, velikostí obrazovek a~způsobu přijímání vstupu od uživatele. Rozdíly mezi PC systémy (Windows, různé distribuce Linuxu, macOS) nejsou tak rozsáhlé, přesto se mohou vyskytnout problémy především kvůli různým implementacím platformy .NET používaných mimo systém Windows.

Při řešení těchto problému jsme narazili na rozdíly, specifika a~omezení, které v~následujících částech přiblížíme.

\subsection{Zobrazení a~ovládání}
Na mobilních systémech je vztah mezi GUI, tedy grafickým uživatelským rozhraním, a~ovládáním mnohem bližší. Oproti PC systémům je zde nejčastěji jediným možným vstupem dotyková obrazovka. GUI musí tedy sloužit jak pro zobrazení informací hráči, tak pro získání převážné většiny vstupu od hráče. Dalším rozdílem je velikost obrazovky, která je obecně mnohem menší než u~jiných systémů. Ze statistik zařízení od společnosti Google \citep{site:materialdesign} můžeme vidět, že rozpětí velikostí obrazovek tzv.~\uv{smartphonů} používajících systém Android sahá od úhlopříčky 3,5~palce po 6,4~palce, kde nejčastější velikostí je rozpětí od 5~palců po 6~palců. Oproti tomu nejčastější velikosti obrazovek laptopů sahají od 11~in po 17~in a~více. 

I~přes to, že herní engine umožňuje tvorbu grafického rozhraní použitelného na všech systémech, různé druhy vstupu a~velikostí obrazovek nutí nás i~potencionální tvůrce her na naší platformě k~výrazně odlišnému návrhu rozhraní pro mobilní systémy. Engine Urho3D poskytuje separátní vývojové prostředí pro návrh scén a~uživatelského rozhraní, které je následně možné exportovat do XML souboru, který je poté možné načíst za běhu hry pro zobrazení specifikovaného uživatelského rozhraní. Tento postup lze použít k~definici uživatelského rozhraní specifického pro cílový systém dané verze aplikace, tedy naší platformy. Oproti tomu by tvůrci balíčků byli nuceni distribuovat definice rozhraní pro všechny systémy, z~kterých by poté za běhu vybírali podle aktuálního systému, nebo by byli nuceni vytvářet a~distribuovat několik separátních balíčků, cílených vždy pro jediný systém. 

Ukázku těchto problémů a~jedno z~možných řešení můžeme vidět na příkladu ze hry Hearthstone \citep{site:hearthstone}, kde  \ref{fig:hearthstone_mobile} ukazuje mobilní verzi hry a \ref{fig:hearthstone_pc} ukazuje PC verzi hry. Přestože návrh této hry je ideální pro přenos na mobilní zařízení, což můžeme vidět například ve velice podobném designu vlastní herní plochy v~obou verzích, existují mezi PC a~mobilní verzí viditelné rozdíly. Jedním z~rozdílů je pozice kamery, která je v~mobilní verzi umístěna mnohem blíže herní ploše. Toto je jedno z~možných řešení problému menší velikosti obrazovek mobilních zařízení, díky kterému budou herní prvky na těchto obrazovkách zobrazeny ve větší velikosti za cenu zobrazení menší části herního světa. Další součástí řešení problému velikosti obrazovek je velikost fontu, která je v~mobilní verzi mnohem větší než u~PC verze. Zároveň s~touto změnou musí být také provedena odpovídající změna designu karet a~oblastí, kde se písmo a~číslice vyskytují. Řešení problému vstupu pomocí dotykové obrazovky můžeme vidět na obrázku \ref{fig:hearthstone_mobile}, který ukazuje dva různé stavy hry. První stav, který můžeme vidět na obrázku \ref{fig:hearthstone_mobile_closed}, je navržen zobrazení celkového stavu hry a~pozorování nepřátelských tahů. V~tomto stavu jsou hráčovy karty zmenšené a~schované v~pravém dolním rohu obrazovky, kde nezakrývají žádnou část herní plochy. Při kliknutí na karty následně rozhraní přechází do druhého stavu, který můžeme vidět na obrázku \ref{fig:hearthstone_mobile_open}. V~tomto stavu jsou karty hráče přemístěny do centrální pozice a~zvětšeny. Tímto přemístěním sice zakryjí velkou část herní plochy, ale v~danou chvíli jsou právě tyto karty cílem hráčovi pozornosti a~jimi zakrytá plocha je pro něj irelevantní. Tento systém vytahovaní, přemisťování a~zvětšování ovládacích prvků aplikace je obecným trendem v~mobilních zařízeních, umožňujícím větší velikost ovládacích prvků za cenu většího počtu interakcí uživatele se zařízením oproti počtu interakcí v~PC verzi aplikace pro dosažení stejného cíle. Toto můžeme vidět na obrázku \ref{fig:hearthstone_pc}, který ukazuje stejnou situaci v~PC verzi hry. Mezi \ref{fig:hearthstone_pc_closed} a \ref{fig:hearthstone_pc_open} nejsou karty nijak přesouvány či zvětšovány a~jsou viditelné neustále, hráč tedy může zahrát kartu bez předchozího kliknutí na schované karty, čím ušetří oproti mobilní verzi jednu interakci.
\begin{figure}[!tbp]
	\centering
	\subfloat[Bez interakce hráče.]{\includegraphics[width=0.47\textwidth]{hearthstone_mobile_closed}\label{fig:hearthstone_mobile_closed}}
	\hfill
	\subfloat[Při interakci hráče.]{\includegraphics[width=0.47\textwidth]{hearthstone_mobile_open}\label{fig:hearthstone_mobile_open}}
	\caption{Uživatelské rozhraní mobilní verze hry Hearthstone.}
	\label{fig:hearthstone_mobile}
\end{figure}

\begin{figure}[!tbp]
	\centering
	\subfloat[Bez interakce hráče.]{\includegraphics[width=0.47\textwidth]{hearthstone_PC_closed}\label{fig:hearthstone_pc_closed}}
	\hfill
	\subfloat[Při interakci hráče.]{\includegraphics[width=0.47\textwidth]{hearthstone_PC_open}\label{fig:hearthstone_pc_open}}
	\caption{Uživatelské rozhraní PC verze hry Hearthstone.}
	\label{fig:hearthstone_pc}
\end{figure}

Jak můžeme vidět na příkladu ze hry Hearthstone \citep{site:hearthstone}, vedou nás problémy s~velikostí obrazovek a~dotykovým ovládáním k~separátnímu designu a~implementaci uživatelského rozhraní a~některých částí her. Pro tuto separátní implementaci jsme v~naší práci připravili základní kostru, upustili jsme ovšem od konečné implementace z~důvodu nedostatku času.

\subsection{Kompilace}
Dalším rozdílem, tentokrát s~rozdílným chováním i~mezi jednotlivými mobilními systémy, je jejich chování k~spustitelným souborům aplikací. Jak píší Joseph a~Ben Albahari \citep[str.~3,4]{book:cs7nutshell}, jsou jazyky cílené na platformu .NET překládány do \uv{Common Intermediate Language}  (CIL), z~kterého jsou obvykle až za běhu aplikace kompilovány do instrukční sady stroje, na kterém právě běží. Tento způsob se označuje jako \uv{Just-In-Time}  (JIT) kompilace, a~je standardním způsobem spouštění .NET aplikací. V~některých případech je ale použit jiný způsob, a~to tzv.~\uv{Ahead-of-time}  (AOT) kompilace, kdy je CIL kód ještě před distribucí zákazníkovy zkompilován do instrukční sady cílového stroje a~následně je distribuována tato již zkompilovaná verze. Tento způsob je používán pro zrychlení odezvy při větších velikostech assembly, čímž se předchází zpoždění v~důsledku kompilace CIL kódu. Další využití, zde již ne pouze za účelem optimalizace, ale vynucené systémem samotným, je při distribuci na systém \emph{iOS} \citep{site:aot}. Jak je napsáno v \textit{iOS Security Guide pro iOS verze 12.1} \citep[str.~27]{book:iossecurityguide}, není možné alokovat paměť zároveň jako \textit{\uv{writable} }, tedy s~možností do ní zapisovat, a \textit{\uv{executable} }, tedy s~možností v~ní uložená data vykonávat přímo jako instrukce procesoru. Tato skutečnost vylučuje jakékoli použití JIT kompilace, která používá právě takto namapované stránky jako výstup kompilace z \textit{intermediate} jazyka, tedy například CIL, do instrukční sady procesoru. Výjimkou jsou aplikace společnosti Apple, podepsané jejich klíčem, kterým je umožněna jedna alokace (jedno zavolání funkce \texttt{mmap}) takto namapovaných stránek. Tato výjimka je použita ve webovém prohlížeči Safari pro implementaci Javascript JIT kompilátoru.  

Tato skutečnost znemožňuje naší platformě jednoduché nahrání assembly pomocí reflexe a~nutila by nás k~složitějšímu řešení, které by se podle aktuálního systému muselo rozhodovat, kterou verzi assembly nahrát. Navíc by tento způsob nutil tvůrce balíčků přeložit svůj kód pro všechny možné architektury. Toto je důvod, proč jsme upustili od podpory systému iOS.

\subsection{Souborové systémy}
\label{sec:filesystems}
Každá aplikace má několik odlišných druhů souborů. Tyto druhy můžeme odvodit z~pro ně navržených adresářů ve Windows API \citep{site:knownfolders}, či z~implementace tohoto API v~platformě .NET \citep{site:specialfolders}. Těmito druhy souborů jsou:
\begin{itemize}
	\item \textit{Roaming user data} - data uživatele přítomná na všech počítačích v~síti, na které se může uživatel přihlásit;
	\item \textit{Local user data} - data uživatele lokální pro aktuální počítač;
	\item \textit{Private app data} - data aplikace přístupná pouze aplikaci;
	\item \textit{Public app data} - data aplikace přístupná všemi aplikacemi;
	\item \textit{Static app data} - neměnná data aplikace distribuovaná spolu s~aplikací.
\end{itemize} 

Některé z~těchto druhů souborů jsou na různých systémech sloučeny. Systém Android například nepodporuje více uživatelů, čímž ztrácí smysl rozdělovat uživatelská data a~data aplikace. Přístup aplikací k~souborovému systému je na tomto systému dále omezen. Jsou definována čtyři místa, kam může aplikace ukládat data \citep{site:androiddata}. Těmito místy jsou:

\begin{enumerate}
	\item \textit{internal file storage},
	\item \textit{external file storage},
	\item \textit{shared preferences},
	\item \textit{databases}.
\end{enumerate}

\textit{Internal file storage} je interní úložný prostor zařízení. Každá aplikace má zde systémem vytvořenou složku, do které má přístup pouze aplikace samotná. Tato složka je odstraněna při odinstalování aplikace.

\textit{External file storage} je \uv{externí}  úložný prostor, není tedy garantováno, že bude vždy přítomný. Tento prostor může být tvořen jak vestavěnou pamětí zařízení, tak fyzicky vyjímatelným prvkem, jako například SD kartou. Soubory na tomto úložišti jsou veřejně přístupné a~nejsou odstraňovány při odinstalování aplikace. 

\textit{Shared preferences} a \textit{Databases} slouží pro ukládání dat bez explicitního využití souborových systémů. Tento přístup k~datům nemá na PC systémech přímou podporu.


Dalším rozdílem mezi systémy je přístup k~souborům distribuovaným spolu s~aplikací, v~našem případě s~naší platformou. Na systému Android je každá aplikace distribuována jako \texttt{.apk} soubor. Formát Apk je zip archiv, obsahující všechny soubory naší aplikace, od kódu, přes assety, po preference. Tento archiv je v~duchu Linuxového VFS přímo namapován do stromu souborového systému viditelného z~naší aplikace. Bohužel .NET filesystem API nedokáže s~tímto mapováním pracovat, tedy není možné ho využít pro čtení těchto souborů. Řešením je využití Xamarin.Android zabalujícího Android Java API, které s~tímto archivem pracovat dokáže.

Přístup k~adresáři aplikace má další rozdíl, a~to v~zápisu do souborů. Na systému Android je zápis do těchto souborů úplně zakázán. Na PC systémech může být pro některé uživatele možné zapisovat do těchto souborů, ale aplikace by s~tímto přístupem neměla počítat.

Všechny tyto rozdíly při přístupu k~souborům nás nutí k~implementaci separátní komponenty pro práci se soubory, která je implementována pro každý systém zvlášť a~následně poskytována přenositelné části platformy.

\subsection{Shrnutí}
I~když je naším cílem pouze implementace pro systém Windows, ukázali jsme, že rozšíření na některé další systémy by nebyl problém a~nastínili jsme řešení možných problémů, které by při tomto rozšíření vznikly. Návrh naší implementace zohledňuje tato řešení a~umožňuje jejich  budoucí implementaci. Vzhledem k~velikosti implementace pro systém Windows nebudou ovšem tato řešení pro zbylé systémy součástí naší práce. Dále jsme zjistili, že systém iOS je neslučitelný s~požadavky naší aplikace a~rozšíření na tento systém tedy nebude možné.

\section{Formát a~načítání dat}
Důležitou součástí implementace naší platformy je systém balíčků pro distribuci vytvořených her. Tyto balíčky obsahují všechny součásti hry, od modelů a~textur, přes logiku a~umělou inteligenci, po mapy a~úrovně vytvořené tvůrcem hry. Všechny tyto součásti musí naše platforma být schopna načíst za běhu a~použít jak pro tvorbu nových map, tak pro hraní již existujících.

\subsection{Struktura balíčku}
\label{sec:packagestructure}
Pro implementaci načítání balíčků musíme definovat strukturu, kterou budou balíčky splňovat, a~podle které bude platforma určovat typy souborů a~jejich závislosti.

První možností je založit strukturu balíčku na pevné adresářové struktuře, kde každý balíček bude tvořen jedním adresářem obsahujícím další pevně specifikovanou podadresářovou strukturu. Jednotlivé typy zdrojů, tedy 3D modely, textury, popis jednotek nebo skripty, by následně byly rozděleny a~identifikovány touto adresářovou strukturou. 

Pro popis typů jednotek, budov, projektilů, dlaždic a~nepřátel jsme se inspirovali v~existujících hrách, ať už Civilisation~V \citep{site:civ5} nebo Kerbal Space Program \citep{site:ksp}, a~využili jsme XML soubor pro definici závislostí. Dalšími možnostmi pro popis typů entit bylo využít formát JSON nebo dokonce definovat vlastní formát. Oproti ostatním zmíněným formátům má formát XML vestavěnou podporu přímo v~platformě .NET. Navíc tento formát umožňuje automatickou validaci vůči schématu, což nám ulehčí  validaci načtených dat. Možnou výhodou formátu JSON je jeho expresivita, umožňující minimalizovat velikost souborů. Vzhledem k~velikosti ostatních druhů dat jsme ovšem usoudili, že tato výhoda není dostačující pro volbu tohoto formátu. 

Implementace pomocí pevné adresářové struktury a~XML souborů pro popis typů entit ovšem vedla ke dvěma problémům. Prvním bylo velké množství malých XML souborů, jejichž správa by mohla vést k~častým omylům a~následným chybně pojmenovaným souborům, špatným cestám a~chybějícím souborům. Druhým problémem bylo přidávání balíčku do běžící hry. Hráč by mohl sice specifikovat adresář reprezentující balíček, následné ověření, zda je tento balíček korektní a~lze ho nahrát by nás ale nutilo k~procházení adresářové struktury, k~pokusům o~jejich načtení a~validaci. Takováto implementace by byla pomalejší, náročnější na správu a~náchylnější k~chybám. 

Řešením bylo vytvořit centrální XML soubor, definující celý balíček. Všechny typy jednotek, budov, projektilů, nepřátel, logik úrovní, všechny úrovně obsažené v~balíčku a~další jsou popsány v~tomto souboru pomocí stejného XML, jako byly v~separátních souborech. Následně všechny assety, tedy modely, textury či assembly mohou být specifikovány relativní cestou vůči adresáři obsahujícímu tento jeden XML soubor. Tímto způsobem lze replikovat předchozí uspořádání, kde je každý typ assetů rozdělen do vlastního adresáře, ale navíc tento způsob umožňuje tvůrci balíčku specifikovat vlastní rozdělení a~umístění assetů. Zároveň toto uspořádání ulehčuje přidání balíčku a~ověření jeho korektnosti, kde stačí, aby uživatel zadal cestu k~tomuto XML souboru, a~pouhou validací podle schématu lze ověřit jeho správnost.

Naším finálním řešením je tedy reprezentovat každý balíček jedním XML souborem, který dále obsahuje relativní cesty odkazující na zbylý obsah balíčku. Tento soubor má formát daný pevným schématem a~tento formát je kontrolován při každém načítání. 

\subsection{Data entit}
Nyní již víme, jak data rozdělit a~odkazovat na ně. Následně musíme určit, jaká data budeme u~jednotlivých typů entit požadovat a~jaká vlastní data umožníme tvůrcům her si zde uložit. Naším cílem je umožnit specifikování typů entit a~k~nim náležejících dat, definujících vlastnosti těchto typů entit. 

Naše platforma bude poskytovat funkcionalitu společnou většině her typu specifikovaného v~úvodní části \ref{sec:uvod}. 
Tato funkcionalita zahrnuje:
\begin{enumerate}
	\item načítání pluginů a~volání jejich metod,
	\item uživatelské rozhraní,
	\item správa balíčků,
	\item vykreslování herního světa.
\end{enumerate} 

Naše platforma bude vynucovat specifikaci nezbytných dat pro implementaci poskytované funkcionality. Těmito daty jsou:
\begin{enumerate}
	\item assembly;
	\item ikony, barvy pro zobrazení na minimapě;
	\item identifikátory a~jména balíčků, typů entit, úrovní, logik hráčů;
	\item 3D modely, textury, animace.
\end{enumerate}
Protože jsou tyto vlastnosti společné většině entit v~námi podporovaných typech her, nahrává naše platforma automaticky zde specifikované vlastnosti při vytvoření entity v~herním světě a~následně je využívá pro implementaci poskytovaných herních prvků.

Dále umožníme přidat libovolná další data do XML elementu reprezentujícího typ entity pro použití tvůrcem hry v~jeho implementaci logiky. Tato data nebude naše platforma nijak validovat, pouze je při vytvoření entity v~herním světě předá načítanému pluginu implementujícímu logiku vytvářené entity. Bude pouze na tvůrci této logiky, aby získaná data validoval a~následně z~nich inicializoval logiku či použil systém výjimek k~oznámení chyby v~datech. Tento způsob implementace umožní tvůrcům vytvářet universálnější logiku, kterou budou schopni odlišit právě načítanými daty ze souboru. Bez této funkcionality by byli tvůrci her nuceni všechna data přesunou z~XML souboru přímo do kódu logiky.

\subsection{Formáty assetů}
V~předešlé části jsme popsali, jak náš systém balíčků umožňuje zaznamenat umístění dat potřebných pro spuštění úrovně. V~této části dále upřesníme, jakých formátů můžou tato data být a~jaká jsou omezení při použití určitých formátů dat. Slovem \textit{\uv{asset} } označujeme jakákoli data, které může tvůrce hry poskytnout naší platformě a~použít je při tvorbě hry. Tato data lze rozdělit do několika kategorií:
\begin{itemize}
	\item grafická data,
	\item logika a~pluginy,
	\item ostatní.
\end{itemize}

V~následujících částech popíšeme jednotlivé kategorie dat, jejich formáty a~využití.

\subsubsection{Grafická data}
Mezi grafická data patří především 3D modely a~k~nim náležící textury a~animace. Podporované formáty těchto dat jsou omezeny námi používaným herním enginem Urho3D. Tento engine interně používá knihovnu Open Asset Import Library (Assimp) \citep{site:assimp}, která umožňuje načítání mnoha formátů 3D modelů do uniformní reprezentace, která je následně používána enginem samotným.

Dalšími grafickými daty používanými naší platformou jsou textury dlaždic. Každý typ dlaždic, jak je popsáno v~části \ref{sec:packagestructure} o~struktuře balíčku, je definován XML elementem. Tento element povinně specifikuje texturu, představující vzhled dlaždic tohoto typu. Kvůli naší implementaci zobrazení mapy, blíže popsaného v~části \ref{sec:mapimpl}, musí být tato textura manipulovatelná z~C\# kódu naší platformy. Bohužel engine Urhosharp v~aktuální verzi neposkytuje přístup k~dekompresi textur, implementované enginem Urho3D. Z~tohoto důvodu není možné využít komprimované formáty textur pro vzhled dlaždic.

Dalšími povinnými grafickými daty, které naše platforma požaduje, jsou textury ikon pro jednotky a~budovy. Pro každý z~těchto dvou druhů entit požadujeme texturu, která je následně použita při implementaci nástrojů  poskytovaných naší platformou, umožňujících editaci a~hraní úrovní. Tyto nástroje umožňují umisťovat jednotky a~budovy do editované úrovně a~dále při jejím hraní vybírat skupiny jednotek a~vydávat jim rozkazy. Definice každého typu jednotky nebo budovy následně obsahuje specifikaci části těchto textur, která má být využita pro reprezentaci tohoto konkrétního typu. Přestože platforma umožňuje tvůrcům her implementovat vlastní nástroje pro editaci a~hraní her, předpokládáme, že i~tyto nástroje budou potřebovat reprezentovat jednotlivé druhy jednotek a~budov a~využijí k~tomu právě tyto ikony.

Grafická data obsažená v~balíčku nejsou omezena pouze na platformou požadovaná a~používaná data, ale je umožněno tvůrci přibalit libovolná data podporovaná enginem UrhoSharp a~následně je využít při hře. Díky tomu, že mají tvůrci her plný přístup k~schopnostem enginu UrhoSharp, mohou, stejně jako platforma, za běhu použít tato data pro úpravu vzhledu jednotek, budov či projektilů, případně pro rozšíření grafického uživatelského rozhraní. Mezi tato data patří například \textit{Shadery} a~animace.

\subsubsection{Logika a~pluginy}
\label{sec:logicandplugins}
Jak jsme uvedli v~části \ref{sec:requirements}, jedním z~hlavních cílů naší práce bylo umožnit využití jazyka C\# pro skriptování logiky hry a~umělé inteligence nepřátel. Díky využití herního enginu UrhoSharp, postaveného nad platformou .NET, máme zajištěno, že v~naší implementaci budeme schopni použít systém \textit{Reflection} pro načítání a~používání pluginů. 

I~v~případě, že by naše práce nevyužívala engine UrhoSharp, či jiný engine postavený na platformě .NET, bylo by možné použít jazyk C\# pro skriptování. Taková implementace, využívající jiný jazyk pro tvorbu platformy, by vyla nucena využít C++/CLI pro vytvoření \uv{mostu}  mezi kódem tvořícím platformu a~managed kódem, umožňujícím nahrávání pluginů za běhu a~jejich využití z~kódu platformy. Tato alternativa je ovšem mnohem složitější pro implementaci, než je tomu při použití enginu založeného na platformě .NET. Při použití herního enginu připraveného pro .NET, jako je právě UrhoSharp, je možné tvůrcům pluginů navíc poskytnout plnou sílu tohoto enginu a~umožnit jim využít všechny jeho schopnosti bez jakéhokoli zásahu či manuálního zprostředkovávání. V~případě využití enginu v~jiném jazyce by musel \uv{most}  explicitně implementovat všechny schopnosti enginu, které by chtěl poskytnout tvůrcům pluginů a~přeposílat každý požadavek vlastnímu hernímu enginu.

Obě předešlá řešení využívají systém \uv{Reflection}  pro načítání tvůrcem dodaných assembly, nalezení tříd odpovídajících jednotlivým jednotkám, vytvoření instancí těchto tříd a~jejich následné použití. Pojmem \uv{Reflection}  označujeme sadu tříd .NET frameworku, umožňující introspekci .NET assembly, poskytující informace o~třídách obsažených v~těchto assemblies, jako například implementovaná rozhraní, poskytované metody či obsažené atributy. \uv{Reflection}  dále umožňuje načítání existujících assembly za běhu programu, či dokonce generování nových assembly. 

Assembly jsou identifikovány jménem, verzí, \textit{culture} a~veřejným klíčem. 
Při načítání je assembly nahrána do tzv.~kontextu. Existují čtyři kontexty, do kterých jsou assembly nahrávány. Těmito kontexty jsou \citep{site:assemblyload}:
\begin{itemize}
	\item \textit{Default Load Context}
	\item \textit{Load-From Context}
	\item \textit{Reflection-only Context}
	\item \textit{No Context}
\end{itemize}

\textbf{Reflection-only} context slouží pro zkoumání metadat assembly pomocí reflection a~znemožňuje vykonání kódu nahraného do tohoto kontextu, proto je pro nás nezajímavý a~dále ho nebudeme rozebírat.

\textbf{Default Load Context} je kontext, ve kterém je nahrána assembly naší aplikace a~všechny její závislosti. Do tohoto kontextu lze manuálně nahrávat další assembly, pokud se tyto assembly nachází v \textit{Global assembly cache} (\emph{GAC}), v~adresáři aplikace (\textit{applicationBase}) nebo v~aplikací specifikovaných podadresářích \textit{applicationBase} (\textit{PrivateBinPath}). Pokud je identická assembly již nahrána, nenahrává se znovu ale vrací se reference na již nahranou assembly. Závislosti nahrávaných assembly jsou automaticky vyhledávány na těchto třech místech.

\textbf{Load-From Context} je kontext, do kterého jsou nahrávány assembly pomocí metody \texttt{Assembly.LoadFrom}. Do tohoto kontextu lze nahrát assembly specifikováním cesty spolu s~výše zmíněnými vlastnostmi identifikujícími assembly, a~tato cesta je přidána jako pátá identifikační vlastnost. Tímto způsobem lze nahrávat assembly ležící mimo \emph{GAC}, \textit{applicationBase} a \textit{PrivateBinPath}. Závislosti jsou hledány nejdříve mezi již nahranými assembly v \textit{Default Load Contextu}, následně v~adresáři, ze kterého byla assembly nahrána a~nakonec na cestách pro nahrávání assembly do Default Load Contextu.

\textbf{No Context} je využíván při načítání  assemblies vygenerovaných pomocí \texttt{Reflection.Emit} a \texttt{Assembly.LoadFile}. Navíc je toto jediný způsob, jak načíst dvě verze té samé assembly. Pod pokličkou je vytvořen každé nahrané assembly zvláštní privátní kontext. Problémem tohoto kontextu je, že nejsou automaticky nahrávány závislosti. Tedy nezbývá nic jiného než závislosti nahrát manuálně, buď před načtením assembly nebo odchycením \texttt{AssemblyResolve} události.

V~naší platformě používáme \texttt{Assembly.LoadFrom}. Tento způsob nám umožňuje nahrávat assembly z~libovolných podadresářů uvnitř tvůrci tvořených balíčků, bez omezení na jejich adresářovou strukturu. Díky načítání závislostí ze zdrojového adresáře assembly mohou tvůrci her přibalit jimi používané knihovny do balíčku a~ty budou následně při použití automaticky načteny.

Alternativou by bylo vynutit tvůrce balíčků specifikovat cesty, ve kterých se mohou vyskytovat assembly, a~tyto cesty následně přidat do PrivateBinPath. Tento přístup by ovšem omezil místa, kde může naše platforma mít umístěné balíčky, na podstrom adresáře, ve kterém je umístěna naše platforma. Jak jsme psali v~části \ref{sec:filesystems} o~rozdílech souborových systémů, některé ze systémů nepodporují změny v~adresářového podstromu aplikace a~není tedy možné do tohoto podstromu umístit balíčky. I~na systémech, na kterých je možné zapisovat do podstromu aplikace, není tento přístup k~souborům považován za dobrý design, jak říká tento, i~když poněkud zastaralý, návod pro umisťování souborů na systému Windows \citep{site:windowsappfiles}. Hlavním důvodem je omezení přístupových práv, které systém uvaluje na standardní cíl instalace aplikací.

Naše aktuální implementace nijak neřeší uvolňování assemblies ve chvíli, kdy hra končí a~hráč načítá jiný balíček. Vzhledem k~velikosti assembly v~paměti by ale toto neměl být při běžných počtech používaných balíčků problém. Vzhledem k~budoucí unifikaci .NET Frameworku, .NET Core a~Mono pod implementací .NET 5 \citep{site:dotnet5}, který je následníkem .NET Core, není v~této chvíli jednoduché zvolit způsob uvolňování nepoužívaných assembly. Aktuální způsob v~námi používaném .NET Framework je použití \texttt{AppDomain}, pomocí které by bylo možné oddělit balíčky od sebe a~umožnit jejich uvolňování. Toto řešení ovšem není podporované v .NET Core, který funkcionalitu uvolňování assembly ze systému \texttt{AppDomain}, který není v~této verzi Frameworku podporován, přemisťuje do třídy \texttt{AssemblyLoadContext} \citep{site:assloadcontext}. Protože .NET 5 vychází z .NET Core, předpokládáme, že by v~budoucnu bylo možné využít právě \texttt{AssemblyLoadContext} pro řešení tohoto problému. 

\subsection{Formát uložených úrovní}
\label{sec:savingformat}
Ukládáním úrovně rozumíme serializaci aktuálního stavu hry a~uložení takto serializovaných dat do souboru. Pro serializaci jsme měli několik požadavků: 
\begin{enumerate}
	\item otevřenost schématu serializovaných dat,
	\item minimalizace velikosti serializovaných dat,
	\item rychlost serializace a~deserializace.
\end{enumerate}

Účelem prvního požadavku je umožnit tvorbu nezávislých editorů úrovní, importujících a~exportujících náš formát dat. Tento požadavek splňují serializace do formátu XML, definovaného pomocí \textit{XSD} schématu, a~binární serializace popsaná pomocí \textit{interface description language (IDL)}. Příkladem takovéto binární serializace je formát Protocol buffers \citep{site:protobuf} od společnosti Google. Z~binárních serializačních frameworků je pro jazyk C\# nejlépe podporován právě formát Protocol buffers, ať už použit sám o~sobě či za podpory knihovny protobuf-net, umožňující automatickou generaci IDL souborů z~anotací ve zdrojovém kódu. 

Druhý a~třetí požadavek nás vedl k~volbě binární serializace, která minimalizuje velikost dat a~má nejrychlejší zpracování. Tuto skutečnost sděluje jak dokumentace Protocol buffers \citep{site:protobufdevguide}, tak ji můžeme vidět experimentálně podloženou v~benchmarku Maxima Novaka \citep{site:serializationspeed}. Tímto jsme vybrali formát Protocol buffers, který lze v~jazyce C\# použít buď manuální specifikací \textit{\uv{message} } pomocí IDL, vygenerováním zdrojového kódu tříd v~jazyce C\# z~této specifikace a~následnou manuální serializací, nebo využitím knihovny Protobuf-net, která z~anotací ve zdrojovém kódu generuje specifikaci \textit{\uv{message} } a  metody pro serializaci a~deserializaci dat.

Naším cílem bylo využití knihovny Protobuf-net, bohužel tato knihovna vyžaduje v~případě použití dědičnosti, aby \textit{base} typy, tedy typy, od kterých se dědí, byly označeny atributem \texttt{ProtoInclude} \citep{site:protobufnet}. Vzhledem k~tomu, že všechny naše třídy, reprezentující jednotky, budovy, projektily a~hráče, dědí od UrhoSharp třídy \texttt{Component}, není možné tyto třídy serializovat pomocí Protobuf-net. Tento problém lze řešit dvěma způsoby: 

\begin{enumerate}
	\item Separovat reprezentaci entit do dvou tříd. První třída, která bude potomkem třídy \texttt{Component} a~bude reprezentovat entitu v~grafu herní scény, a~druhá, která bude reprezentovat entitu z~pohledu logiky, s~kterou by pracovala naše platforma i~implementace pluginů, a~která by vlastnila a~ovládala první třídu. V~této reprezentaci by bylo možné využít Protobuf-net, znamenalo by to ovšem složitou delegaci volání mezi třídami a~možné problémy při využití grafu scény pro nalezení entity. 
	\item Využít pro serializaci přímo technologii protocol buffers, která je na pozadí používána knihovnou Protobuf-net, a~manuálně implementovat serializaci hry. Tato možnost se nám zdála jako lepší, protože nám také umožňovala manuálně specifikovat \texttt{.proto} soubory, z~kterých protocol buffers generují třídy umožňující serializaci a~deserializaci v~různých jazycích. Navíc nám tento postup umožňuje specifikovat přesnou posloupnost akcí při deserializaci.
\end{enumerate}

\section{Poskytovaná funkcionalita}
V~úvodní kapitole jsme vymezili druh her, které bude naše platforma podporovat, pomocí společných rysů a~funkcionality. Cílem naší platformy je poskytovat implementaci této společné funkcionality a~tím zjednodušit tvorbu her. Tato kapitola rozebere naši implementaci, její výhody a~omezení, a~popíše možnosti modifikace chování těchto implementací z~uživatelského kódu.

\subsection{Mapa}
\label{sec:mapimpl}
Jednou z~hlavních funkcionalit poskytovaných naší platformou tvůrcům her je implementace herní mapy. Tato implementace poskytuje rozhraní pro dotazování na aktuální stav, manipulaci s~mapou a~grafickou reprezentaci zobrazenou hráči. Naše implementace odděluje logickou reprezentaci, která je poskytována zbylému kódu platformy a~kódu uživatelských pluginů, a~grafickou reprezentaci, která je vytvářena z~logické reprezentace a~zobrazována hráči. Toto rozdělení je určeno pro možné změny grafické reprezentace bez ovlivnění logické reprezentace a~na ní závislého kódu. 

\subsubsection{Logické rozdělení}
\label{sec:maplogic}
Mapa je ve hrách využívána pro zjišťování typu a~výšky terénu či přítomnosti entit, jako například budov a~jednotek, nejčastěji za účelem vyhledání cesty mezi dvěma pozicemi v~herním světě. Dalším možným důvodem pro zjišťování mapou poskytovaných informací může být zjištění viditelnosti mezi dvěma entitami, možnosti střelby mezi dvěma entitami či interakce hráče se skupinami entit.  Existuje několik způsobů, jak implementovat tyto funkce herní mapy.

V~úvodní kapitole jsme specifikovaly, jaké hry bude naše platforma podporovat. Z~vlastností těchto her nám vyplynula obdélníková mapa, rozdělená na čtvercové dlaždice. Toto ovšem není jediná možná implementace. V~této části popíšeme další možné typy implementací map, jejich výhody, nevýhody a~použití.

\textbf{Mapa bez rozdělení} vede při každém dotazu na přítomnost budov či jednotek v~dané oblasti mapy k~výpočtu průsečíků ploch zabíraných všemi jednotkami a~budovami s~danou oblastí mapy. Asymptotická složitost této operace je tedy O(N), kde N je počet budov či jednotek v~naší hře.

\textbf{Mapa s~rozdělením} na části umožňuje při dotazu na přítomnost jednotek či budov v~oblasti mapy prozkoumat pouze části tvořící tuto oblast mapy. Části, označovány jako \textit{\uv{Tiles} }, v~překladu dlaždice, mohou být různých tvarů. Nejčastějšími tvary jsou čtverec či šestiúhelník. 

Při rozdělení na čtverce lze mapu jednoduše uložit do dvourozměrného pole, což dále usnadňuje dotazy na prohledání oblasti mapy. Nevýhodou čtvercového tvaru dlaždic je různá vzdálenost sousedních dlaždic, pokud umožníme pohyb mezi všemi osmi sousedními dlaždicemi. Kde vzdálenost středů dlaždic sousedících celou hranou je rovna délce hrany, vzdálenost středů dlaždic sousedících pouze rohem je rovna \(\sqrt{2a^2}\), kde \textit{a} je délka hrany čtverce. Oproti tomu při použití šestiúhelníkových dlaždic je vzdálenost mezi všemi sousedy rovna. 

\subsubsection{Grafická reprezentace} 
\label{sec:mapgraphicsanalaysis}
Jak jsme popsali v~předešlé části, mapa je obdélníkového tvaru, rozdělena na čtvercové dlaždice o~velikosti 1x1. Existují dva základní druhy možných implementací grafického zobrazení takovéto mapy.

První možností, ilustrovanou obrázkem \ref{fig:mapbigtexture}, je vytvoření textury, ve které bude každé dlaždici odpovídat separátní část textury. Tuto implementaci nazýváme \uv{Textura dlaždic}. Velikost textury je závislá jak na velikosti mapy, v~obrázku značené \textit{m}, \textit{n} a~udávané v~počtu dlaždic v~daném rozměru, tak na velikosti textury jednotlivých typů dlaždic, zde značené \textit{k}. S~ohledem na čtvercový tvar dlaždic požadujeme i~po texturách jednotlivých typů čtvercový tvar. Velikost výsledné textury bude rovna \(k^2 * m * n\). Tuto texturu můžeme vidět v~dolní části obrázku \ref{fig:mapbigtexture}. Na objekt reprezentující terén je následně tato textura promítnuta pomocí standardního \textit{UV} mapování, což je na ukázce reprezentováno šipkami mapujícími vrcholy geometrie na texturu. V~takovéto textuře má každá dlaždice přiřazenu disjunktní část textury, lze tedy vzhled jednotlivých dlaždic upravovat nezávisle na vzhledu ostatních dlaždic. Tímto způsobem lze vytvořit různé přechody mezi typy dlaždic a~provést další úpravy na celkovém vzhledu herní mapy. Ukázku takovýchto přechodů můžeme vidět na dlaždicích 1 a~2.
 
Nevýhodou této implementace je především velikost textury, která v~závislosti na velikostech textur jednotlivých typů dlaždic může dosáhnout stovek MiB až jednotek GiB. Takováto velikost textury by byla neúnosná i~pro slabší počítače, nemluvě o~mobilních zařízeních. Další nevýhodou této implementace je složitost změny typu dlaždice. Při změně typu je v~této reprezentaci nutné najít část textury odpovídající měněné dlaždici a~do této části nakopírovat texturu nového typu. Vzhledem k~velikosti textur dlaždic by tato operace mohla být časově velice náročná, především při změně typu velkého počtu dlaždic. Možnou výhodou této implementace je schopnost reprezentovat každý z~vrcholů, ve kterém se vyskytují rohy sousedních dlaždic, pomocí jednoho \uv{vertexu}, tedy grafického vrcholu. Díky tomu je omezena velikost \textit{vertex bufferů} a~zaručena spojitost mapy. Vzhledem k~velikosti textury je ovšem úspora paměti nedostatečná.

\begin{figure}[h]
	\centering
	\input{img/BigTextureMap.pdf_tex}
	\caption{Implementace pomocí textury dlaždic.}
	\label{fig:mapbigtexture}
\end{figure}

Druhou možností, ilustrovanou obrázkem \ref{fig:mapsmalltexture}, je vytvořit texturu, ve které bude textura každého typu dlaždic právě jednou. Tuto texturu, obsahující vzhled typů dlaždic, můžeme vidět v~levé dolní části obrázku. Velikost takovéto textury závisí pouze na počtu typů dlaždic, oproti předchozí implementaci, ve které závisela velikost textury na velikosti mapy. i~když typů dlaždic může být neomezeně, nepředpokládáme, že by tento počet přesáhl malé stovky. Díky tomu je tato implementace paměťově mnohem úspornější. Při takovéto implementaci je poté každá dlaždice určitého typu namapována na stejnou část textury, což je na obrázku ukázáno na příkladu dlaždic 1 a~2. Vzhledem k~tomuto mapování všech dlaždic stejného typu na stejnou část textury nelze jednoduše vytvořit přechody mezi typy dlaždic, jak tomu bylo v~první implementaci. Jedinou možností je vytvořit separátní typ, který bude reprezentovat přechod mezi dvěma typy dlaždic. Další možnou nevýhodou této implementace je nutnost reprezentovat každou dlaždici separátní čtveřicí tzv.~\uv{vertexů}, tedy grafických vrcholů. Oproti první implementaci tedy bude pro reprezentaci mapy stejné velikosti potřeba čtyřnásobný počet vrcholů. Tato možná nevýhoda je ale vyvážena úsporou paměti v~rámci textury typů, dále díky jednoduší implementaci změny typů dlaždic, a~v~neposlední řadě jednodušším rozdělením mapy na části, popsané dále. Změna typu dlaždice v~této reprezentaci spočívá ve změně \texttt{UV} souřadnic čtyř vrcholů, odpovídajících měněné dlaždici. Jedná se tedy o~zapsání osmi čísel, které oproti kopírování částí textury v~první implementaci mění mnohem méně dat. V~obrázku si můžeme toto přemapování představit jako přesunutí všech čtyř šipek dané dlaždice na jinou část textury. Vzhledem k~těmto výrazným úsporám času a~paměti jsme zvolili právě tuto možnost pro naši implementaci.

\begin{figure}[h]
	\centering
	\input{img/SmallTextureMap.pdf_tex}
	\caption{Implementace pomocí textury typů.}
	\label{fig:mapsmalltexture}
\end{figure}


Další součástí naší implementace je rozdělení mapy na části, kterým my, podle hry Minecraft, která nás k~tomuto rozdělení inspirovala, říkáme \uv{Chunk} . Hlavním důvodem pro toto rozdělení je velikost vertex bufferů, která je v~enginu Urho3D, a~tedy i~enginu UrhoSharp, limitovaná na 64 tisíc vertexů. To při naší implementaci odpovídá mapě o~velikost 127 krát 127 dlaždic, což je z~našeho pohledu příliš malá velikost pro typ her, který chceme podporovat. Zřejmou výhodou tohoto rozdělen je rozšíření množiny možných velikostí map, která je při této implementaci z~vrchu omezena pouze výkoností hardwaru a~velikostí paměti. Na druhou stranu toto rozdělení klade na možné velikosti map jiné omezení, a~to nutnost, aby každá velikost map byla celočíselným násobkem velikosti chunků v~obou rozměrech. V~aktuální verzi platformy je velikost chunku nastavena na 50 krát 50 dlaždic, což určuje krok mezi velikostmi map. Každá mapa tedy musí být velikosti 50*x na 50*y. Tuto velikost jsme zvolili podle velikostí mapy ve hře Stronghold Crusader, kterou jsme v~úvodní kapitole\ref{sec:uvod} označili jako zástupce typu her, které chceme v~naší platformě podporovat.

Další výhodou tohoto rozdělení je možnost tzv.~\uv{cullingu} , kdy engine nevykresluje modely vzdálené od kamery více než nastavitelný limit. Účelem této vlastnosti enginu je zmenšení hardwarových nároku, který plní i~v~tomto případě. Dále můžeme při úpravě terénu využít tohoto rozdělení pro zamknutí pouze upravovaných chunků, čímž zmenšujeme velikost dat, které je potřeba přenést do paměti grafické karty, a~dále zmenšujeme výpočetní náročnost reprezentace mapy.

\subsubsection{Deformace terénu}
Jak jsme v~předchozích částech popsali, je naše mapa logicky rozdělena na čtvercové dlaždice, kde každá z~dlaždic je graficky reprezentována čtyřmi vrcholy v~modelu mapy. Tyto dvě reprezentace, logickou a~grafickou, musí naše implementace držet synchronizované, aby hráči byl zobrazován skutečný stav hry, jak ho vidí jednotky a~nepřátelé. 

Logickou myšlenkou je zvolit si jednu z~reprezentací, kterou budeme upravovat manuálně, a~následně druhou reprezentaci generovat z~reprezentace první. Pro manuální upravování jsme určili logickou reprezentaci, která je tvořena C\# objekty a~lze ji tedy jednoduše upravit bez použití složitějších nástrojů jazyka C\#. Tímto máme zajištěno, že logiky hráčů, jednotek a~budov budou pracovat s~aktuálním stav mapy. Nyní musíme zajistit, aby i~hráči byl zobrazen tento stav.

Oproti úpravě logické reprezentace je úprava grafické reprezentace složitější. Z~předchozí části víme, že chceme využít manuální práce s~vertexy, především pro nízkou výpočetní složitost takovýchto úprav. Pro práci s~vertexy poskytuje engine UrhoSharp dva nástroje. 

Prvním nástrojem je třída \texttt{CustomGeometry}, která překrývá VertexBuffery a~IndexBuffery a~poskytuje API pro zjednodušení práce s~těmito strukturami. Bohužel zde narážíme na aktuální implementaci UrhoSharp, která neumožňuje přístup k~metodám, která v~Urho3D pracují s~třídou \texttt{PODVector}. Jak říká dokumentace \citep{site:urho3DPOD}, tato třída slouží jako vektor (ve smyslu jazyka C++) pro uložení tzv.~\textit{POD} (\textit{plain old data}) typů, což jsou typy \textit{\uv{nevolají konstruktor nebo destruktor a~používají block move}} . Bohužel tuto třídu nebyli zatím autoři UrhoSharp schopni poskytnout v~jazyce C\#, a~tím pádem ani jakoukoli metodu, která v~enginu Urho3D tuto třídu používá. V~případě CustomGeomtery je to metoda \texttt{GetVertices()}, která umožňuje přístup ke všem vertexům CustomGeometry najednou, bez režie metod poskytujících přístup k~vertexům jednotlivě, jako například \texttt{GetVertex}. Tato skutečnost nám znemožňuje použití \texttt{CustomGeometry} s~dostatečnou výkoností, a~proto jsme byli nuceni použít druhý nástroj.

Druhým nástrojem jsou třídy \texttt{VertexBuffer} a \texttt{IndexBuffer}. Tyto třídy poskytují přímý přístup k~datům geometrií, umožňující měnit jednotlivé vrcholy a~indexy. Za použití těchto tříd dokážeme generovat geometrii či ji upravovat přímo za běhu, a~to s~minimální možnou režií. Tato minimální režie je umožněna poskytnutím přímého přístupu k~datům v~paměti a~jejich následnou manipulací pomocí \texttt{unsafe} C\# kódu. K~této minimální režii nám oproti použití CustomGeometry také tento způsob umožňuje do paměti zamykat pouze části geometrií, kterých se změny týkají, oproti CustomGeomtery, která, pokud by UrhoSharp tuto funkcionalitu implementoval, by nás nutila zamknout do paměti veškerá data geometrie.


\subsection{Hledání cesty}
\label{sec:pathfinding}
Hledání cesty, neboli \textit{PathFinding}, je jednou z~nejuniversálněji používaných funkcionalit ve všech typech her, nejen RTS. Každá hra obsahující jednotky pohybující se po mapě používá nějakou formu pathFindingu. Nejčastěji ve hrách hledáme nejkratší cestu mezi dvěma body, odpovídajícími aktuální pozici jednotky a~cílové pozici jednotky. Standardní algoritmy používané pro hledání cesty pracují nad váženým grafem. Tato reprezentace zmenšuje prohledávaný prostor a~umožňuje rychlejší nalezení cesty. Graf je složen z~vrcholů, reprezentujících pozice v~herním světě, a~hran, reprezentujících možnost průchodu mezi spojenými vrcholy. Tyto hrany mohou být orientované, umožňující průchod pouze jedním směrem.

V~této části popíšeme různé možnosti převodu logické reprezentace mapy na graf pro vyhledávání cesty, implementaci těchto převodů v~algoritmu pro hledání cesty poskytovaném naší platformou a~možnost použití tvůrcem implementovaného algoritmu pro hledání cesty naší platformou.


\subsubsection{Převod mapy}
Převod mapy na graf lze dělat dvěma způsoby, dynamicky, kdy je graf generován při výpočtu cesty, nebo staticky, kdy je graf generován na počátku hry, případně je generován znovu při změně mapy.

Dynamické generování grafu umožňuje reprezentovat v průběhu času měnící se stav mapy. Tímto způsobem lze do hledání cesty zahrnout aktuální pozice jednotek, postavené budovy či změny terénu. Problémem tohoto typu generování grafu je výpočetní náročnost.

Statické generování grafu umožňuje jedno vygenerování grafu, které předzpracuje logickou reprezentaci mapy a~maximálně optimalizuje graf pro zmenšení náročnosti hledání cesty. Nevýhodou je neschopnost odrážet změny v~terénu, pozice entit či stavbu budov. Dalším problémem je rozdílná průchodnost částí mapy pro různé typy jednotek. Pro řešení tohoto problému pouze statickým generováním musí být pro každý typ jednotky vygenerován zvláštní graf, což se negativně projeví na paměťové složitosti programu.

\subsubsection{V~naší platformě}
\label{sec:astar}
Naše platforma definuje rozhraní \texttt{IPathFindAlg}, které reprezentuje algoritmus pro hledání cesty používaný součástmi platformy a~poskytovaný pro použití tvůrci hry. Zároveň platforma poskytuje tvůrcům hry jednu možnou implementaci tohoto rozhraní, a~to pomocí algoritmu A*.

Algoritmus A* je jedním z~nejpoužívanějších algoritmů pro hledání cest ve hrách, jak říká Amit Patel ve svém článku \citep{site:introastar}. Z~tohoto důvodu jsme se tento algoritmus rozhodli použít jako základní implementaci hledání cesty v~naší platformě. A* je grafový algoritmus pracující nad váženým orientovaným grafem. Algoritmus dostává jako zadání graf, počáteční vrchol a~koncový vrchol. Oproti Dijkstrovu algoritmu, který prochází vrcholy od startovního vrcholu v~pořadí vzrůstající ceny cesty do vrcholu, modifikuje A* toto uspořádání vrcholů pomocí heuristiky, kterou přičítá k~ceně cesty do vrcholu. Toto porovnání můžeme vidět na obrázku \ref{fig:astardijkstra}, které ukazují pořadí prohledaných vrcholů v~grafu. Můžeme vidět, že Dijkstrův algoritmus vytváří okolo startovního vrcholu rovnoměrný kruh prohledaných vrcholů (pokud vzdálenost vrcholu bereme jako cenu cesty ze startovního vrcholu). Oproti tomu algoritmus A* prohledává vrcholy nejen podle jejich vzdálenosti od počátečního vrcholu, ale za použití heuristiky zohledňuje i~jejich vzdálenost od cílového vrcholu. Díky tomu algoritmus A* ve většině případů zmenšuje počet prohledaných vrcholů.

\begin{figure}[h]
	\centering
	\subfloat[Dijkstrův algoritmus.]{\includegraphics[width=0.47\textwidth]{Dijkstra}\label{fig:dijkstra}}
	\hfill
	\subfloat[A* algoritmus.]{\includegraphics[width=0.47\textwidth]{AStar}\label{fig:astar}}
	\caption{Porovnání Dijkstrova algoritmu s~algoritmem A*. Ilustrace převzaty z~článku Amita Patela \citep{site:introastar}.}
	\label{fig:astardijkstra}
\end{figure}

Problémem algoritmu A* může být vlastní heuristika. Algoritmus A* požaduje, aby heuristika splňovala dvě vlastnosti, a~to přípustnost a~monotonnost. Přípustnost označuje, že hodnota heuristiky pro každý vrchol je menší než minimální délka cesty do tohoto vrcholu. Monotonnost poté označuje tento vztah \(h(x) <= d(x,y) + h(y)\), kde \(h(x)\) je hodnota heuristiky vrcholu x a \(d(x,y)\) délka hrany z~vrcholu x do vrcholu y. Splnění těchto vlastností zajišťuje, že nalezená cesta bude optimální (přípustnost) a~každý vrchol prohledáme nejvýše jednou (monotonnost). 

Naše implementace algoritmu A* používá spojení statického a~dynamického generování grafu, spolu s~tvůrcem dodanou heuristikou. Graf, reprezentující terén úrovně, je vygenerován staticky. Tedy pro každou dlaždici je vytvořen vrchol a~každý vrchol je spojen se všemi vrcholy reprezentujícími sousední dlaždice. Dále je umožněno do grafu za běhu přidávat vrcholy a~spojovat je s~již existujícími. To je použito pro umožnění pohybu po budovách, u~kterých jsou při jejich stavbě do grafu dodány vrcholy reprezentující schůdné plochy budovy a~tyto vrcholy jsou spojeny jak mezi sebou, tak s~existujícími vrcholy grafu. 

Dynamická část generování grafu se projevuje při výpočtu vah hran. Hrany nemají pevně stanovenou váhu, místo toho je pro výpočet cesty požadováno po tvůrci hry, aby každá jednotka hledající cestu poskytla tzv. kalkulátor, který dynamicky ohodnocuje hrany a~definuje jejich průchodnost. Tento přístup nám umožňuje zachovávat pouze jeden graf, který se díky využití kalkulátorů chová jako rozdílný graf pro každý kalkulátor. 
Vzhledem k~požadavkům algoritmu A* na vztah heuristiky a~ceny hran, popsaných výše, požadujeme po tvůrci hran poskytnutí heuristiky v~rámci kalkulátoru. Zároveň tento přístup umožňuje tvůrci místo vlastní implementace algoritmu pro hledání cesty pouze změnit poskytované hodnoty hran a/nebo heuristiky, čímž může tvůrce změnit chování naší implementace algoritmu A* pro svoji potřebu.


Naše implementace A* ovšem není jediným použitelným algoritmem pro hledání cesty v~naší platformě. Pokud tvůrci nestačí úprava váhy hran a~heuristiky, umožňuje platforma implementaci vlastního algoritmu splňujícího rozhraní \texttt{IPathFindAlg}, který bude následně používán všemi součástmi platformy. Používaný algoritmus je specifikován při startu úrovně, je tedy možné, aby i~v~jednom balíčku různé úrovně používaly různé algoritmy pro hledání cesty.

\subsection{Projektily}
Simulace projektilů je častým problémem v~RTS hrách, tedy v~typu her, který naše platforma chce podporovat. Rozhodli jsme se proto implementovat tuto funkcionalitu, umožňující výpočet počátečního směru projektilu při střelbě na pohyblivý cíl, simulaci letu se stálou gravitací a~detekci zásahů. 

\subsubsection{Typy projektilů}
Existuje několik typů simulací projektilů, rozdělených podle stupně přesnosti simulace reálného světa. Těmito typy, seřazenými podle vzrůstající realističnosti simulace, jsou:
\begin{itemize}
	\item \textit{Hitscan} (bez simulace letu),
	\item s~konstantní horizontální rychlostí (proměnlivá gravitace, bez odporu vzduchu),
	\item s~konstantní dopřednou rychlostí (konstantní gravitace, bez odporu vzduchu),
	\item s~plně realistickým chováním (konstantní gravitace, simulace odporu vzduchu).
\end{itemize}

\textbf{Hitscan projektily} jsou zvláštním typem projektilů, který má nulovou dobu letu. V~okamžiku výstřelu je proveden výpočet, zda se ve směru, kterým byl výstřel mířen, vyskytuje cíl. Pokud ano, je tomuto cíli okamžitě uděleno poškození. Hitscan projektily lze dále rozdělit podle chování po zasáhnutí prvního cíle. Projektil může být zastaven prvním cílem, nebo pokračovat dále skrz první cíl a~případně zasáhnout další cíle.

\textbf{Projektily s~konstantní horizontální rychlostí} umožňují tvůrci hry z~pohledu logiky uvažovat pouze pohyb v~rovině terénu. Jak můžeme vidět na obrázku \ref{fig:horizontalproj}, výška oblouku nemá u~tohoto druhu projektilu vliv na dobu letu a~je závislá pouze na vzdálenosti zdroje a~cíle v~rovině. Tato vlastnost umožňuje zvolit výšku oblouku tak, aby graficky vypadala dobře. Z~pohledu vyvážení síly jednotek a~budov umožňuje tento typ projektilů jednoduše spočítat dobu letu jako \( t=\text{vzdálenost}(\text{střelec}, \text{cíl}) / v \), kde \textit{v} značí určenou rychlost projektilu. Tento typ projektilů je především vhodný pro hry odehrávají se v~jedné rovině, jako například Warcraft nebo Starcraft.

\begin{figure}[h]
	\centering
	\def\svgwidth{0.7\textwidth}
	\input{img/horizontal.pdf_tex}
	\caption{Trajektorie a~čas letu projektilu s~konstantní horizontální rychlostí.}
	\label{fig:horizontalproj}
\end{figure}

\textbf{Projektily s~konstantní dopřednou rychlostí} se pohybuje po balistické křivce, určené počáteční rychlostí, směrem výstřelu a~sílou gravitace. Jak můžeme vidět na obrázku \ref{fig:forwardproj}, existují vždy nejvýše dvě trajektorie, kterými lze zasáhnout při daných hodnotách výše vyjmenovaných vlastností. Díky konstantní dopředné rychlosti a~různým délkám trajektorií vždy platí vztah \(x_1 <= x_2\) Oproti předešlým typům projektilů vyžaduje tento typ složitější výpočty pro zasažení cíle, především pohybujícího se cíle. Tyto výpočty byli v~naší platformě převzaty z~článku Forresta Smitha \citep{site:projectilecalc}. Tento článek poskytuje vzorec, podle kterého lze vypočítat směr střelby projektilu, pokud známe pozice střelce, cíle, gravitaci a~počáteční rychlost projektilu. Pohyb cíle musí být konstantní rychlostí konstantním směrem. Vzhledem k~implementaci pohybu v~naší platformě, který se odehrává po spojnicích středů dlaždic, je potřeba najít spojnici, na které se bude cíl právě vyskytovat. Tento postup je iterativní, kdy procházíme cestu cíle a~v~každém bodě změny směru počítáme dobu letu projektilu. Ve chvíli, kdy najdeme dva body, kde v~prvním bude jednotka dříve než projektil, a~v~druhém později než projektil, našli jsme požadovanou spojnici. Následně využijeme převzatý vzoreček pro spočtení směru výstřelu. Tento postup by bylo možné celý provést iterativně, kde místo použití vzorečku by jsme znovu prohledávali spojnici dvou bodů a~přibližovali se k~cíli.

\begin{figure}[h]
	\centering
	\def\svgwidth{0.7\textwidth}
	\input{img/forward.pdf_tex}
	\caption{Trajektorie a~čas letu projektilu s~konstantní dopřednou rychlostí.}
	\label{fig:forwardproj}
\end{figure}

\textbf{Projektily s~realistickým chováním} přidávají oproti předchozímu typu ještě jednu vlastnost, a~to odpor vzduchu. Jak můžeme při porovnání trajektorií na obrázcích \ref{fig:forwardproj} a \ref{fig:withdragproj}, má trajektorie tohoto typu projektilu složitější tvar.  Výpočet tohoto typu projektilů je nejčastěji prováděn iterativně, kdy je odhadnuta doba letu, podle této doby letu je predikován pohyb cíle, a~následně spočítána opravdová délka letu. Tento postup je opakován do dosažení potřebné přesnosti.

\begin{figure}[h]
	\centering
	\def\svgwidth{0.7\textwidth}
	\input{img/withdrag.pdf_tex}
	\caption{Trajektorie a~čas letu projektilu s~plně realistickým chováním.}
	\label{fig:withdragproj}
\end{figure}


Podle priorit a~zasazení hry jsou často voleny méně realistické modely, které z~pohledu hry mohou mít výhody oproti plně realistickému chování. Těmito výhodami může být lepší zasazení do tématu hry, což můžeme vidět u~laserových zbraní v~sci-fi hrách jako Starcraft~2 \citep{site:starcraft}, které jsou typu \textit{Hitscan}. I~když by se zdálo, že je vždy nejlepší co nejvíce se blížit realitě a~tedy vybrat si plně realistické chování, v~některých hrách je důležitější možnost vyvážení síly jednotek, předvídatelnosti chování a/nebo vzhledu.  


Naše platforma poskytuje implementaci výpočtu směru střelby pro projektily s~konstantní dopřednou rychlostí a~simulaci letu tohoto typu projektilů. Naším cílem ovšem bylo nijak neomezovat tvůrce her, kvůli čemuž jsou tyto prvky poskytovány jako komponenty, které lze připojit k~projektilu a~které následně řídí jeho chování. Nijak ale nejsou provázány s~vlastní třídou projektilu, čímž je tvůrcům hry umožněno implementovat kterýkoli z~typů projektilů.



\chapter{Programátorská dokumentace}
V~této kapitole popíšeme naší implementaci platformy MHUrho a~ukázkové hry. K~implementaci platformy bylo použito Visual Studio 2017 Education a .NET Framework 4.7.2, který je zároveň cílovým frameworkem naší platformy. Celá implementace je obsažena v~jediném \textit{solution}, MHUrho, dostupném v přílohách práce \ref{sec:appendix}. Výsledný program i jeho zdrojový kód jsou licencovány pod MIT licencí.

\section{Struktura solution}

\begin{figure}[h]
	\centering
	\fontsize{10pt}{11pt}\selectfont
	\def\svgwidth{0.9\textwidth}
	\input{img/Project_structure.pdf_tex}
	\caption{Struktura solution. Zelené značí součásti naší práce, oranžové místo budoucí rozšiřitelnosti.}
	\label{fig:solution_structure}
\end{figure}

Hlavním cílem naší práce byla tvorba platformy pro tvorbu RTS her. Jak můžeme vidět v~diagramu \ref{fig:solution_structure}, vlastní implementace platformy je realizována několika projekty. Účelem tohoto rozdělení je separace přenositelně implementovatelných částí platformy, které díky využití enginu UrhoSharp a~celkovému návrhu práce tvoří velkou většinu funkcionality, od částí závislých na cílovém systému. 

Přenositelné části jsou obsaženy v~projektu \texttt{MHUrho}. Výstupem tohoto projektu je knihovna využívaná jak implementacemi pro specifické systémy, tak balíčky tvořícími jednotlivé hry. 

Implementace pro specifické systémy obsahují především inicializační kód, který dostává spuštěný program do konzistentního stavu bez ohledu na systém, a~dále implementace rozhraní specifikovaných v~přenositelné části, které umožňují této části pracovat s~oblastmi, pro které ani framework .NET ani používaný herní engine neposkytují přenositelnou implementaci. Tyto oblasti byli popsány v~části \ref{sec:system_dif}.

Dále diagram ukazuje dvě součásti solution, které nejsou přímou součástí funkcionality platformy. První z~těchto součástí je projekt \texttt{ShowcasePackage}, který obsahuje implementaci ukázkové hry sloužící pro demonstraci schopností platformy a~jako referenční implementace pro budoucích tvůrce balíčku. Druhou součástí je \texttt{Installer}, jehož výstupem je instalátor umožňující jednoduchou instalaci platformy na systém Windows spolu s~instalací všech závislostí a~vložení \textit{Ukázkového balíčku} do nainstalované instance platformy.

\section{Spuštění aplikace}
\label{sec:init}
Spuštění aplikace je specifické pro každý ze systémů. Vzhledem k~tomu, že cílem naší práce byla implementace platformy pro systém Windows, popíšeme v~této části spuštění aplikace v~rámci tohoto systému.

\subsection{Systémová část}

\begin{figure}[h]
	\centering
	\fontsize{8pt}{11pt}\selectfont
	\def\svgwidth{\textwidth}
	\input{img/MHUrhoApp2.pdf_tex}
	\caption{Spouštění aplikace.}
	\label{fig:startup}
\end{figure}

Spuštění začíná v~implementaci specifické pro daný systém. Hlavní třídy účastnící se spuštění aplikace a~posloupnost volání metod můžeme vidět na obrázku \ref{fig:startup}. U~každé ze tříd jsou ilustrovány pouze prvky účastnící se inicializace. 

Pro systém Windows začíná celý proces zavoláním metody \texttt{Program.Main}. Tato metoda jako první inicializuje \texttt{FileManager}, implementující práci se soubory na systému Windows, a~uloží ho do statické položky třídy \texttt{MHUrhoApp}. Tento systém je nezbytné inicializovat před spuštěním herního enginu, protože je používán při inicializaci třídy \texttt{MHUrhoApp}.

Jak ukazuje diagram, třída \texttt{MHUrhoApp} je potomkem třídy \texttt{Application} z~enginu UrhoSharp. Tato třída, již podle názvu, reprezentuje celou aplikaci v~enginu UrhoSharp a~poskytuje přístup ke všem součástem enginu. 

Dalším krokem v~metodě \texttt{Main} je zpracování parametrů aplikace z~příkazového řádku do platformou definované systémově nezávislé reprezentace. Instance třídy \texttt{StartupOptions} obsahující tuto reprezentaci je následně také uložena do statické položky třídy \texttt{MHUrhoApp}, protože je stejně jako \texttt{FileManager} používána při inicializaci MHUrhoApp.

Položky \texttt{FileManager} a \texttt{StartupArgs} jsou statické pro budoucí rozšiřitelnost na platformu Android.
Důvodem je nutnost spuštění enginu pomocí následujícího kódu:

\begin{lstlisting}[
	style=csharp,
	emph={[1]MHUrhoApp, ApplicationOptions}
]
await surface.Show<MHUrhoApp>(new ApplicationOptions("Data"));
\end{lstlisting}

Metoda \texttt{Show} bohužel nedovoluje přidat konstruktoru \texttt{MHUrhoApp} další parametry, kterými by bylo možné předat \texttt{FileManager} a \texttt{StartupOptions}.

Na systému Windows následuje vytvoření instance \texttt{MHUrhoApp} a~zavolání metody \texttt{Run}. Tímto voláním je předána kontrola nad procesem enginu UrhoSharp, který je tímto inicializován a~je v~něm spuštěna herní smyčka.

V~rámci své inicializace zavolá třída \texttt{Application}, reprezentující herní engine, svou virtuální metodu \texttt{Start}, jejíž přetížením nám umožňuje inicializovat naší část aplikace. Tímto se dostáváme do přenositelné části inicializace.

\subsection{Přenositelná část}

Jak bylo řečeno v~předchozí části, přenositelná část inicializace začíná v~okamžiku volání metody \texttt{Start} herním enginem.

V~rámci této metody provede naše platforma inicializaci všech svých součástí a~přejde do režimu reagování na interakce uživatele s~grafickým rozhraním.

Důležitým předpokladem naší platformy i~enginu UrhoSharp je přítomnost pouze jedné instance třídy \texttt{MHUrhoApp}, respektive jejího předka \texttt{Application} v~dané \texttt{AppDomain}. Vzhledem k~tomuto předpokladu je tato jediná instance třídy zpřístupněna pomocí statické proměnné \texttt{Instance}.

Dalšími kroky inicializace jsou:

\begin{enumerate}
	\item Vytvoření \texttt{SynchronizationContextu}, který umožňuje implementaci načítání popsaného v~částech \ref{sec:packageloading} a \ref{sec:loading}.
	\item Načtení a~nastavení konfigurace aplikace za využití třídy \texttt{FileManager} inicializované v~systémové části. Toto nastavení ovlivňuje například vzhled okna aplikace či maximální vzdálenost vykreslování ve hře.
	\item Načtení seznamu přítomných balíčků, popsané blíže v~části \ref{fig:packagemanager}. Při chybě v~načítání kteréhokoli z~balíčků je po spuštění uživatelského rozhraní uživateli zobrazeno hlášení o~této chybě.
	\item Inicializace vstupu a~výstupu. Tato inicializace je implementována podle návrhového vzoru \textit{Abstract factory} \citep[str.~87]{book:gangoffour} pro zjednodušení přepínání mezi různými schématy vstupu a~výstupu. V~rámci naší práce je implementováno pouze schéma myši a~klávesnice, ale v~implementaci platformy je připravena základní struktura pro implementaci schématu pro dotykové obrazovky. 
\end{enumerate}

Tímto je inicializace kompletní. Volání metody \texttt{Start} se vrací do metody \texttt{Run}, ve které engine předchází do smyčky obsluhující události uživatelského vstupu.

\section{Uživatelské rozhraní}
\label{sec:ui}

\begin{figure}[h]
	\centering
	\fontsize{8pt}{11pt}\selectfont
	\def\svgwidth{\textwidth}
	\input{img/UIReferences.pdf_tex}
	\caption{Datová struktura uživatelského rozhraní.}
	\label{fig:uireferences}
\end{figure}

Jak můžeme vidět na diagramu \ref{fig:uireferences}, centrálními třídami uživatelského rozhraní v~rámci menu jsou třídy \texttt{MenuController} a \texttt{MenuUIManager}. Dále můžeme vidět, že z~pohledu herního enginu je grafické rozhraní reprezentováno třídou \texttt{UI} s~kořenovým prvkem \texttt{Root}.

Účelem třídy \texttt{MenuController} je odstínění implementace uživatelského rozhraní od detailů implementace zbytku platformy. Pokud je tedy výsledkem uživatelovi akce změna stavu jiné části platformy než UI, je tato akce delegována právě třídě \texttt{MenuController}. Zároveň se také tato třída snaží odstiňovat zbytek platformy od implementace uživatelského rozhraní, tedy pokud část hry požaduje změnu uživatelského rozhraní, měla by tuto akci delegovat třídě \texttt{MenuController}.

Třída \texttt{MenuUIManager} je, jak můžeme vidět, centrálním bodem implementace uživatelského rozhraní menu. Její hlavní funkcí je management obrazovek menu, implementace přepínání mezi nimi a~poskytnutí přístupu k~třídám reprezentujícím tyto obrazovky. V~rámci přepínání obrazovek umožňuje tato třída jednoduše přepínat zpět na předchozí obrazovky, protože při každém přepnutí na novou obrazovku je stará obrazovka přidána do zásobníku předchozích obrazovek.

Každá z~obrazovek menu je reprezentována samostatnou třídou implementující chování obrazovky a~zajišťující správu prostředků herního enginu příslušících dané obrazovce. Jak můžeme vidět na diagramu \ref{fig:uireferences}, slouží každá ze tříd reprezentujících obrazovky menu jako \textit{proxy} pro třídu, která implementuje chování obrazovky a~je vlastníkem prostředků enginu pro její zobrazení uživateli. 

Z~pohledu enginu tvoří uživatelské rozhraní strom s~kořenem \texttt{Root}. Každá obrazovka je tedy dítětem tohoto kořenu, s~dalšími prvky jako tlačítky a~popisky jako svými dětmi. Tyto prostředky, reprezentující danou obrazovku, jsou spravovány právě implementační třídou a~mají stejnou životnost jako tato třída.

Zobrazení obrazovky je uskutečněno v~reakci na zavolání metody \texttt{Show} na \textit{proxy} třídě obrazovky. Metoda \texttt{Show} vytvoří novou instanci implementační třídy, která v~rámci konstruktoru použije engine a \texttt{UI.Root} pro nahrání obrazovky. Při následném skrytí obrazovky v~reakci na volání metody \texttt{Hide} na proxy třídě je implementační třída zahozena pomocí volání \texttt{Dispose} a~nastavení reference na null. V~rámci volání \texttt{Dispose} jsou z~reprezentace uživatelského rozhraní uvnitř enginu odstraněny všechny prvky příslušící této obrazovce. Díky této implementaci nedochází k~hromadění naalokovaných obrazovek i~při cyklickém přepínání mezi několika obrazovkami.



\subsection{Definice vzhledu}
Engine UrhoSharp umožňuje definici rozložení a~vzhledu uživatelského rozhraní jak v~kódu, tak pomocí XML souborů. Podle složitosti jednotlivých obrazovek používáme kombinaci obou těchto možností. Některé jednodušší obrazovky, jako například hlavní menu, mají celý svůj vzhled definován pomocí XML souborů. Složitější obrazovky, měnící svůj vzhled v~reakci na uživatelské akce, mají základní vzhled definován pomocí XML a~následně měněn pomocí kódu dané obrazovky. Použité XML soubory se rozdělují na dva druhy:
\begin{enumerate}
	\item definující vzhled,
	\item definující rozložení.
\end{enumerate}

Soubory definující uživatelské rozhraní platformy pro systém Windows jsou uloženy v~adresáři \texttt{/src/MHUrho/MHUrho.Desktop/Data/UI/} v přílohách práce \ref{sec:appendix}.
Vzhled prvků rozhraní je definován především v~souboru \texttt{MainMenuStyle.xml}, spolu s~několika menšími soubory pojmenovanými \uv{\texttt{...Style.xml}}, definujícími vzhled dynamicky přidávaných prvků. Hlavní styl uživatelského rozhraní \texttt{MainMenuStyle.xml} je nastaven v~kořeni \texttt{UI.Root} jako \textit{defaultní}, je tedy aplikován na všechny prvky v~celém stromu, pokud jim není explicitně přiřazen jiný vzhled.

Rozložení prvků je pro každou obrazovku definováno v~separátním souboru, pojmenovaném \texttt{[Jméno obrazovky]Layout.xml}. Tyto soubory byli vytvořeny za použití editoru enginu Urho3D, který je distribuován spolu s~enginem UrhoSharp. 

\subsection{Přepínání obrazovek}
Přepínání obrazovek je jedním z~hlavních úkolů třídy \texttt{MenuUIManager}. Tato třída poskytuje metody pro přepnutí na každou z~obrazovek menu, spolu s~uložením aktuální obrazovky na zásobník předchozích obrazovek pro možnost návratu. Každá z  metod pro přepnutí obrazovek deklaruje data potřebná pro zobrazení nové obrazovky, která následně poskytne třídě reprezentující danou obrazovku.

Posloupnost možných přechodů obrazovek můžeme vidět na diagramu \ref{fig:screen_structure}. Při spuštění aplikace je počáteční obrazovkou hlavní menu, v diagramu označená jako \textit{Main menu}. Jak můžeme vidět na diagramu, z~tohoto menu se lze dostat na obrazovky \textit{Package picking}, \textit{Load game}, \textit{Options} a \textit{About}. Dále lze z~tohoto menu ukončit celou aplikaci. Každý z~černých přechodů přidává starou obrazovku na zásobník obrazovek a~nastavuje novou obrazovku jako \texttt{CurrentScreen}, tedy aktuální obrazovku. Tímto způsobem poskytuje třída \texttt{MenuUIManager} implementaci tlačítka zpět, které je schopno vracet se v~protisměru černých šipek.

Červené šipky v~diagramu značí přechod do hry, který vyčistí zásobník obrazovek a~zahodí aktuální obrazovku. Tímto je uvolněna veškerá nepotřebná paměť zabraná uživatelským rozhraním menu a~je přenechána pro běh hry.

Dále můžeme vidět přerušované šipky, které značí přechody bez možnosti návratu touto cestou. Při pozastavení úrovně je hra převedena do obrazovky \textit{Pauza}. Tato obrazovka se mírně liší podle toho, zda jsme se do ní dostali dostali z~editované či hrané úrovně. Tento rozdíl je implementován podle návrhového vzoru \texttt{State} \citep[str.~305]{book:gangoffour}.

\begin{figure}[h]
	\centering
	\includegraphics[width=\textwidth]{img/ScreenStructure.png}
	\caption{Obrazovky menu a~přechody mezi nimi.}
	\label{fig:screen_structure}
\end{figure}


Vlastní akce přepnutí obrazovky je implementována vytvořením instance implementační třídy, která v~rámci svého konstruktoru vytvoří danou obrazovku. Toto vytvoření probíhá za použití souborů definujících rozložení prvků, které jsou voláním \texttt{UI.LoadLayoutToElement} použity pro vytvoření objektové reprezentace uživatelského rozhraní uvnitř enginu. Tato objektová reprezentace je následně přístupná i~z~našeho kódu, a~je použitá pro další modifikace, například v~reakci na akci uživatele.

Herní data potřebná pro zobrazení některých obrazovek, například herní balíček při zobrazení výběru úrovní, jsou zpřístupněna jako veřejná \textit{property} proxy tříd. Při zavolání metody \texttt{Show} je kontrolována přítomnost a~validita poskytnutých dat. Implementační třída při své alokaci dostává referenci na proxy třídu, čímž je jí umožněn přístup k~poskytnutým herním datům a~je schopna tato data použít pro zobrazení obrazovky.

\subsection{Automatizace uživatelského rozhraní}

Pro účely testování byl vytvořen systém pro automatické přepínání obrazovek bez zásahu uživatele. Tento systém je možné použít vytvořením XML souboru a~specifikací parametrů příkazové řádky.

Parametrem spouštějícím tento systém je \uv{-ui [d|s] <xmlFilePath>}, který z~kombinace poskytnuté cesty a~přepínačů \uv{d|s} vytvoří cestu, z~které nahraje XML soubor. Přepínače \uv{d|s} určují, zda je cesta \uv{xmlFilePath} relativní vůči adresáři aplikace či adresáři dynamických dat, blíže popsanému v~části \ref{sec:packagedir}. Soubor XML musí být validní vůči schématu \texttt{MenuActions.xsd}. Z~tohoto souboru jsou  nahrány třídy specifikující akce pro jednotlivé obrazovky. Tyto třídy jsou potomkem \texttt{MenuScreenAction} a~jsou vždy určeny pro konkrétní obrazovku. Implementace každé z~obrazovek zná sobě příslušící třídu a~umí v~ní zakódovanou akci vykonat. Dále je vytvořena instance \texttt{ActionManager}, která slouží jako úložiště těchto nahraných tříd a~řídí jejich předávání jednotlivým obrazovkám.

Diagram \ref{fig:uiautomationload} je zjednodušením diagramu \ref{fig:startup} a~ukazuje části inicializace platformy, ve kterých je prováděn tento systém. Jak můžeme vidět, při startu aplikace bez parametrů příkazové řádky je tento systém úplně přeskočen. 

\begin{figure}[h]
	\centering
	\includegraphics[width=\textwidth]{img/MenuActions.png}
	\caption{Nahrávání akcí v~průběhu inicializace platformy.}
	\label{fig:uiautomationload}
\end{figure}

\section{Systém balíčků}
V~části \ref{sec:packagestructure} jsme popsali základní strukturu systému balíčků, jejich popis pomocí XML souboru a~podporované formáty dat pro nahrávání do hry. V~této části popíšeme implementaci této funkcionality.

\subsection{Adresář balíčků}
\label{sec:packagedir}
Při instalaci hry je vytvořen adresář, ve kterém jsou uložena data generovaná aplikací či přidávaná uživatelem do aplikace. Tento adresář je v~kódu označován jako \texttt{DynDataDir}, tedy adresář pro dynamická (měnící se) data. Jako součást tohoto adresáře je při instalaci vytvořen podadresář \texttt{Packages}, jehož účelem je centralizované umístění balíčků.

Obsah tohoto podadresáře můžeme vidět na diagramu \ref{fig:packagesdir}. Z~pohledu platformy nejdůležitějším obsahem je soubor \texttt{Packages.xml}, ve kterém jsou uloženy záznamy o~všech balíčcích dostupných ve hře. Tento soubor je ve formátu XML, splňujícím schéma \texttt{GamePack.xsd}, dostupné v přílohách práce \ref{sec:appendix}. Záznamy je možné do tohoto souboru přidávat dvěma způsoby. Prvním je pomocí grafického rozhraní při běhu platformy, konkrétně stisknutím tlačítka \texttt{Add} na obrazovce pro výběr balíčků \texttt{PackagePickingScreen}. Druhým je manuální editace soubor \texttt{Packages.xml}, při které musí být dodrženo schéma souboru. Tímto způsobem je tedy možné přidat balíčky mimo běh platformy. 

\begin{figure}[h]
	\centering
	\fontsize{6pt}{7pt}\selectfont
	\def\svgwidth{0.7\textwidth}
	\input{img/DynData.pdf_tex}
	\caption{Typická struktura adresáře \texttt{DynDataDir}.}
	\label{fig:packagesdir}
\end{figure}

Tato struktura ukládání balíčků je vystavěna s~myšlenkou uložení všech balíčků jako podadresářů adresáře \texttt{Packages}. Tato vlastnost ale není nijak vynucována či kontrolována, je tedy možné přidat i~balíčky mimo tento adresář. Jedinou překážkou je nutnost použití relativní cesty vůči adresáři \texttt{Packages}.

\subsection{PackageManager}
\label{sec:packagemanager}
Třída \texttt{PackageManager} je základem celého systému balíčků. Jak můžeme vidět v~části \ref{sec:init} na obrázku \ref{fig:startup}, vytvoření instance \texttt{PackageManager} je součástí inicializace platformy. První akcí takto vytvořené instance je načtení souboru \texttt{Packages.xml}, popsaného v~předchozí části \ref{sec:packagedir}.

\begin{figure}[h]
	\centering
	\fontsize{8pt}{11pt}\selectfont
	\def\svgwidth{\textwidth}
	\input{img/PackageManager.pdf_tex}
	\caption{Datová struktura balíčků.}
	\label{fig:packagemanager}
\end{figure}

Na diagramu \ref{fig:packagemanager} můžeme vidět datovou strukturu systému balíčků. Při načtení souboru \texttt{Packages.xml}, případně při přidání balíčku do běžící hry, je pro každý balíček vytvořena jedna instance třídy \texttt{GamePackRep}. Tato instance a data v ní obsažená jsou použity pro prezentaci tohoto balíčku uživateli v seznamu balíčků na obrazovce \texttt{PackagePickingScreen}, ze kterých si následně může uživatel vybrat balíček pro načtení. Instance třídy \texttt{GamePackRep} obsahuje pouze data potřebná pro její použití, tedy pro prezentaci balíčku a~případné následné načtení balíčku. Těmito daty jsou:

\begin{itemize}
	\item jméno,
	\item popis,
	\item ikona,
	\item cesta k~XML souboru.
\end{itemize}

Tato data jsou z~XML souboru balíčku načtena při startu hry či přidání balíčku a~zůstávají přítomna v~paměti až do ukončení hry. Zbylá data balíčku jsou načítána až po zvolení jednoho konkrétního balíčku hráčem. Ve hře je vždy nejvýše jeden načtený balíček, dostupný jako property \texttt{ActivePackage} na třídě \texttt{PackageManager}.

Druhou funkcí třídy \texttt{PackageManager} je obalení funkcionality poskytované třídou \texttt{ResourceCache} enginu UrhoSharp. Tato třída umožňuje načítaní všech druhů assetů podporovaných enginem a~jejich cachování. Bohužel přístup k~této třídě je omezen na hlavní vlákno aplikace, tedy při přístupu z~jiného vlákna, ve kterém implementujeme načítání hry, dochází k~pádu aplikace. Pro zamezení chyb při programování načítání jsme vytvořili identické rozhraní na třídě \texttt{PackageManager}, které automaticky přesměrovává všechna volání na hlavní vlákno aplikace, kde následně zavolá odpovídající metodu \texttt{ResourceCache}.


\subsection{GamePack}
\label{sec:gamepack}


Třída \texttt{GamePack} představuje kompletně načtený balíček. Jak můžeme vidět na diagramu \ref{fig:packagemanager}, obsahuje tato třída referenci na instanci \texttt{GamePackRep} reprezentující daný balíček. Dále obsahuje tato data:

\begin{itemize}
	\item typy dlaždic,
	\item typy jednotek,
	\item typy budov,
	\item typy projektilů,
	\item typy surovin,
	\item typy hráčů,
	\item typy logik úrovní,
	\item seznam úrovní,
	\item defaultní typ dlaždice,
	\item textury ikon.
\end{itemize}

Tato data jsou načítána podle jejich popisu v~XML souboru, splňujícího schéma \texttt{GamePack.xsd}, které můžete nalézt v přílohách práce \ref{sec:appendix}. Soubor je vůči tomuto schématu validován jak při načítání reprezentanta balíčku (\texttt{GamePackRep}), tak při načítání celého balíčku. Pro zjednodušení manipulace s~tímto XML souborem v~naší platformě jsme vytvořili třídy v~souboru \texttt{GamePackXml.cs}, které tvoří objektovou reprezentaci tohoto schématu. Pokud je tedy v~elementu \texttt{TileType} požadován potomek \texttt{texturePath}, pak \textit{singleton} instance třídy \texttt{TileTypeXml} bude obsahovat položku \texttt{TexturePath}, kterou je následně možné použít v~metodě \texttt{XElement.Element()} jako argument pro získání tohoto potomka. Tímto způsobem je izolován zbytek kódu platformy od změn ve formátu či jménech elementů ve schématu. Další výhodou tohoto přístupu, a~důvodem použití návrhového vzoru \textit{Singleton} \citep[str.~127]{book:gangoffour}, je reprezentace dědičnosti ve schématu pomocí dědičnosti v~jeho objektové reprezentaci.

Při hře slouží třída \texttt{GamePack} jako databáze typů, tedy obsahuje metody pro získání referencí na typy z~výčtu výše podle jména či podle číselného identifikátoru. Jméno i~identifikátor jsou načítány z~XML souboru, ve kterém je jejich přítomnost vyžadována schématem. Systém typů je blíže popsán v~části \ref{sec:types}.

Poslední funkcí je přístup k~úrovním obsaženým v~balíčku. Úrovně jsou, podobně jako balíčky samotné, před vlastním načtením reprezentovány separátní třídou \texttt{LevelRep}. Tato třída, podobně jako \texttt{GamePackRep}, obsahuje data potřebná pro výběr a~nastavení úrovně před jejím spuštěním. Dále tato třída umožňuje manipulaci s~úrovní z~pohledu assetu, tedy její ukládání při editaci, vytvoření kopie pro editaci, načtení pro hru atd. Toto chování je implementováno podle návrhového vzoru \textit{State} \citep[str.~305]{book:gangoffour}.

\subsection{Assety}
\label{sec:assets}
Pojmem assety označujeme v~naší práci veškerá data použitelná pro implementaci hry. Patří mezi ně například:

\begin{itemize}
	\item 3D modely,
	\item textury,
	\item XML data,
	\item části logiky poskytované enginem.
\end{itemize}

Každý typ entity, tedy budovy, jednotky či projektilu, má definovanou svou množinu assetů, které jej reprezentují při zobrazení či při výpočtu kolizí. Pro dodání assetů instancím těchto typů poskytujeme tři způsoby definice.

Prvním způsobem je explicitní specifikace assetů přímo v~XML souboru balíčku. Tento způsob je implementován třídou \texttt{ItemsAssetContainer}, která z~XML elementu vyčte všechna potřebná data a~při požadavku vytvoří reprezentaci entity v~herním enginu. Bližší popis reprezentace hry v~herním enginu můžete nalézt v~části \ref{sec:engineview}.

Druhý způsob využívá funkcionality poskytované herním enginem a~s~ním distribuovaným editorem. V~tomto editoru lze definovat reprezentaci entity a~následně tuto definici serializovat do XML či binárního souboru, do takzvaného \textit{prefab}, neboli prefabrikátu. Tento způsob je implementován dvěma třídami, a to \texttt{XmlPrefabAssetContainer} a \texttt{BinaryPrefabAssetContainer}, které obalí soubor definující reprezentaci entity a~při žádosti o~vytvoření instance entity použijí engine a~tyto soubory pro její vytvoření.

Použitý způsob vytváření je specifikován v~XML popisu typu entity, a~v~závislosti na zvoleném způsobu jsou dále požadována další data, specifická pro daný způsob. Použití XML či Binary prefab umožňuje využití všech možností UrhoSharp enginu. Námi implementovaný \texttt{ItemsAssetContainer} umožňuje zjednodušenou specifikaci bez externích nástrojů, jakým je editor Urho3D, zato ale omezuje použitelné součásti enginu na tuto podmnožinu:

\begin{itemize}
	\item \texttt{Static model},
	\item \texttt{Animated model},
	\item \texttt{Collision shape},
	\item nastavení \texttt{Scale}.
\end{itemize}

Ke každému z~modelů navíc umožňuje definovat jeho textury. Tato podmnožina je podle nás dostatečná pro tvorbu graficky jednodušších her,

\subsection{Načítání balíčku}
\label{sec:packageloading}

Načítání balíčku je implementováno metodou \texttt{Load}, jejíž prototyp je:

\begin{lstlisting}[
	style=csharp,
	emph={[1]GamePack, GamePackRep, XmlSchemaSet}, 
	emph={[2]IProgressEventWatcher}
]
public static async Task<GamePack> Load(
	string pathToXml,
	GamePackRep gamePackRep,
	XmlSchemaSet schemas,
	IProgressEventWatcher loadingProgress = null)
\end{lstlisting}

Už podle deklarace metody vidíme, že se implementace pokouší o~asynchronní načítání balíčku. Nejedná se o~úplné asynchronní načítání v~pravém slova smyslu, protože drtivou většinu dat je nutno načítat na hlavním vlákně z~důvodu restrikcí herního enginu, blíže popsaným v~části \ref{sec:packagemanager}. Hlavním cílem je tedy rozdělit načítání na části, mezi kterými je platforma schopna nadále zpracovávat vstup od hráče. Tímto se snažíme zamezit tzv. \uv{zamrzání} aplikace, kdy aplikace z~pohledu uživatele nic nedělá a~odmítá reagovat na jakýkoli vstup hráče. V~prvních verzích bez tohoto přístupu jsme naráželi na problém, kdy systém Windows považoval náš proces za mrtvý a~navrhoval nám jej zabít. Touto implementací jsme tomuto problému zabránili a~navíc jsme umožnili notifikovat hráče o~průběhu načítání.

Z~pohledu implementace vlastního načítání je princip vcelku jednoduchý. Posloupnost akcí je následovná:

\begin{enumerate}
	\item načíst a~validovat XML soubor;
	\item načíst typy dlaždic, jednotek, budov, projektilů, surovin;
	\item načíst typy logik hráčů a~úrovní;
	\item načíst textury ikon;
	\item načíst reprezentanty úrovní,
	\item zavřít XML soubor.
\end{enumerate}

Při jakékoli chybě načítání, ať už nevalidnímu XML či chybějícímu souboru některého assetu, je načítání zastaveno, dosud načtená data znovu odalokována, chyba zalogována a~zpět vypuštěna výjimka typu \texttt{PackageLoadingException}, signalizující chybu při načítání balíčku.

\section{Logika hry}
V~této sekci popíšeme implementaci vlastního průběhu hry, nejdříve z~pohledu herního enginu a~následně z~pohledu naší platformy. Oba tyto pohledy jsou poskytnuty tvůrcům balíčků pro tvorbu pluginů.

\subsection{Herní engine}
\label{sec:engineview}

\begin{figure}[h]
	\centering
	\fontsize{8pt}{11pt}\selectfont
	\def\svgwidth{\textwidth}
	\input{img/EngineRepre.pdf_tex}
	\caption{Reprezentace hry z~pohledu herního enginu.}
	\label{fig:scenegraph}
\end{figure}


Diagram \ref{fig:scenegraph} ukazuje reprezentaci hry z~pohledu herního enginu. Aktuálně spuštěná úroveň je v~herním enginu reprezentována tzv.~\textit{Scene graph} strukturou, tedy grafem scény. Tento graf je strom s~vrcholy typu \texttt{Node}. Kořenem tohoto stromu je instance třídy \texttt{Scene}, která je specializací \texttt{Node}.

Diagram \ref{fig:scenegraph} ukazuje část grafu scény, ve které můžeme vidět jeho základní strukturu. Názvy v~instancích třídy \texttt{Node} ukazují, který herní objekt daná instance reprezentuje. Všechny jsou ale instancí stejné třídy \texttt{Node}. Oproti tomu názvy komponentů ukazují jméno třídy, jejíž jsou instancí. Tyto třídy jsou všechny potomkem třídy \texttt{Component}. Tento rozdíl vychází z~návrhu enginu, ve kterém je k~přidávání uživatelské logiky zamýšlena třída \texttt{Component}. Třída \texttt{Node} obecně není určena pro vytváření potomků.

Na diagramu můžeme vidět stav hry, ve kterém jsou v~herním světě umístěny dvě budovy, jednotka a~projektil. Vztah \texttt{Component} a \texttt{Node} v~naší platformě je blíže popsán v~části \ref{sec:platformimpl}. Dále můžeme vidět mapu a~její reprezentaci. Tato reprezentace je blíže popsána v~částech \ref{sec:mapimpl} a \ref{sec:mapimpldoc}. Jak můžeme vidět na diagramu, lze \texttt{Node} použít k~reprezentaci jedné entity, částí entit nebo skupiny entit. Příkladem reprezentace částí může být právě rozdělení mapy na \uv{Chunky}. Budovy, jednotky či projektily mohou mít pod touto jednou \texttt{Node} také přidány podstrom, reprezentující jejich součásti. Za reprezentaci skupiny prvků jednou \texttt{Node} můžeme považovat situaci, kdy kamera sleduje jednotku. V~tu chvíli \texttt{Node} dané jednotky reprezentuje jak jednotku samotnou, tak kameru, a~obě současně přesouvá v~herním světě.



Každá instance třídy \texttt{Node} má několik atributů, které umožňují její umístění v~herním světě a~další manipulaci. Nejdůležitějšími z~těchto atributů jsou:

\begin{itemize}
	\item \texttt{Position} (pozice),
	\item \texttt{Rotation} (rotace),
	\item \texttt{Scale} (velikost),
	\item \texttt{Enabled}.
\end{itemize}

Již podle názvu slouží \texttt{Position} a \texttt{Rotation} pro umístění dané \texttt{Node} do herního světa a~určení jejího otočení. Obě tyto vlastnosti jsou počítány vůči rodičovské \texttt{Node}, jak můžeme vidět na obrázku \ref{fig:relativeposition}. Atribut \texttt{Scale} určuje závislost souřadnic v~podstromu této \texttt{Node} vzhledem k~souřadnicím v~podstromu otce této \texttt{Node}. Na ukázce můžeme vidět dvě \texttt{Node}, kdy levá horní \texttt{Node} je otcem pravé spodní \texttt{Node}. Otec je umístěn na pozici \((1,0,1)\) a~má nastaven \texttt{Scale} na \((2,2,2)\). Potomek má vlastní atribut \texttt{Position} nastaven na \((1,0,1)\). Výslednou pozicí potomka v~herním světě, označovanou jako \texttt{WorldPosition}, pak získáme tímto výpočtem:\[\scalebox{0.9}{Potomek.WorldPosition = Otec.WorldPosition + Potomek.Position * Předek.Scale}\] Díky tomu bude výsledná pozice potomka v~herním světě \((3,0,3)\). Výpočet použité \texttt{WorldPosition} otce proběhne stejným způsobem.

\begin{figure}[h]
	\centering
	\fontsize{8pt}{11pt}\selectfont
	\def\svgwidth{\textwidth}
	\input{img/RelativePosition.pdf_tex}
	\caption{Ukázka výpočtu pozice \texttt{Node} v~herním světě.}
	\label{fig:relativeposition}
\end{figure}

Pro implementaci chování hry lze na každou instanci třídy \texttt{Node} připojit potomky třídy \texttt{Component}. Hlavním způsobem implementace logiky při použití herního enginu UrhoSharp je vytvoření vlastních specializací třídy \texttt{Component}, implementace virtuálních metod a~přidání této naší komponenty k~instanci \texttt{Node} v~grafu scény. Tímto způsobem je vytvořena také implementace naší platformy, blíže popsána v~části \ref{sec:platformimpl}. 

Engine sám poskytuje několik základních typů komponent, implementujících nejčastější potřeby her. Těmito typy jsou:

\begin{itemize}
	\item \texttt{StaticModel} pro vykreslování bez animací,
	\item \texttt{AnimatedModel} pro vykreslování s~animacemi,
	\item \texttt{RigidBody} pro simulaci fyziky,
	\item \texttt{CollisionShape} pro výpočet kolizí ve spolupráci s \texttt{RigidBody},
	\item \texttt{Camera} pro vykreslení na obrazovku,
	\item \texttt{Light} pro přidání světla.
\end{itemize}

Jak bylo řečeno v~části \ref{sec:assets}, umožňuje platforma tvůrcům balíčků specifikovat assety příslušící jednotlivým typům herních prvků mnoha způsoby. Všechny tyto způsoby ale využívají tento popis právě k~vytvoření instancí \texttt{Node} a \texttt{Component}, které jsou spojeny a~inicializovány podle uložených dat. 

Vlastní výpočet stavu hry je iniciován enginem, který ve všech prvcích grafu scény, tedy \texttt{Node} i \texttt{Component}, zavolá jejich virtuální metodu \texttt{OnUpdate}. Tato metoda je volána s~parametrem \texttt{timeStep}, který představuje čas uběhlý od předchozího výpočtu stavu. V~rámci této metody provádí standardní i~uživatelské komponenty výpočet následujícího stavu.

Zmíněný atribut \texttt{Enabled} třídy \texttt{Node} ovlivňuje šíření volání metody \texttt{OnUpdate}. Pokud je tento atribut nastaven na hodnotu \texttt{false}, není na dané instanci \texttt{Node} ani na jejím podstromu proveden \uv{update}. Tento atribut je přítomný i~na třídě \texttt{Component}, na které ovládá volání \texttt{OnUpdate} na dané instanci \texttt{Component}.

\subsection{Pohled platformy}
\label{sec:platformimpl}

\begin{figure}[h]
	\centering
	\fontsize{8pt}{11pt}\selectfont
	\def\svgwidth{\textwidth}
	\input{img/PlatformStructure.pdf_tex}
	\caption{Třída \texttt{LevelManager}.}
	\label{fig:platform}
\end{figure}

Z~pohledu platformy se hra skládá z~několika částí, ilustrovaných na diagramu \ref{fig:platform}. Vztah některých těchto částí s~grafem scény popsaným v~části \ref{sec:engineview} můžeme vidět na diagramu \ref{fig:scenegraph}.

Centrální částí je instance třídy \texttt{LevelManager}, která slouží jako přístupový bod ke všem součástem tvořících implementaci platformy. Dále tato třída slouží jako databáze všech budov, jednotek, projektilů a~hráčů přítomných ve hře. Pro budovy, jednotky a~projektily umožňuje dále jejich vytvoření či zničení. Jak můžeme vidět na diagramu \ref{fig:scenegraph}, je tato třída potomkem třídy \texttt{Component} a~je připojena k \texttt{Node} obsahující jako podstrom celý zbytek herního světa. Tímto způsobem je pohled platformy spojen s~pohledem herního enginu.

Každá entita, tedy každá budova, jednotka či projektil je reprezentována instancí odpovídající třídy implementující chování tohoto druhu. Těmito třídami jsou \texttt{Building}, \texttt{Unit} a \texttt{Projectile}. Jak můžeme vidět na diagramu \ref{fig:componenthierarchy}, jsou všechny tyto třídy potomkem třídy \texttt{Entity}, která je sama potomkem třídy \texttt{Component}. Tímto způsobem je provázána reprezentace entity z~pohledu herního enginu a~z~pohledu platformy. 


\begin{figure}[h]
	\centering
	\fontsize{9pt}{11pt}\selectfont
	\def\svgwidth{0.7\textwidth}
	\input{img/PlatformHier.pdf_tex}
	\caption{Potomci třídy \texttt{Component} v~platformě.}
	\label{fig:componenthierarchy}
\end{figure}

\subsection{Typy}
\label{sec:types}
Platforma zavádí systém typů herních prvků. Tento systém je separován od typového systému jazyka a~je aplikován pro tyto prvky:
\begin{itemize}
	\item logiky úrovně (\texttt{LevelLogicType}),
	\item hráče (\texttt{PlayerType}),
	\item dlaždice (\texttt{TileType}),
	\item budovy (\texttt{BuildingType}),
	\item jednotky (\texttt{UnitType}),
	\item projektily (\texttt{ProjectileType}),
	\item suroviny (\texttt{ResourceType}).
\end{itemize}

Pro každý z~druhů prvků definuje typ jejich vzhled a/nebo chování. Vzhled je definován pomocí množiny assetů specifikovaných v~XML elementu definujícím tento typ, které jsou následně použity při vytváření instancí tohoto typu. Chování je definováno pomocí systému pluginů, popsaném v~části \ref{sec:plugins}. 

Výjimkou z~tohoto pravidla jsou typy surovin, které ve hře nemají vlastní instance a~jsou používány pouze jako klíč pro přístup k~množství daného typu suroviny vlastněného hráčem.

Každá instance výše vyjmenovaných druhů herních prvků obsahuje referenci na svůj typ. Příklad můžeme vidět na diagramu \ref{fig:pluginstructure}, na kterém vidíme instance \texttt{Building}, tedy budov, které obsahují reference na instance \texttt{BuildingType} reprezentující typ budov \textit{Gate}, tedy brána, nebo \textit{Wall}, tedy zeď.

Typy jsou jednou z~hlavních součástí obsahu balíčku. Instance reprezentující jednotlivé typy jsou vytvořeny při načtení balíčku a~zanikají při odalokování balíčku. 

\subsection{Pluginy}
\label{sec:plugins}
Hlavním cílem naší práce bylo umožnit tvůrcům balíčků použít jazyk C\# pro tvorbu logiky hry ve formě pluginů. Návrh tohoto systému a~použité technologie byly popsány v~části \ref{sec:logicandplugins}. Platforma definuje dva druhy pluginů:

\begin{enumerate}
	\item pluginy typů,
	\item pluginy instancí.
\end{enumerate}

Ukázku zapojení těchto pluginů do struktury hry můžeme vidět na diagramech \ref{fig:simplepluginstructure} a \ref{fig:pluginstructure}. Diagram \ref{fig:simplepluginstructure} ukazuje základní princip propojení pluginů, instancí herních prvků a~typů těchto herních prvků. Tyto vztahy jsou dále popsány v~následující části \ref{sec:typeplugins}. Diagram \ref{fig:pluginstructure} obsahuje ukázku několika budov různých typů v~herním světě spolu s~jejich pluginy.

\begin{figure}[h]
	\centering
	\fontsize{8pt}{11pt}\selectfont
	\def\svgwidth{\textwidth}
	\input{img/SimplePluginStructure.pdf_tex}
	\caption{Princip zapojení pluginů do struktury hry.}
	\label{fig:simplepluginstructure}
\end{figure}

\subsubsection{Pluginy typů}
\label{sec:typeplugins}
Jak jsme popsali v~části \ref{sec:types}, velká část herních prvků má určený svůj typ. XML definice každého z~těchto typů podle schématu vyžaduje specifikaci assembly, která je při načítán balíčku pomocí systému \textit{Reflection} nahrána do procesu platformy. Tato akce je blíže popsána v~části \ref{sec:logicandplugins}. V~této assembly je následně podle \textit{ID} a~jména typu nalezena třída, která slouží jako typový plugin tohoto typu. Tato třída musí být potomkem jedné z~těchto tříd:
\begin{itemize}
	\item \texttt{LevelLogicTypePlugin},
	\item \texttt{PlayerAITypePlugin},
	\item \texttt{BuildingTypePlugin},
	\item \texttt{UnitTypePlugin},
	\item \texttt{ProjectileTypePlugin}.
\end{itemize}

Dále je vyžadováno, aby tato třída měla bezparametrický konstruktor. Inicializace je implementována separátní metodou \texttt{Initialize}, které je předán XML element \texttt{<extension>} z~definice typu. Tento element slouží pro uložení tvůrcem balíčku definovaných dat a~jeho obsah není nijak omezen či validován.

Příklad můžeme vidět na diagramu \ref{fig:pluginstructure} a \ref{fig:typeplugincreation}, kde jsou v~baličku definovány dva typy budov, \textit{Gate}, neboli brána, a \textit{Wall}, neboli zeď. Při vytváření instance \texttt{BuildingType} pro každý z~těchto typů je z~cesty uvedené v~XML elementu \texttt{assemblyPath} v~záznamu daného typu nahrána odpovídající assembly. V~této assembly jsou nalezeny všechny typy odvozené od jedné z~výše uvedených tříd platformy, v~naší ukázce odvozené od třídy \texttt{BuildingTypePlugin}. Od každé z~těchto tříd je vytvořena instance, jejíž hodnoty v~položkách \texttt{Name} a \texttt{ID} jsou porovnány s hodnotami načtenými z~XML typu. Při rovnosti obou hodnot je daná instance připojena do atributu \texttt{Plugin} instance \texttt{BuildingType} reprezentující daný typ.

\begin{figure}[h]
	\centering
	\fontsize{8pt}{11pt}\selectfont
	\def\svgwidth{\textwidth}
	\input{img/TypePluginCreation.pdf_tex}
	\caption{Načítání pluginu typu.}
	\label{fig:typeplugincreation}
\end{figure}

Jedním z hlavních použití pluginu typu je tvorba pluginů pro instance herních prvků daného typu. Tvorba těchto instančních pluginů je implementována metodami \texttt{CreateNewInstance} a \texttt{GetInstanceForLoading}. Tento systém byl navržen podle návrhových vzorů \textit{Factory method} \citep[str.~107]{book:gangoffour} a \textit{Abstract Factory} \citep[str.~87]{book:gangoffour}.

Dále je např. u \texttt{BuildingTypePlugin} definována metoda \texttt{CanBuild}, která dostává aktuální úroveň a~definovanou pozici v~mapě a~je používána pro zjištění, zda je možné v~této pozici vytvořit budovu.

\subsubsection{Pluginy instancí}
Každá instance následujících tříd má při svém vytváření přiřazenu svoji privátní instanci třídy z~balíčku jako plugin:
\begin{enumerate}
	\item\texttt{LevelManager}, 
	\item\texttt{Player}, 
	\item\texttt{Building},
	\item\texttt{Unit},
	\item\texttt{Projectile}.
\end{enumerate}
   
Třída z~balíčku sloužící jako plugin instance některé z~předcházejících tříd musí být potomkem odpovídající z~těchto tříd:

\begin{enumerate}
	\item\texttt{LevelLogicInstancePlugin}, 
	\item\texttt{PlayerAIInstancePlugin}, 
	\item\texttt{BuildingInstancePlugin}, 
	\item\texttt{UnitInstancePlugin}, 
	\item\texttt{ProjectileInstancePlugin}.
\end{enumerate}

Tyto třídy mají definovány virtuální metody, které jsou volány při určitých událostech v~herním světě, které se týkají daného herního prvku. Implementací těchto metod mohou tvůrci balíčků reagovat na tyto události a~tím implementovat chování těchto herních prvků.

Jednou z~hlavních metod pro implementaci logiky v~pluginech je metoda \texttt{OnUpdate}. Tato metoda je volána z~metod \texttt{OnUpdate} výše vyjmenovaných potomků třídy \texttt{Component}. Třída \texttt{Component} a~její metoda \texttt{OnUpdate} jsou blíže popsány v~části \ref{sec:engineview}. Metoda \texttt{OnUpdate} u~pluginů má stejnou sémantiku, je tedy volána při každém výpočtu stavu hry, kde jako parametr dostává čas uběhlý od předchozího výpočtu stavu.

Množina metod je různá pro různé druhy herních prvků, příkladem ale může být událost přidání jednotky hráči, odstranění jednotky hráči či změna objemu surovin vlastněných hráčem.

Jak bylo popsáno v~části \ref{sec:types}, obsahuje každá instance herního prvku referenci na svůj typ. Tento typ dále obsahuje referenci na svůj typový plugin, jak bylo popsáno v~části \ref{sec:typeplugins} a~jak můžeme vidět na diagramech \ref{fig:pluginstructure} a \ref{fig:typeplugincreation}. Pro vytvoření pluginu pro vytvářenou instanci herního prvku je použita jedna z \texttt{Factory} metod pluginu typu. 

\begin{figure}[h]
	\centering
	\fontsize{8pt}{11pt}\selectfont
	\def\svgwidth{\textwidth}
	\input{img/PluginStructure.pdf_tex}
	\caption{Ukázka zapojení pluginů budov do struktury hry.}
	\label{fig:pluginstructure}
\end{figure}

\section{Mapa}
\label{sec:mapimpldoc}

\begin{figure}[h]
	\centering
	\fontsize{8pt}{11pt}\selectfont
	\def\svgwidth{\textwidth}
	\input{img/Mapimpl.pdf_tex}
	\caption{Implementace mapy.}
	\label{fig:mapimpl}
\end{figure}

Jak můžeme vidět na diagramu \ref{fig:mapimpl}, mapa je v~naší platformě reprezentována instancí třídy \texttt{Map}. Tato instance je přístupná všem součástem platformy i~pluginům jako položka třídy \texttt{LevelManager}, jak můžeme vidět na diagramu \ref{fig:platform}. 

Jak jsme popsali v~části \ref{sec:mapimpl}, je naše implementace mapy obdélníkového tvaru, rozdělená na čtvercové dlaždice. Tyto dlaždice můžeme na diagramu \ref{fig:mapimpl} vidět jako privátní položku třídy \texttt{Map}. Dlaždice jsou uloženy v~jednorozměrném poli, k~jehož indexaci poskytuje třída \texttt{Map} několik pomocných metod, které převádějí pozici v~souřadnicích herního světa do indexu dlaždice v~tomto poli. 

Při popisování a~implementaci mapy jsou využívány termíny jako horní levý roh, pravý spodní roh, vršek dlaždice a~podobné. Tyto termíny vycházejí z~představy ilustrované obrázkem \ref{fig:mapreprelogic}, kdy rovinu mapy položíme do roviny monitoru a~levý horní roh monitoru určíme jako počátek, tedy bod \((0,0)\). Tato představa vychází z~jedné z~prvních implementací naší platformy, kdy byla mapa reprezentována pouze dvojrozměrně a~přišlo nám logické využívat stejný systém souřadnic jako je používán při definici prvků v~uživatelském rozhraní. Bohužel tuto představu nelze jednoduše reprezentovat ve 3D prostoru, kdy pro pohled světa odpovídající výše popsané představě, tedy levý horní roh v~levém horním rohu obrazovky, je nutné dívat se na mapu ze spodní strany, tedy s~kamerou pod úrovní terénu. Dalším problémem ve 3D je poloha herní roviny, která se nachází v~rovině definované osami X a~Z, kde osa Y udává výšku nad rovinou. Toto se projevuje u~některých metod při pojmenování jejich parametrů a~jejich volání, kdy souřadnice Y dvourozměrného vrcholu je předávána do parametru se jménem Z~metody operující v~rovině mapy. Pro úplnost tedy:

\begin{itemize}
	\item levá strana je strana s~nižší souřadnicí X,
	\item pravá strana je strana s~vyšší souřadnicí X,
	\item horní strana je strana s~nižší souřadnicí Z,
	\item dolní strana je strana s~vyšší souřadnicí Z.
\end{itemize}

Dále můžeme na obrázku \ref{fig:mapreprelogic} vidět dva různé druhy dlaždic. Toto rozdělené bude popsáno v~následující části \ref{sec:tiles}.

\begin{figure}[h]
	\centering
	\fontsize{8pt}{11pt}\selectfont
	\def\svgwidth{\textwidth}
	\input{img/MapRepreLogic.pdf_tex}
	\caption{Pohled třídy \texttt{Map} na reprezentaci mapy.}
	\label{fig:mapreprelogic}
\end{figure}
\subsection{Dlaždice}
\label{sec:tiles}
Jednotlivé dlaždice mapy jsou reprezentovány instancí třídy \texttt{Tile}. Tyto instance obsahují několik atributů popisujících danou dlaždici:

\begin{enumerate}
	\item seznam jednotek přítomných na dlaždici,
	\item referenci na budovu přítomnou na dlaždici,
	\item referenci na typ dlaždice,
	\item pozici v~herním světě,
	\item výšku každého ze svých rohů.
\end{enumerate}

Jak můžeme vidět na obrázku \ref{fig:mapreprelogic}, jsou všechny dlaždice čtvercového tvaru s~velikostí hrany 1. Pro identifikaci instancí třídy \texttt{Tile} používáme souřadnice levého horního rohu dlaždice. Protože je tento výběr rohu v~podstatě náhodný, poskytuje třída \texttt{Tile} položku \texttt{MapLocation}, odstiňující nás od výběru rohu reprezentujícího dlaždice.

Dále nám umístění rohů dlaždic na celočíselné souřadnice a~jejich jednotková velikost umožňují jednoduché zjištění dlaždice obsahující libovolný bod v~herním světě. Použitím operace dolní celé části na souřadnice \textit{X} a \textit{Z} daného bodu získáme horní levý roh dlaždice, která tento bod obsahuje.

Pro redukci duplikace informací obsahuje každá dlaždice informaci o~výšce pouze svého levého horního rohu. Výšky ostatních rohů jsou získávány od sousedních dlaždic, jejichž levé rohy se nachází ve stejné pozici jako zkoumaný roh dané dlaždice. Tímto způsobem máme zajištěnu celistvost terénu a~předcházíme tak možným programátorským chybám. Nevýhodou této implementace je nutnost speciálního druhu dlaždic na okraji mapy, které můžeme vidět na obrázku \ref{fig:mapreprelogic} označené šedou barvou. V~naší implementaci jsou tyto dlaždice reprezentovány instancemi třídy \texttt{BorderTile}. Tyto dlaždice nejsou viditelné z~pohledu hráče či pluginů, jejich jedinou funkcí je udržování informace o~výšce všech svých rohů a~tím i~rohů sousedních herních dlaždic. 


\subsection{Zobrazení}
Zobrazení mapy je implementováno třídou \texttt{MapGraphics}, která je vnitřní privátní třídou třídy \texttt{Map}. Tento vztah je použit pro přístup k~privátním položkám třídy \texttt{Map} a~zjištění celkového stavu mapy. Obrázek \ref{fig:mapdisplay} ilustruje systém zobrazení mapy. Jak můžeme vidět, mapa je rozdělena na určitý počet částí stejné velikosti. Tyto části označujeme jako \textit{Chunk} a~v~kódu jsou reprezentovány instancemi třídy \texttt{MapChunk}. Důvody rozdělení mapy a~omezení z~toho plynoucí jsou rozebrány v~části \ref{sec:mapimpl}. 

Rozdělení na chunky je pouze grafickou záležitostí, proto se netýká okrajových dlaždic mapy, jak můžeme vidět na diagramu \ref{fig:mapdisplay}. Díky tomu se omezení velikosti mapy, popsané v~části \ref{sec:mapimpl} týká rozměrů samotné hrací plochy.

\begin{figure}[h]
	\centering
	\fontsize{10pt}{12pt}\selectfont
	\def\svgwidth{0.7\textwidth}
	\input{img/ChunksDia.pdf_tex}
	\caption{Princip přiřazení dlaždic do chunků.}
	\label{fig:mapdisplay}
\end{figure}

Každý z~chunků odpovídá jedné instanci \texttt{Node}, která je potomkem \texttt{Node} mapy. Toto rozdělení můžeme vidět na obrázku \ref{fig:scenegraph}. Každá z~těchto \texttt{Node} obsahuje právě jeden \texttt{StaticModel}, použitý pro zobrazení odpovídajícího chunku. Tento model je vytvořen v~průběhu volání konstruktoru třídy \texttt{MapChunk}, za využití dynamické generace vertex a~index bufferů. Jak bylo popsáno v~části \ref{sec:mapimpl}, používá naše implementace pro vygenerování a~úpravy grafické reprezentace mapy přímý \textit{unsafe} přístup k~vertex a~index bufferům, v~kódu reprezentovaných instancemi tříd \texttt{VertexBuffer} a \texttt{IndexBuffer}.


Ukázku chunků ve hře můžeme vidět na obrázku \ref{fig:chunks}, kde se v~uživatelem nastavené vzdálenosti od kamery přestávají chunky vykreslovat, čímž snižují výpočetní výkon vykreslování.

\begin{figure}[h]
	\centering
	\includegraphics{Chunks2}
	\caption{Ukázka limitu vzdálenosti vykreslování \textit{Chunků}.}
	\label{fig:chunks}
\end{figure}

Jak jsme popsali v~části \ref{sec:mapgraphicsanalaysis}, je každá dlaždice reprezentována čtyřmi vertexy. Z~těchto čtyřech vertexů jsou následně za použití indexů v \texttt{IndexBufferu} vytvořeny dva trojúhelníky. Rozdělení na trojúhelníky je provedeno podle vyšší diagonály, tedy podle diagonály, která je uprostřed dlaždice výše. 

Chunky dále umožňují zamknout jejich vertex a~index buffer do paměti pro umožnění modifikace výšek dlaždic. Při každé změně výšky vertexů dlaždice je výše zmíněné rozdělení dlaždic na trojúhelníky přepočítáváno.

\section{Hledání cesty}
\label{sec:pathfindingdocu}
Systém hledání cesty je v~naší platformě implementován s~důrazem na rozšiřitelnost tvůrci balíčků, jak bylo popsáno v~části \ref{sec:pathfinding}. Základem této implementace je rozhraní \texttt{IPathFindAlg}, které musí být implementováno kterýmkoli algoritmem, který chce tvůrce balíčku použít jako algoritmus pro hledání cesty ve svém balíčku.

Rozhraní \texttt{IPathFindAlg} je navrženo pro podporu mnoha algoritmů pro hledání nejkratší cesty. Z~tohoto důvodu je spolu s~rozhraním \texttt{IPathFindAlg} vytvořeno rozhraní pro reprezentaci grafu. V~rámci jednoho běhu úrovně je podporován pouze jediný algoritmus, což umožňuje implementaci algoritmu využít znalosti opravdových typů objektů, které získává jako reference na rozhraní. Díky tomu je možné využít přetypování pro přístup ke konkrétním typům objektů specifických pro daný algoritmus.

Jak bylo popsáno v~části \ref{sec:pathfinding}, reprezentaci grafu je možné generovat z~logické reprezentace mapy dvěma způsoby, a~to staticky či dynamicky. Návrh rozhraní v~naší platformě je cílen spíše pro kombinaci dynamické a~statického generování grafu, předpokládáme ovšem, že by bylo možné jeho využití i~pro čistě statické či pro čistě dynamické generování. Rozhraní pro reprezentaci grafu se skládá ze tří částí. Těmito částmi jsou:

\begin{enumerate}
	\item reprezentace vrcholů a~hran,
	\item ohodnocení hran,
	\item reprezentace cesty v~grafu.
\end{enumerate}

\subsection{Reprezentace vrcholů a~hran}
Reprezentace vrcholů a~hran tvoří statickou část generování grafu. Vrcholy grafu jsou tří typů. Prvním je \texttt{ITileNode}, reprezentující dlaždice mapy, druhým je pak \texttt{IBuildingNode}, reprezentující dostupné části budov.

Vzhledem k~tomu, že velikost mapy, a~tedy počet dlaždic není možné měnit, předpokládáme, že vrcholy typu \texttt{ITileNode} bude každý algoritmus generovat jednou, při své konstrukci. 

Existence vrcholů typu \texttt{IBuildingNode} závisí na existenci budov, které tyto vrcholy vytvořily. Předpokládáme tedy, že budou vytvářeny při stavbě budov a~mazány při zničení těchto budov. 

Posledním typem vrcholu je \texttt{ITempNode}, využívaná pro přesnější reprezentaci hrany v~herním světě. Konkrétní sémantika a~životnost vrcholů je ale v~rukou tvůrce konkrétního algoritmu. Naše platforma nijak tyto vlastnosti nekontroluje. 

Každá hrana mezi dvěma vrcholy má přiřazen způsob pohybu. V~aktuální implementaci podporujeme explicitně pouze dva typy pohybu, a~to lineární pohyb a~teleportaci. Význam těchto typů pohybu není daný algoritmem vyhledávání cesty, ale koncovou implementací pohybu jednotek. Platforma poskytuje jednu možnou implementaci pohybu jednotek pomocí komponenty \texttt{WorldWalker}, popsané v~části \ref{sec:defaultcomponents}.

Pro přidávání akcí nad hranami grafu, tedy dvojicemi vrcholů, je definováno rozhraní podle návrhového vzoru \textit{Visitor} \citep[str.~331]{book:gangoffour}. Podpora tohoto návrhového vzoru je požadována po třídách implementujících rozhraní \texttt{INode}, které je předkem rozhraní všech tří druhů vrcholů. Jedním z~možných využití rozhraní \texttt{Visitor} je tzv.~\textit{double dispatch}, který umožňuje jednoduší implementaci \texttt{INodeDistCalculator}. Této implementace využívá třída \texttt{NodeDistCalculator}, která je součástí implementace algoritmu A* poskytované naší platformou. Bližší popis rozhraní \texttt{INodeDistCalculator} můžete nalézt v následující části.

\subsection{Dynamická část generování}
Dynamickou část generování grafu tvoří rozhraní \texttt{INodeDistCalculator}. Instance třídy implementující toto rozhraní je vyžadována pro každé spuštění výpočtu cesty. Konkrétní využití a~vlastnosti třídy implementující toto rozhraní závisí především na algoritmu, pro který je kalkulátor vytvářen. 

Protože naše platforma umožňuje a požaduje využití právě jednoho algoritmu, reprezentovaného třídou implementující rozhraní 
\texttt{IPathFindAlg}, v~průběhu úrovně, může tento algoritmus využít přetypování na svůj konkrétní typ kalkulátoru implementující \texttt{INodeDistCalculator} a~následně použít plných schopností tohoto typu. 


Jedinou požadovanou metodou v~rámci rozhraní \texttt{INodeDistCalculator} je metoda \texttt{GetTime}, která je využívána pro zjištění existence hrany mezi dvěma vrcholy a~v~případě existence pak váhy této hrany. Váha hrany je v~rozhraní označována jako \texttt{time}, tedy čas potřebný pro přesun z~prvního vrcholu na druhý. Tato metoda je využívána ve třetí části \ref{sec:path}, popsané dále.

\subsection{Reprezentace cesty}
\label{sec:path}
Třetí částí reprezentace grafu je reprezentace cest. Cesty jsou reprezentovány pomocí dvou tříd, \texttt{Waypoint}, tedy bodu v~cestě, a~z~nich složené \texttt{Path}, tedy cesty. Při použití v~platformě jsou instance \texttt{Waypoint} chápány jako body, mezi kterými se jednotka pohybuje po úsečce spojující tyto body konstantní rychlostí. Cesta je potom posloupností těchto úseček. Každý z \texttt{Waypoint} bodů by měl odpovídat jedné instanci \texttt{INode}, tedy vrcholu z~reprezentace grafu. V~platformě je tato reprezentace použita pro implementaci komponent \texttt{WorldWalker}, poskytující pohyb jednotek po mapě, a \texttt{MovingRangeTarget}, umožňující střelbu na pohyblivý cíl. V~případě, že tvůrce balíčku nepoužívá tyto dvě komponenty, může se interpretace cesty vracené jeho algoritmem lišit.


\subsection{Výběr algoritmu}
\label{sec:pathfindselection}
Instance algoritmu pro hledání nejkratší cesty je přístupná pomocí property \texttt{PathFinding} na instanci mapy aktuální úrovně. Získávání instance třídy implementující \texttt{IPathFindAlg} při načítání úrovně je implementováno podle návrhového vzoru \texttt{Abstract Factory} \citep[str.~87]{book:gangoffour}. Instance \textit{IPathFindAlgFactory} je získávána od pluginu logiky aktuální úrovně. Tato \textit{factory} je následně předána mapě, která jako poslední krok své inicializace vytvoří instanci algoritmu. Tento způsob jsme zvolili pro umožnění přístupu k~načtené mapě v~konstruktoru algoritmu.

\subsection{Implementace v~platformě}
Platforma poskytuje implementaci rozhraní \texttt{IPathFindAlg} pomocí algoritmu A*, popsaného v~části \ref{sec:astar}. 

Graf vytvářený tímto algoritmem obsahuje všechny možné hrany mezi sousedními dlaždicemi. Dále při připojení vrcholů budov předpokládá vytvoření všech hran, které by mohla kterákoli z~jednotek použít. Následně jsou tyto hrany filtrovány a~ohodnocovány v~průběhu výpočtu nejkratší cesty za použití potomka třídy \texttt{NodeDistCalculator}, definovaného v~balíčku. 

Třída \texttt{NodeDistCalculator} oproti obecnému rozhraní \texttt{INodeDistCalculator} navíc vyžaduje implementaci metody pro výpočet heuristiky. Hodnoty heuristiky závisí pouze na tvůrci balíčku, není tedy nijak kontrolována podmínka přípustnosti a~monotonie.

\section{Kamera}
\label{sec:camera}
Kamera je v~enginu UrhoSharp reprezentována komponentou \texttt{Camera}. V~naší platformě je pro tuto komponentu vytvořena separátní \texttt{Node}, která je následně přesouvána po herním světě. Ovládání pohybu této \texttt{Node} po herním světě implementuje platforma pomocí vlastní komponenty \texttt{CameraMover}, která je přidána na stejnou \texttt{Node} jako \texttt{Camera}.

Kamerou lze pohybovat třemi způsoby. Tyto způsoby jsou implementovány pomocí návrhového vzoru \texttt{State} \citep[str.~305]{book:gangoffour} a~jsou jimi:
\begin{enumerate}
	\item top-down kamera (state \texttt{FixedCamera}),
	\item kamera sledující jednotku (state \texttt{EntityFollowingCamera}),
	\item volně létající kamera (state \texttt{FreeFloatCamera}).

\end{enumerate}

Třída \texttt{CameraMover}, popisovaná v~této části, pouze poskytuje rozhraní pro pohyb kamerou zbytku platformy a~tvůrcům baličku. Vlastní řízení pohybu kamery na základě uživatelského vstupu je implementováno ve jmenném prostoru \texttt{Input} a~je popsáno v~části \ref{sec:input}. 

\subsection{Top-down kamera}
Pohyb top-down kamery je implementován třídou \texttt{FixedCamera}. Tento druh kamery sleduje bod v~herním světě v~pevné vzdálenosti od pozice kamery. Toto je implementováno pomocí hierarchie instancí \texttt{Node}, do které mezi vlastní \texttt{Node} obsahující kameru a \texttt{Node} reprezentující celou úroveň přidáváme třetí \texttt{Node}, kterou nazýváme \texttt{CameraHolder}, tedy úchyt kamery. 

Kamera je při přepnutí do top-down módu umístěna v~základní pozici vůči úchytu. Pro pohyb kamery po herním světě pak není pohybováno přímo kamerou, ale pouze úchytem.

V~naší implementaci úchyt při pohybu kopíruje terén mapy, což zjednodušuje hráči pohyb kamerou po mapě, kdy se nemusí starat o~oddalování a~přibližování kamery se změnami výšky terénu. 

Při rotaci je naopak měněna pozice kamery vůči úchytu, tedy je měněn atribut \texttt{Position} na instanci \texttt{Node} obsahující kameru. Rotace jsou ilustrovány na obrázku \ref{fig:rotation}. Při horizontálním otáčení se kamera otáčí okolo osy Y, tedy vertikální osy, procházející úchytem.  Při vertikálním otáčení je kamera také otáčena okolo úchytu, ale tentokrát okolo osy mířící vpravo z~pohledu kamery. Po každém otočení je směr pohledu upraven tak, aby znovu mířil na úchyt.

\begin{figure}[h]
	\centering
	\fontsize{8pt}{11pt}\selectfont
	\def\svgwidth{\textwidth}
	\input{img/Rotation.pdf_tex}
	\caption{Ukázka rotací s~pohledem ze směru osy rotace.}
	\label{fig:rotation}
\end{figure}

Dále lze kameru přibližovat a~oddalovat od úchytu ve směru pohledu. Tato akce je pouhou změnou velikosti vektoru \texttt{Position} kamery. Pro přiblížení kamery tedy vynásobíme velikost tohoto vektoru číslem menším než jedna, pro oddálení větším než jedna. Maximální přiblížení je limitováno pro zamezení pohledu skrz terén.

Pohyb \texttt{CameraHolderu} je omezen na herní plochu, čímž omezujeme možnost pohledu kamery mimo herní plochu. Dále je kontrolována vlastní pozice kamery, které není dovoleno dostat se pod úroveň terénu. Pokud by se v~aktuální pozici kamera dostala pod úroveň terénu, je její pozice dočasně posunuta nad terén.

\subsection{Kamera sledující jednotku}
Sledování jednotky kamerou je z~pohledu platformy pouhé sledování jiné \texttt{Node} než \texttt{CameraHolder}. Proto jsou stavy \texttt{FixedCamera} a \texttt{EntityFollowingCamera} potomkem jedné třídy, a~to \texttt{PointFollowingCamera}. 

Jedinou změnou oproti sledování \texttt{CameraHolderu} je odstínění kamery od otáčení jednotky. Oproti \texttt{CameraHolderu}, u~kterého zachováváme po celou dobu počáteční rotaci a~pouze s~ním pohybujeme po herním světě, mohou jednotky měnit jak svoji pozici, tak rotaci. Dále může mít jednotka nastaven jiný \texttt{Scale} na své \texttt{Node} než je \texttt{Scale} u~úchytu kamery. 

Pro řešení nechtěných rotací kamery s~jednotkou si ukládáme separátně požadovaný směr pohledu. Následně při každém výpočtu stavu hry otočíme kameru tak, aby mířila tímto směrem. Toto řešení nám umožní ignorovat rotace jednotky, kterou sledujeme, a~udržovat konstantní směr pohledu. 

Problém se \texttt{Scale} nastává při přepnutí sledování mezi dvěma různými jednotkami či jednotkou a \texttt{CameraHolderem}, a~dále při přibližování a~oddalování kamery. Řešením změny mezi dvěma různými sledovanými \texttt{Node} je vynásobení odsazení kamery poměrem jejich \texttt{Scale}. Při přibližování či oddalování stačí potom vydělit chtěnou změnu pozice hodnotou \texttt{Scale} sledované \texttt{Node}, čímž následně změníme pozici kamery v~herním světě nezávisle na \texttt{Scale} sledované \texttt{Node}.

Při sledování jednotky neumožňujeme hráči ovládat pohyb kamery po herním světě. Při pokusu o~pohyb dochází k~přepnutí zpět na sledování \texttt{CameraHolderu}, který je přemístěn na aktuální pozici jednotky.

\subsection{Volně létající kamera}
V~tomto módu se stává \texttt{Node} obsahující kameru přímým potomkem \texttt{LevelNode} a~je jí umožněn nezávislý pohyb po celé úrovni. Otáčení je v~tomto módu prováděno okolo vlastní pozice kamery, jak v~horizontálním, tak ve vertikálním směru. 

Stejně jako při předchozích způsobech pohybu je i~zde kamera omezena na herní plochu mapy. Není tedy možné mapu opustit, ani se dostat pod úroveň terénu.

\section{Vstup}
\label{sec:input}
Vstup je implementován separátně pro menu a~pro hru. Každá z~těchto částí je dále rozdělena na definici přenositelného rozhraní a~implementace pro myš a~klávesnici či dotykový display. Toto rozdělení můžeme vidět na diagramu \ref{fig:inputhier}. Jak bylo popsáno v~části \ref{sec:system_dif}, cílem naší práce je implementace pro systém Windows a~ovládání pomocí myši a~klávesnice. Proto je implementace tříd dotykového rozhraní pouhou kostrou pro budoucí rozšíření. 

Vzhledem k~úzkému provázání uživatelského rozhraní se vstupem, především v~menu, jsou třídy ovládající uživatelské rozhraní vytvářeny, spravovány a~uvolňovány třídami kontrolujícími vstup. Implementace uživatelského rozhraní je blíže popsána v~části \ref{sec:ui}. Spolu s~vlastnictvím uživatelského rozhraní slouží třídy vstupu také pro odstínění zbytku platformy od implementace grafického rozhraní. 

\begin{figure}[h]
	\centering
	\fontsize{7pt}{10pt}\selectfont
	\def\svgwidth{\textwidth}
	\input{img/InputHier.pdf_tex}
	\caption{Hierarchie tříd tvořících systém vstupu.}
	\label{fig:inputhier}
\end{figure}


V~menu je vstup ovládán třídou \texttt{MenuController}. Vzhledem k~tomu, že veškerý vstup uživatele je v~menu získáván pomocí grafického rozhraní, slouží třída \texttt{MenuController} především jako fasáda navržená podle návrhového vzoru \texttt{Facade} \citep[str.~185]{book:gangoffour}, překrývající složitost grafického rozhraní z~pohledu zbytku platformy. 

Při hře je uživatelský vstup zprostředkováván třídou \texttt{GameController}. Účelem této třídy je, stejně jako u \texttt{MenuController}, sloužit jako fasáda, tentokrát jak nad grafický uživatelským rozhraním, tak nad systémem herního enginu pro zpracovávání uživatelského vstupu. 

Vstup ve hře je přiřazen právě jednomu hráči. Tento hráč je určen hodnotou \texttt{Player} property třídy \texttt{GameController}. Ve zbytku platformy lze potom při zpracování vstupu měnit chování podle aktuálního hráče, kterému patří vstup. Příkladem takovéto změny chování může být přiřazení aktuálního hráče se vstupem jako vlastníka právě vytvořené budovy či jednotky. Další změny chování jsou popsány v~části \ref{sec:tools}.

Pro ovládání kamery je vytvořena separátní třída, která za pomoci třídy \texttt{GameController} a~grafického rozhraní přijímá vstup od uživatele a~pomocí třídy \texttt{CameraMover},  popsané v~části \ref{sec:camera}, převádí tento vstup na pohyb kamery.

Pro zjednodušení přepínání mezi schématy ovládání implementuje každá ze tříd přenositelné rozhraní, což můžeme vidět na diagram \ref{fig:inputhier}. Díky tomu lze vytváření instancí tříd implementovat pomocí návrhového vzoru \texttt{Abstract Factory} \citep[str.~87]{book:gangoffour}. Použitím této \texttt{Factory} vytváříme pouze jediné místo ve kterém se rozhodujeme, jaké schéma ovládání bude použito. Toto místo se nachází v~metodě \texttt{Start} třídy \texttt{MHUrhoApp}, rozhodování je tedy provedeno pouze při inicializaci platformy. Po vytvoření \texttt{Factory} jsou pak využívány pouze metody přenositelného rozhraní, případně ve třídách specifických pro dané ovládací schéma dochází k~přetypování zpět na konkrétní typy pro dané schéma ovládání.

\section{Editace mapy a~ovládání hry}
\label{sec:tools}
Editování a~ovládání úrovně je v~platformě spojeno do systému, který nazýváme \uv{Tools}. Tools, neboli nástroje, umožňují platformě i~tvůrcům pluginů definovat třídy, které přijímají vstup od uživatele a~převádí jej do modifikací mapy, vytváření jednotek či budov, ovládání jednotek či jinou manipulaci s~herním světem. Protože implementace jednotlivých nástrojů je závislá na použitém schématu ovládání, má systém nástrojů podobnou strukturu jako systém vstupu, popsaný v~části \ref{sec:input}. Struktura nástrojů obsahuje přenositelnou část, zde definovanou ve jmenném prostoru \texttt{MHUrho.EditorTools.Base}, a~z~této části odvozené implementace pro jednotlivá schémata ovládání. Implementace nástrojů pro klávesnici a myš je obsažena ve jmenném prostoru \texttt{MHUrho.EditorTools.MouseKeyboard}. Pro ovládání pomocí dotykové obrazovky je, stejně jako u~vstupu pomocí dotykové obrazovky, obsažena ve jmenném prostoru \texttt{MHUrho.EditorTools.Touch} pouze kostra implementace připravená pro budoucí rozšíření.

Platforma poskytuje několik nástrojů umožňujících editaci terénu. Těmito nástroji jsou:
\begin{itemize}
	\item \texttt{TileTypeTool}, umožňující změnu typu dlaždic;
	\item \texttt{TerrainManpulatorTool}, poskytující několik způsobů změny reliéfu.
\end{itemize} 

Dále pak platforma poskytuje nástroje pro vytváření jednotek a~budov v~podobě těchto nástrojů:
\begin{itemize}
	\item \texttt{BuildingBuilderTool} pro stavbu budov;
	\item \texttt{UnitSpawningTool} pro tvorbu jednotek.
\end{itemize} 
Tyto nástroje umožňují stavbu všech budov a~tvorbu všech jednotek přítomných v~balíčku. Předpokládáme ovšem, že při vlastní hře, případně i~při editaci, bude hráči omezena množina dostupných jednotek a~budov, případně bude při pokusu o~vytvoření kontrolováno a~měněno množství surovin vlastněné hráčem. Z~tohoto důvodu předpokládáme, že tvůrci balíčků tyto nástroje nahradí svojí vlastní implementací.

Posledním nástrojem je \texttt{UnitSelectorTool}, umožňující označení skupiny jednotek a~vydávání rozkazů této skupině. Tento nástroj implementuje jednoduché schéma ovládání postačující pro jednodušší hry, ale stejně jako u~předchozích nástrojů předpokládáme, že tvůrci balíčků tento nástroj vymění za svoji vlastní implementaci.

Pro výběr poskytovaných nástrojů používá platforma podobný systém jako u~výběru algoritmu pro hledání nejkratší cesty, popsaný v~části \ref{sec:pathfindselection}. Po pluginu logiky úrovně je požadována definice metody \texttt{GetToolManager}, vracející instanci tvůrcem pluginu definovaného potomka třídy \texttt{ToolManager}. Z~této instance je poté získán seznam  všech nástrojů dostupných hráči v~aktuální úrovni.

\section{Základní komponenty}
\label{sec:defaultcomponents}
\texttt{DefaultComponents}, neboli základní komponenty, jsou komponenty poskytované platformou, implementující funkcionalitu společnou velké části podporovaných typů her. Tato funkcionalita zahrnuje:

\begin{itemize}
	\item pohyb po terénu (\texttt{WorldWalker});
	\item označování jednotek a~vydávání rozkazů (\texttt{UnitSelector});
	\item střelbu na statický a~pohyblivý cíl (\texttt{Shooter}, \texttt{StaticRangeTarget},\\ \texttt{MovingRangeTarget});
	\item útok na blízko (\texttt{StaticMeeleAttacker}, \texttt{MovingMeeleAttacker});
	\item reakci na kliknutí (\texttt{Clicker});
	\item simulaci balistického projektilu (\texttt{BallisticProjectile}).
\end{itemize}

Schéma vztahů \texttt{DefaultComponent} k~ostatním částem platformy a~pluginů můžeme vidět na diagramu \ref{fig:defaultcomponentstructure}. Pro použití těchto komponent musí tvůrce pluginu při vytvoření instance entity, tedy jednotky, budovy či projektilu, vytvořit požadovanou základní komponentu a~připojit ji k~této instanci entity. Platforma sama základní komponenty entitám nikdy nepřiděluje. Implementace základních komponent této skutečnosti využívá, a~v~metodách vytvářející instance základních komponent požaduje referenci na instanční plugin, který navíc musí implementovat rozhraní \texttt{IUser} specifikované danou komponentou. Tímto způsobem je umožněna implementace základních komponent, která není závislá na konkrétních implementacích ostatních částí platformy, nijak neomezuje chování entit používajících tuto komponentu a~umožňuje tvůrci balíčku modifikovat chování této komponenty implementací požadovaných metod. 

\begin{figure}[h]
	\centering
	\fontsize{8pt}{11pt}\selectfont
	\def\svgwidth{\textwidth}
	\input{img/DefaultComponent.pdf_tex}
	\caption{Vztah \texttt{DefaultComponent} k~ostatním částem platformy.}
	\label{fig:defaultcomponentstructure}
\end{figure}


Příkladem může být třída \texttt{WorldWalker}, která po uživateli, tedy instančním pluginu, požaduje metodu vracející \texttt{INodeDistCalculator}, který je následně použit pro výpočet nejkratší cesty. Tímto způsobem je implementace \texttt{WorldWalker} oproštěna od závislosti na použitém algoritmu pro hledání nejkratší cesty. Druhým příkladem může být \texttt{MovingMeeleAttacker}, který po uživateli požaduje tři metody, a~to \texttt{IsInRange}, \texttt{PickTarget}, \texttt{MoveTo}. Metody \texttt{IsInRange} a \texttt{PickTarget} slouží pro specifikaci chování komponenty tvůrcem balíčku, tedy umožňují výběr cíle a~rozhodování, zda je cíl v~dosahu, podle vlastních kritérií. Metoda \texttt{MoveTo} naopak slouží pro izolaci komponenty od implementace pohybu jednotky. Jednotka tedy není omezena na pohyb pomocí komponenty \texttt{WorldWalker}, ale umožňuje tvůrci vytvořit vlastní implementací pohybu jednotky po mapě.

Další události, na které by mohl tvůrce pluginu chtít reagovat, jsou poskytovány jako \texttt{eventy}, jejichž obsluhu si může tvůrce pluginu zaregistrovat. Příkladem těchto událostí může být začátek pohybu, dokončení pohybu či zrušení pohybu u~komponenty \texttt{WorldWalker}.

Důležitou vlastností těchto komponent je jejich schopnost automatického ukládání a~načítání spolu s~entitou, ke které jsou přidány. Z~pohledu tvůrce pluginu tedy stačí přidat tyto komponenty k~entitě při jejím vytvoření. Jediným omezením jsou \texttt{eventy}, které si musí tvůrce pluginu zaregistrovat po načtení úrovně znovu, protože nelze jednoduše implementovat jejich automatické ukládání a~načítání bez nutnosti jejich implementace každým uživatelem komponenty. 

\section{Ukládání a~načítání úrovní}
\label{sec:loading}

\noindent{Načítání úrovní lze iniciovat třemi způsoby:}
\begin{enumerate}
	\item vytvoření nové úrovně,
	\item načtení úrovně existující v~balíčku,
	\item načtení uložené hrané úrovně.
\end{enumerate}

Při prvním způsobu je nová úroveň načtena do základní podoby, ve které je celá mapa umístěna v~rovině s~výškou nula a~všechny dlaždice mají základní typ definovaný v~balíčku. Tento způsob vždy načítá úroveň v~editačním módu.

Druhý způsob načítá úrovně do stavu, do kterého byli uvedeny při předchozí editaci. Takovéto úrovně označujeme jako \uv{Prototype} úrovně. Tyto úrovně lze načíst pro další editaci, nebo je možné je spustit pro hraní.

Třetí možností je vybrání hry, která byla uložena již v~průběhu hraní. Tyto úrovně již nelze nahrát pro editaci, což umožňuje zjednodušení logiky pluginů. 

Pro reprezentaci těchto různých stavů je podle návrhového vzoru \textit{State} \citep[str.~305]{book:gangoffour} implementována třída \texttt{LevelRep}. Všechny možné stavy a~přechody mezi nimi můžeme vidět na diagramu \ref{fig:levelrepstates}. 

Každá úroveň začíná ve stavu \texttt{NewlyCreated}. V~tomto stavu je úroveň při uživatelově manipulaci s~obrazovkou \texttt{LevelCreationScreen}. Na této obrazovce uživatel nastavuje parametry vytvářené úrovně, kterými jsou jméno úrovně, velikost mapy, ikona, plugin a~popis úrovně. Následně je úroveň vygenerována podle zadaných parametrů a~převedena do stavu editace. Vlastní editace úrovně je blíže popsána v~části \ref{sec:tools}. 

Z~editoru lze úroveň uložit jako \texttt{Prototyp}, čímž je zapsána do balíčku a~je následně možné ji z~grafického rozhraní spouštět v~herním módu. Dále lze úroveň uložit pod novým jménem, implementující standardní akci \texttt{SaveAs}, čímž přechází do \texttt{ClonedEditing} stavu. 

Ze stavu \texttt{Prototype} lze úroveň nahrát pro editaci, čímž lze upravit aktuální prototyp nebo pomocí \texttt{SaveAs} vytvořit odvozený. Dále lze prototyp nahrát pro hru. Při tomto nahrání hráč specifikuje počáteční parametry hry, jako přiřazení pluginů hráčům, rozdělení hráčů do týmů a~další, přidané pluginem logiky úrovně. Následně je úroveň nahrána a~přechází do stavu \texttt{Playing}. Poslední možnou akcí ze stavu \texttt{Prototype} je duplikace prototypu, umožňující nahrání prototypu pod novým jménem a~vytvoření nového prototypu. Tato funkcionalita do jisté míry duplikuje funkcionalitu \texttt{SaveAs}, přišlo nám ale vhodné umožnit hráči duplikaci úrovně již před načtením pro zamezení nechtěné změny prototypu.

Při hraní lze aktuální stav úrovně uložit do tzv.~Save file. Stav hry je uložen do adresáře specifikovaného platformou a~může následně být nahrán pro pokračování z~daného stavu hry. Stav hrané úrovně nelze uložit zpět do prototypu, při skončení hry je tedy aktuální stav ztracen a~úroveň přechází zpět do stavu \texttt{Prototype} se stavem úrovně před spuštěním hry.

Uložené hry lze nahrát do stavu \texttt{LoadedSaved}, kdy je z~uložené hry načtena část informace potřebná pro identifikaci balíčku, ze kterého uložená hra vznikla. Následně může být hra nahrána do plného stavu \texttt{Playing}, kde pokračuje z~uloženého stavu.

\begin{figure}[h]
	\centering
	\fontsize{8pt}{11pt}\selectfont
	\def\svgwidth{\textwidth}
	\input{img/LevelRepStates.pdf_tex}
	\caption{Stavy \texttt{LevelRep} a~přechody mezi nimi.}
	\label{fig:levelrepstates}
\end{figure}

\subsection{Stav úrovně}
Pro ukládání a~načítání aktuálního stavu úrovně používá platforma serializaci pomocí \textit{protocol buffers}, jak bylo popsáno v~části \ref{sec:savingformat}. 

Systém \textit{protocol buffers} obsahuje interface definition language (IDL), tedy jazyk pro popis ukládaných struktur nezávislý na cílovém programovacím jazyce, a~kompilátor souborů v~IDL do cílových programovacích jazyků. 

Platforma definuje čtyři \texttt{.proto} soubory, obsahující definici struktur a~kompilované do \texttt{.cs} souborů, které jsou používány zbytkem platformy. Těmito čtyřmi \texttt{.proto} soubory jsou:

\begin{enumerate}
	\item \texttt{UrhoTypes.proto}, definující serializaci typů enginu UrhoSharp;
	\item \texttt{MHUrhoTypes.proto}, definující serializaci potomků \texttt{DefaultComponent} a \texttt{Path};
	\item \texttt{PluginStorage.proto}, definující strukturu ukládání poskytovanou pluginům;
	\item \texttt{GameState.proto}, definující serializaci celkového stavu úrovně.
\end{enumerate}

Uvnitř těchto souborů je použit systém importování, kterým lze obsah jednoho \texttt{.proto} souboru importovat a~použít v~jiném souboru. 

Ze struktur definovaných v~těchto souborech potom \texttt{protoc} kompilátor vygeneruje zdrojové soubory v~jazyce C\#, které obsahují definici tříd odpovídajících těmto strukturám a~umožňující serializaci.

Serializované součásti platformy pak vyplní jim odpovídající třídu vygenerovanou z \texttt{.proto} souborů, kterou následně použijeme pro serializaci a~zapsání do souboru.

\subsection{Pluginy}
Pro uložení stavu pluginů jsme vytvořili datovou strukturu umožňující ukládat libovolnou posloupnost podporovaných typů. Podporovanými typy jsou jak základní typy jazyka C\#, tak typy enginu UrhoSharp a~pole těchto typů.

Ukládaná data lze ukládat a~načítat podle pořadí, indexovat čísly či pojmenovávat textovými jmény. Těmto třem způsobům odpovídají tyto třídy:
\begin{enumerate}
	\item \texttt{SequentialPluginData},
	\item \texttt{IndexedPluginData},
	\item \texttt{NamedPluginData}.
\end{enumerate}

Uložení pomocí \texttt{SequentialPluginData} produkuje nejmenší množství dat, protože nemusí ukládat mimo vlastní data a~identifikaci typu dat žádnou identifikaci navíc. \texttt{IndexedPluginData} musí navíc k~vlastnímu obsahu ukládat číselný index, \texttt{NamedPluginData} potom celý text jmen.

Při ukládání je spolu s~vlastní hodnotou dat ukládán také jejich typ. Toto je implementováno pomocí systému \texttt{oneof} protocol buffers, který umožňuje uložit jeden z~vyjmenovaných typů a~poté při načtení obsahuje hodnotu identifikující který z~typů byl uložen. Tento uložený typ je pak při načítání dat používán pro kontrolu, zda se tvůrce pluginu opravdu snaží číst typ, který byl uložen. Pro specifikaci čteného typu je vytvořeno generické rozhraní, které umožňuje číst libovolný typ a~provádí kontrolu typového parametru vůči uloženému typu. Tím je zajištěna typová bezpečnost tohoto systému za cenu zvětšení ukládaných dat.



\chapter{Tvorba balíčku}
\label{sec:packagemaking}
Každý balíček je tvořen jedním hlavním XML souborem spolu s dalšími soubory, představujícími assety hry. Těmito assety mohou být 3D modely, textury, pluginy či definice uživatelského rozhraní. Tato část dokumentace popíše pouze tvorbu hlavního XML souboru. Tvorbu 3D modelů, textur a definice uživatelského rozhraní není tématem této práce. Pro získání 3D modelů a textur použitelných v enginu UrhoSharp a naší platformě je možné postupovat podle tohoto tutoriálu od tvůrců enginu Urho3D \citep{site:blendertourho3D}. Pro tvorbu definic uživatelského rozhraní a prefabrikátů jednotek, budov a projektilů je dále možné použít editor distribuovaný spolu s enginem Urho3D \citep{site:urho3deditor}.

Při tvorbě balíčku je prvním krokem umístění všech assetů, tedy 3D modelů, textur, definic uživatelského rozhraní, assembly pluginů a dalších do adresářového podstromu kořenového adresáře balíčku. Konkrétní struktura není definována platformou, ale závisí čistě na tvůrci balíčku. Platforma pouze načítá relativní cesty z XML souboru definujícího balíček a používá adresář, ve kterém je tento XML soubor umístěn, jako kořen těchto relativních cest. 

V následujících částech popíšeme tvorbu XML souboru a význam jednotlivých elementů a atributů tohoto souboru.


\section{Struktura XML}
XML soubor představující balíček je popsán schématem \texttt{Data/Schemas/GamePack.xsd}. Spolu s tímto schématem obsahuje distribuce platformy šablonu XML souboru balíčku, umístěnou v adresáři instalace na cestě \texttt{Data/Templates/PackageDefinition.xml}. Příklad XML souboru funkčního balíčku můžeme vidět v rámci ukázkového balíčku. 

Kořenovým elementem XML souboru balíčku je element \texttt{gamePack}. Tento element musí specifikovat jmenný prostor \texttt{xmlns="http://www.MobileHold.cz/GamePack.xsd"} pro možnost validace balíčku a jeho nahrání do platformy. Dále zde  definujeme jméno balíčku, které bude zobrazováno uživateli při výběru balíčků. 

\lstset{
	language=XML
}
\begin{lstlisting}
<gamePack xmlns="http://www.MobileHold.cz/GamePack.xsd" name="[Package name]">
\end{lstlisting}

Prvním elementem uvnitř elementu \texttt{gamePack} je element \texttt{description}. Tento element může podle schématu mít pouze textový obsah, který se následně zobrazí uživateli při označení balíčku na obrazovce pro výběr balíčků.
\begin{lstlisting}
<description>[Description text]</description>
\end{lstlisting}

Dalším elementem je element \texttt{levels}, jehož jediný potomek \texttt{dataDirPath} definuje cestu k adresáři, ve kterém budou ukládány datové soubory uložených prototypů úrovní. Všechny cesty uvedené kdekoli v balíčku jsou brány jako relativní vůči adresáři, ve kterém je umístěn XML soubor definující balíček.
\begin{lstlisting}
<levels>
	<dataDirPath>[Path to directory where level data will be stored]</dataDirPath>
</levels>
\end{lstlisting}

Následuje sekvence elementů umožňujících definice typů herních prvků. Těmito elementy jsou:

\begin{enumerate}
	\item \texttt{levelLogicTypes} (typy logik úrovní),
	\item \texttt{playerAITypes} (typy logik hráčů),
	\item \texttt{resourceTypes} (typy surovin),
	\item \texttt{tileTypes} (typy dlaždic),
	\item \texttt{unitTypes} (typy jednotek),
	\item \texttt{projectileTypes} (typy projektilů),
	\item \texttt{buildingTypes} (typy budov).
\end{enumerate}
Obsah těchto prvků bude popsán v následujících částech. 

Jako poslední obsahuje element \texttt{gamePack} šestici elementů obsahujících cesty k texturám ikon. Těmito elementy jsou:
\begin{enumerate}
	\item \texttt{resourceIconTexturePath} (textura ikon surovin),
	\item \texttt{tileIconTexturePath} (textura ikon dlaždic),
	\item \texttt{unitIconTexturePath} (textura ikon jednotek),
	\item \texttt{buildingIconTexturePath} (textura ikon budov),
	\item \texttt{playerIconTexturePath} (textura ikon hráčů),
	\item \texttt{toolIconTexturePath} (textura ikon nástrojů),
\end{enumerate}
Tyto cesty jsou, stejně jako všechny cesty v celém XML souboru balíčku, relativní vůči adresáři, ve kterém je umístěn XML soubor definující balíček. Cesty v těchto elementech mohou odkazovat na různé soubory, jeden soubor nebo libovolnou množinu souborů textur. Při použití editačních nástrojů platformy je předpokládána podoba textury ikon zobrazená na obrázku \ref{fig:icontexture}. Textura se skládá ze čtyř čtverců o stejné velikosti, které jsou používány pro zobrazení v grafickém prvku \texttt{Checkbox}, tedy zaškrtávací box. Čtverce jsou zobrazeny v následující situaci:
\begin{enumerate}
	\item prvek není stlačen, kurzor není nad prvkem,
	\item prvek není stlačen, kurzor je nad prvkem,
	\item prvek je stlačen, kurzor není nad prvkem,
	\item prvek je stlačen, kurzor je nad prvkem.
\end{enumerate}

V obrázku \ref{fig:icontexture} můžeme vidět čtverce označené odpovídajícími čísly.

\begin{figure}[h]
	\centering
	\includegraphics{icontexture}
	\caption{Požadovaná podoba textury ikon.}
	\label{fig:icontexture}
\end{figure}

\section{Přidání jednotek}
\label{sec:addunit}
Přidání typu jednotky do balíčku znamená přidání elementu \texttt{unitType} jako potomka elementu \texttt{unitTypes}, uvedeného v předchozí části. Element \texttt{unitType} má tuto základní strukturu:

\begin{lstlisting}
<unitType name="[Name]" ID="[Number]">
	<assets type="[xmlprefab|binaryprefab|items]">[content dependent on type]</assets>
	<assemblyPath>[Path to assembly of the plugin]</assemblyPath>
	<extension>[User defined part of XML]</extension>
	<iconTextureRectangle left="[Number]" top="[Number]" right="[Number]" bottom="[Number]"/>
</unitType>
\end{lstlisting}

Atributy \texttt{name}, tedy jméno, a \texttt{ID}, tedy identifikátor, identifikují daný typ jednotky. Jméno i identifikátor musí být unikátní mezi typy jednotek, ale můžou být shodné se jménem a identifikátorem nějakého typu jiného druhu herního prvku. Jméno může být libovolný neprázdný textový řetězec, ID může být libovolné celé nenulové číslo.

Element \texttt{assets} definuje assety, tedy model, texturu či tvar pro výpočet kolizí, nahrávané při vytváření instance tohoto typu jednotky. Tyto assety mohou být definovány třemi způsoby, rozlišenými hodnotou atributu \texttt{type}:

\begin{enumerate}
	\item \texttt{xmlprefab},
	\item \texttt{binaryprefab},
	\item \texttt{items}
\end{enumerate}

XML a binary prefab jsou soubory vytvářené editorem Urho3D. Obsahem těchto elementů je jediný potomek \texttt{path}, v němž je vepsána cesta k souboru obsahujícímu prefabrikát. Tato cesta je, jako všechny cesty v balíčku, relativní vůči adresáři obsahujícímu soubor XML definující balíček.
\begin{lstlisting}
<path>[Path to prefab file]</path>
\end{lstlisting}

Pro způsob \textit{items} obsahuje element \texttt{assets} několik potomků, kupříkladu následující:
\begin{lstlisting}
<scale x="0.2" y="0.2" z="0.8"/>
<model type="static">
	<modelPath>Assets/Box.mdl</modelPath>
	<material>
		<simpleMaterialPath>Assets/StoneMaterial.xml</simpleMaterialPath>
	</material>
</model>
\end{lstlisting}
Element \texttt{scale}, neboli škálování, určuje nastavení hodnoty \texttt{Scale} vytvořené instance \texttt{Node} reprezentující jednotku. Tato hodnota škáluje pozice a rozměry potomků a komponent této \texttt{Node}, tedy například rozměr 3D modelu. 

Element \texttt{model} určuje 3D model použitý pro zobrazení této jednotky. Tento model je přidán jako komponenta na vytvořenou \texttt{Node}, čímž je střed modelu udržován na pozici \texttt{Node}. Atribute \texttt{type} určuje, zda je model statický (hodnota \textit{static}), či animovaný (hodnota \textit{animated}). Statický model je úspornější z pohledu výkonu, ale jak už název napovídá, nelze na něj použít animace. Oproti tomu animovaný model použití animací podporuje. Animace jsou ovládány v pluginech, proto popis využití animací naleznete v části \ref{sec:unitinstanceplugin}.

Součástí definice modelu je definice jeho textur. Textury jsou v enginu UrhoSharp reprezentovány materiálem, který definuje vlastní texturu, použitý \texttt{shader} a jeho parametry. Systém materiálů je popsán v dokumentaci enginu UrhoSharp a Urho3D, nebudeme proto v této práci tento systém rozebírat. Model může být složen z více geometrií, kde ke každé z nich je přiřazen separátní materiál. Přiřazení materiálů lze provést dvěma způsoby:

\begin{enumerate}
	\item souborem \texttt{MaterialList.txt},
	\item explicitním vyjmenováním materiálů.
\end{enumerate}

Následuje element \texttt{assemblyPath}, který obsahuje cestu k assembly obsahující plugin tohoto typu jednotek. Cesta je, jako všechny cesty v balíčku, relativní vůči adresáři obsahující XML soubor. Tento plugin je následně využit pro tvorbu instančních pluginů a definici chování jednotek ve hře. Blíže je význam a funkce pluginů popsán v části \ref{sec:pluginmaking}. Plugin typu je identifikován podle hodnoty \texttt{name} a \texttt{ID}, jejichž hodnoty se musí shodovat s hodnotami uvedenými zde v XML. 

Element \texttt{extension} slouží tvůrci pluginů pro uložení vlastních dat. Obsah tohoto elementu není platformou nijak omezen či validován, a je předáván pluginu typu při jeho načítání do platformy. V ukázkovém balíčku se tento element používá například pro specifikaci ceny jednotky, vymezení druhů dlaždic, kterými může jednotka procházet či specifikaci používaného projektilu.

Poslední element, \texttt{iconTextureRectangle}, vymezuje část textury specifikované elementem \texttt{unitIconTexture}, popsaným v předchozí části. Obdélník, vymezený hodnotou atributů tohoto elementu definuje pozici textury první situace, tedy případu, kdy je \texttt{Checkbox} nezmáčknutý a myš není umístěna nad ním. Pozice dalších stavů jsou potom předpokládány postupně s posunem o šířku obdélníku vpravo, jak můžeme vidět na obrázku \ref{fig:icontexture}.


\section{Přidání budov}
Přidání typu budov do balíčku je z velké části identické přidání typu jednotky, popsaném v předchozí části \ref{sec:addunit}. Z tohoto důvodu zde nebudeme rozebírat identické části a blíže popíšeme pouze části rozdílné.

Pro přidání typu budov přidáme element \texttt{buildingType} jako potomka elementu \texttt{buildingTypes}. Základní struktura obsahu elementu \texttt{buildingType} je následující:
\begin{lstlisting}
<buildingType name="[Name]" ID="[Number]">
	<assets type="[xmlprefab|binaryprefab|items]">[content dependent on type]</assets>
	<assemblyPath>[Path to assembly of the plugin]</assemblyPath>
	<extension>[User defined part of XML]</extension>
	<iconTextureRectangle left="[Number]" top="[Number]" right="[Number]" bottom="[Number]"/>
	<size x="[Number of tiles]" y="[Number of tiles]"/>
</buildingType>
\end{lstlisting}

Jak můžeme vidět, jediným rozdílem oproti elementu definujícímu typ jednotek je element \texttt{size}. Tento element definuje velikost obdélníku mapy zabíraného tímto typem budov. Rozměry obdélníku jsou uváděny v počtu dlaždic.

Stejně jako u jednotek i zde musí být hodnoty atributů \texttt{name} a \texttt{ID} unikátní mezi všemi typy budov. Oproti jednotkám \texttt{iconTextureRectangle} určuje pozici v textuře ikon budov, tedy v textuře určené elementem \texttt{buildingIconTexture}.


\section{Přidání projektilů}
Přidání projektilů je, stejně jako přidání budov, také z velké části identické přidání jednotek, popsaném v částí \ref{sec:addunit}. Přidání typu projektilů je provedeno přidáním elementu \texttt{projectileType} jako potomka elementu \texttt{projectileTypes}. Struktura obsahu elementu \texttt{projectileType} je následující:

\begin{lstlisting}
<projectileType name="[Name]" ID="[Number]">
	<assets type="[xmlprefab|binaryprefab|items]">[content dependent on type]</assets>
	<assemblyPath>[Path to assembly of the plugin]</assemblyPath>
	<extension>[User defined part of XML]</extension>
</projectileType>
\end{lstlisting}

Jak můžeme vidět, obsahuje \texttt{projectileType} elementy identické obsahu \texttt{unitType}, se stejným významem.


\section{Přidání dlaždic}
Typy dlaždic mají oproti jednotkám, budovám a projektilům jiný účel. Kde jednotky, budovy a projektily jsou přidávány jako nové herní prvky do herního světa, slouží typy dlaždic pouze pro dodání vzhledu prvkům již existujícím v úrovni, tedy dlaždicím mapy. Dále mohou být typy dlaždic využity při implementaci logiky pluginů pro rozlišení chování na různých typech dlaždic.

Typ dlaždice je reprezentován elementem \texttt{tileType}, který musí podle schématu XML být potomkem \texttt{tileTypes}. Obsah elementu \texttt{tileType} je následující:
\begin{lstlisting}
<tileType name="[Name]" ID="[Number]">
	<iconTextureRectangle left="[Number]" top="[Number]" right="[Number]" bottom="[Number]"/>
	<texturePath>[Path to texture file]</texturePath>
	<minimapColor R="[0-255]" G="[0-255]" B="[0-255]"/>
</tileType>
\end{lstlisting}

Atributy \texttt{name} a \texttt{id} mají stejnou sémantiku jako v předchozích částech, musí tedy být unikátní mezi všemi druhy dlaždic. Stejně tak \texttt{iconTextureRectangle} má stejný význam jako v předcházejících částech s tím rozdílem, že je textura specifikována elementem \texttt{tileIconTexturePath}.

Element \texttt{minimapColor} určuje barvu dlaždice při zobrazení na minimapě. Jak vidíme z ukázky XML, barva je specifikována pomocí červené, zelené a modré složky s hodnotami od 0 do 255.

Specifikem typů dlaždic je povinnost definice základního typu dlaždic, který je při generování nové úrovně určen jako počáteční typ všech dlaždic. Tento typ je specifikován elementem \texttt{defaultTileType}. Obsah tohoto elementu je identický s obsahem elementu \texttt{tileType}, jediným rozdílem je tedy jméno tohoto elementu, které platforma využívá pro jeho odlišení. Element \texttt{defaultTileType} musí být první definovaný typ dlaždice, a stejně jako ostatní \texttt{tileType} musí i jeho jméno a identifikátor být unikátní mezi všemi typy dlaždic.

\section{Přidání surovin}
Suroviny jsou v platformě používány pouze pro označení čísla, které určuje jejich množství. Z tohoto důvodu není potřeba u tohoto druhu herního prvku specifikovat velké množství informací.

Každý typ surovin je specifikován elementem \texttt{resourceType}, který je musí být potomkem elementu \texttt{resourceTypes}. Obsah elementu \texttt{resourceType} je následující:
\begin{lstlisting}
<resourceType name="[Name]" ID="[Number]">
	<iconTextureRectangle left="[Number]" top="[Number]" right="[Number]" bottom="[Number]"/>
</resourceType>
\end{lstlisting}

Stejně jako u všech předchozích i zde musí jméno a identifikátor být unikátní mezi typy surovin. Element \texttt{iconTextureRectangle} potom určuje část textury určené elementem \texttt{resourceIconTexturePath}, která reprezentuje ikonu suroviny. Tato ikona není v tuto chvíli žádnou částí platformy využívána, nejsou tedy na její vzhled kladeny žádná omezení.

\section{Přidání AI hráče}
Hlavní součástí umělé inteligence hráče je plugin, jehož vytváření popíšeme v následující části \ref{sec:pluginmaking}. Z pohledu balíčku je každý typ umělé inteligence hráče reprezentována elementem \texttt{playerAIType} v podstromu elementu \texttt{playerAITypes}. Element \texttt{playerAIType} má tuto strukturu:

\begin{lstlisting}
<playerAIType name="[Name]" ID="[Number]" category="[ai|human|neutral]">
	<iconTextureRectangle left="[Number]" top="[Number]" right="[Number]" bottom="[Number]"/>
	<assemblyPath>[Path to assembly of the plugin]</assemblyPath>
	<extension>[User defined part of XML]</extension>
</playerAIType>
\end{lstlisting}

Umělé inteligence hráčů se dělí do tří skupin:
\begin{enumerate}
	\item umělá inteligence oponentů (ai),
	\item umělá inteligence pomáhající hráčovy (human),
	\item umělá inteligence neutrálního hráče (neutral).
\end{enumerate}

Umělé inteligence oponentů řídí akce hráčů, kteří jsou součástí některého z týmů ale nejsou ovládání uživatelem. Umělá inteligence pomáhající hráčovy řídí stejného hráče, jakého řídí uživatel svým vstupem. Tato umělá inteligence může být použita pro implementaci herních událostí či automatizaci některých součástí hry. Neutrální hráč potom ovládá prvky mapy, které jsou součástí scenérie a neúčastní se přímo souboje hráčů. Těmito prvky mohou být zvířata pohybující se v úrovni, stromy či součásti terénu. 

\section{Přidání AI úrovně}
Definice typu umělé inteligence úrovně je reprezentována elementem \texttt{levelLogicType}, který je potomkem \texttt{levelLogicTypes}. Obsah elementu \texttt{levelLogicType} je následující:

\begin{lstlisting}
<levelLogicType name="[Name]" ID="[Number]">
	<assemblyPath>[Path to assembly of the plugin]</assemblyPath>
	<extension>[User defined part of XML]</extension>
</levelLogicType>
\end{lstlisting}
Jak můžeme vidět, obsahuje definice typu umělé inteligence úrovně elementy popsané v předcházejících částech se stejnou sémantikou.


\chapter{Tvorba pluginů}
Pluginy rozumíme .NET assembly nahrávané platformou MHUrho a implementující chování úrovní, hráčů, jednotek, budov a projektilů. Pluginy jsou v balíčku uloženy jako \texttt{.dll} knihovny, cílené pro .NET Framework 4.7.2. 

V této části dokumentace popíšeme postup vytvoření takovéto knihovny. Její následné připojení do balíčku je popsáno v předchozí části \ref{sec:packagemaking}.

Platforma nijak neomezuje rozdělení pluginů pro různé herní prvky do různých knihoven. Lze tedy všechny pluginy shromáždit v jedné, jak je to provedeno v ukázkové hře, nebo je možné vytvořit více různých knihoven s libovolným rozdělením pluginů.

Stejně jako v předchozí části i zde je velká část tvorby shodná pro různé druhy prvků. Celý proces proto popíšeme pouze v části tvorby pluginu jednotky a v následujících částech uvedeme pouze rozdíly oproti této části.


\section{Vytvoření pluginu jednotky}
\label{sec:unittypeplugin}
V této části popíšeme tvorbu typu jednotky. Tato jednotka bude mít tyto vlastnosti:

\begin{itemize}
	\item 3D model s animacemi,
	\item pohyb po terénu přes omezenou množinu typů dlaždic,
	\item střelba na dálku za použití ypu projektilu specifikovaného v XML.
\end{itemize}

Jako první vytvoříme definici jednotky v XML. Tento krok je popsán v předchozí části \ref{sec:packagemaking}, nebudeme ho zde proto opakovat. Výsledné XML typu jednotky tedy bude takovéto:
\begin{lstlisting}
<unitType name="Chicken" ID="3">
<assets type="xmlprefab">
<path>Assets/Units/Chicken/prefab.xml</path>
</assets>
<assemblyPath>ShowcasePackage.dll</assemblyPath>
<extension>
<cost>
<Wood>0.2</Wood>
<Gold>0.5</Gold>
</cost>
<canPass>
<Sand/>
<Xamarin/>
<Grass/>
<Water/>
</canPass>
</extension>
<iconTextureRectangle left="0" top="200" right="100" bottom="300"/>
</unitType>
\end{lstlisting}

Hotovou jednotku můžete vidět v ukázkové hře pod názvem \texttt{Chicken}.

\section{Plugin typu}
Prvním krokem je vytvoření dvou veřejných tříd, jedné reprezentující plugin typu a druhé reprezentující plugin instance. 

Jako první vytvoříme třídu reprezentující plugin. Tato třída musí být veřejná a dědit od třídy \texttt{UnitTypePlugin}, definované platformou. Takováto třída poté bude nalezena pomocí \texttt{Reflection} a za běhu platformy načtena jako plugin daného typu.  

\begin{lstlisting}
public class ChickenType : UnitTypePlugin {

}
\end{lstlisting}

Následuje implementace požadovaných vlastností a metod třídy. Jako první přidáme vlastnosti \texttt{Name} a \texttt{ID}.
\begin{lstlisting}
public override string Name => "Chicken";
public override int ID => 3;
\end{lstlisting}

Tyto vlastnosti jsou používány pro nalezení pluginu pro daný typ. Hodnoty \texttt{Name} a \texttt{ID} se tedy musí shodovat s hodnotami uvedenými v atributech \texttt{name} a \texttt{ID} v XML definici typu.

Dále platforma po všech typových pluginech požaduje implementaci tří hlavních metod:

\begin{enumerate}
	\item \texttt{Initialize},
	\item \texttt{GetInstanceForLoading},
	\item \texttt{CreateNewInstance}.
\end{enumerate}


První metodou společnou všem pluginům typů je metoda \texttt{Initialize}. Tato metoda je poskytnuta z důvodu použití \texttt{Reflection} pro vytváření instancí typových pluginů, což vynucuje použití konstruktoru bez parametrů. Metoda \texttt{Initialize} tedy do jisté míry nahrazuje konstruktor. Tato metoda je volána pouze jednou, při načtení balíčku do hry. Typická implementace načte data z \texttt{extension} elementu v XML definici typu. Tato data často budou odkazovat na jiné typy herních prvků, metoda \texttt{Initialize} tedy dostává referenci na balíček, kterou může použít pro získání referencí na ostatní typy z balíčku. Implementace pro naši jednotku bude vypadat takto: 

\begin{lstlisting}
protected override void Initialize(XElement extensionElement, GamePack package) {
XElement costElem = extensionElement.Element(package.PackageManager.GetQualifiedXName(CostElement));
cost = Cost.FromXml(costElem, package);

XElement canPass =
extensionElement.Element(package.PackageManager.GetQualifiedXName(PassableTileTypesElement));
PassableTileTypes = ViableTileTypes.FromXml(canPass, package);

myType = package.GetUnitType(ID);
ProjectileType = package.GetProjectileType("EggProjectile");
}
\end{lstlisting}

\texttt{Cost} a \texttt{ViableTileTypes} jsou pomocné třídy, které zpracují části XML a podle načtených dat získají reference na typy. Ukázku manuálního získání typu můžeme vidět při inicializaci \texttt{ProjectileType}, kde používáme balíček pro získání typu projektilu.


Druhou metodou je metoda \texttt{CreateNewInstance}, která je volána při vytvoření nové instance jednotky tohoto typu. Jejím účelem je získání pluginu pro nově vytvářenou instanci jednotky. Tvorba pluginu instance bude popsána v následující část \ref{sec:instanceplugincreation}, zde pouze uveďme že plugin instance bude představován třídou \texttt{Chicken}.
\begin{lstlisting}
public override UnitInstancePlugin CreateNewInstance(ILevelManager level, IUnit unit){
return Chicken.CreateNew(level, unit, this);
}
\end{lstlisting}
Metoda získává jako první argument instanci \texttt{ILevelManager}, který reprezentuje aktuální úroveň a slouží jako přístupový bod ke všem službám platformy. Druhým argumentem je potom \texttt{IUnit}, která je instancí reprezentující nově vytvářenou jednotku v platformě. Právě pro tuto jednotku vytváříme instanční plugin.


Poslední požadovanou metodou je metoda \texttt{GetInstanceForLoading}. Tato metoda má podobný účel jako předchozí metoda \texttt{CreateNewInstance}, a to získání instančního pluginu jednotky. Rozdíl je ale v tom, že tato jednotka není nově vytvářená v průběhu hry úrovně, ale právě dochází k načtení úrovně z úložky a tedy počáteční stav jednotky i pluginu bude načten z uloženého souboru. Z tohoto důvodu musí být instanční plugin inicializován takovým způsobem, aby mohl následně načíst uložený stav. Implementace této metody bude tedy vypadat takto:

\begin{lstlisting}
public override UnitInstancePlugin GetInstanceForLoading(ILevelManager level, IUnit unit) {
return Chicken.CreateForLoading(level, unit, this);
}
\end{lstlisting}

Typy jednotek mají jedinou metodu odlišnou od ostatních pluginů typů, a to:

\begin{lstlisting}
public override bool CanSpawnAt(ITile tile);
\end{lstlisting}

Tato metoda je volána před vytvořením nové instance jednotky a jejím účelem je zjistit, zda lze jednotku vytvořit na dlaždici \texttt{tile}. Typická implementace ověří, zda se na dané dlaždici vyskytuje budova, zda je dlaždice správného typu a případně zda se na dané dlaždici vyskytují další jednotky. Naše implementace využije pomocnou třídu \texttt{ViableTileTypes}, která slouží právě jako seznam správných typů dlaždic.

\begin{lstlisting}
public override bool CanSpawnAt(ITile tile) {
return PassableTileTypes.IsViable(tile) && 
(tile.Building == null);
}
\end{lstlisting}

Tímto jsme vytvořili funkční plugin typu, který nám umožní vytvářet nové instance jednotek tohoto typu, kontrolovat místa v mapě, na kterých jsou vytvářeny a načítat uložené instance jednotky do hry.

\section{Plugin instance}

Každá instance jednotky má přiřazenu jednu instanci instančního pluginu. Tato instance pluginu je získávána voláním jedné z funkcí \texttt{CreateNewInstance} a \texttt{GetInstanceForLoading}, popsaných v předchozí části \ref{sec:unittypeplugin}. Metoda \texttt{CreateNewInstance} je volána při tvorbě nové instance jednotky v běžící úrovni, metoda \texttt{GetInstanceForLoading} je pak volána při načítání jednotky v průběhu načítání uložené úrovně.

Plugin instance je tvořen třídou, která je potomkem \texttt{UnitInstancePlugin}. Pro naší jednotku tedy vytvoříme takovouto třídu:

\begin{lstlisting}
class Chicken : UnitInstancePlugin {

}
\end{lstlisting}

Tuto definici budeme ještě v průběhu tvorby upravovat, ale pro začátek takováto definice stačí.


\subsection{Vytvoření instance}

Pro implementaci metod \texttt{CreateNewInstance} a \texttt{GetInstanceForLoading} musíme poskytnout dva různé způsoby inicializace třídy. Toho lze docílit několika způsoby, my jsme si zvolili vytvoření dvou statických metod, které provedou inicializaci třídy.

Pro vytvoření nové nové instance v běžící úrovni vytvoříme metodu \texttt{CreateNew}. Volání této metody jsme mohli vidět v části \ref{sec:unittypeplugin} při implementaci \texttt{CreateNewInstance}. Pro získání instance při načítání pak vytvoříme metodu \texttt{CreateForLoading}, která je použita pro implementaci \texttt{GetInstanceForLoading}:

\begin{lstlisting}
public static Chicken CreateNew(ILevelManager level, IUnit unit, ChickenType type);

public static Chicken CreateForLoading(ILevelManager level, IUnit unit, ChickenType type);
\end{lstlisting}


\subsubsection{Pro načítání uloženého stavu}

Metoda \texttt{CreateForLoading} bude pouze jednoduché volání konstruktoru, tedy celá implementace bude vypadat následovně:

\begin{lstlisting}
public static Chicken CreateForLoading(ILevelManager level, IUnit unit, ChickenType type) {
return new Chicken(level, unit, type);
}
\end{lstlisting}

Konstruktor bude také jednoduchý, protože drtivou většinu dat budeme v tomto případě načítat z uloženého stavu. Implementace konstruktoru tedy bude takováto:

\begin{lstlisting}
public Chicken(ILevelManager level, 
			   IUnit unit, 
			   ChickenType type) 
:base(level,unit)
{
	this.myType = type;
	this.distCalc = new ChickenDistCalc(this);	
	unit.AlwaysVertical = true;
}
\end{lstlisting}

Položka \texttt{myType} ukládá referenci na \texttt{ChickenType}, který obsahuje data o schůdných typech dlaždic, používaném typu projektilu a další data načtená z XML, společná všem jednotkám tohoto typu.

Pro implementaci pohybu po mapě bude naše jednotka využívat algoritmus pro hledání cesty, který bude požadovat instanci kalkulátoru ohodnocující hrany grafu. Touto instancí je právě instance \texttt{ChickenDistCalc}, popsaná v následujících částech.

\texttt{AlwaysVertical} položka instance jednotky určuje, zda se jednotka při pohybu otočí čelem přímo do směru pohybu, nebo zda se otočí pouze okolo osy \texttt{Y} a tedy její hlava bude stále kolmo nad terénem.

\subsubsection{Vytvoření v běžící hře}
\label{sec:instantiation}

Implementace metody \texttt{CreateNew} specifikujeme, které součásti platformy budeme využívat. Tyto součásti, které platforma nazývá \texttt{DefaultComponent}, jsou schopny samostatného ukládání a načítání, stačí je tedy přidat na nově vytvářenou či existující jednotku. Vlastní implementace \texttt{CreateNew} bude začínat následovně: 

\begin{lstlisting}
public static Chicken CreateNew(
	ILevelManager level, 
	IUnit unit, 
	ChickenType type)
{
	Chicken newChicken = 
		new Chicken(level, unit, type);
	newChicken.health = 100;
/*...*/
\end{lstlisting}

Jak můžeme vidět, jako první vytvoříme instanci pluginu. Důvodem jsou \texttt{DefaultComponenty}, které ve při svém vytváření potřebují referenci na instanční plugin. Navíc na tento plugin mají další požadavky, které uvedeme později. Dále inicializujeme počet životů jednotky na 100. Při načítání jednotky je počet životů načten z uloženého stavu, není proto inicializován v předchozí metodě \texttt{CreateForLoading}.

Následuje vytvoření a nastavení \texttt{DefaultComponentů} spolu s uložením referencí na tyto komponenty pro budoucí ovládání:

\begin{lstlisting}
newChicken.Walker = WorldWalker.CreateNew(newChicken, level);
newChicken.Shooter = Shooter.CreateNew(newChicken, level,
type.ProjectileType, new Vector3(0, 0.7f, -0.7f), 20);
newChicken.Shooter.SearchForTarget = true;
newChicken.Shooter.TargetSearchDelay = 2;

MovingRangeTarget.CreateNew(newChicken, level, new Vector3(0, 0.5f, 0));

var selector = UnitSelector.CreateNew(newChicken, level);
\end{lstlisting}

Jak jsme specifikovali na počátku tohoto tutoriálu, naším cílem je jednotka, která je schopna pohybovat se po terénu, střílet po nepřátelských jednotkách, sama může být cílem nepřátelských jednotek a může být označena a ovládána hráčem. Tyto vlastnosti jsou popořadě implementovány komponentami \texttt{WorldWalker}, \texttt{Shooter}, \texttt{MovingRangeTarget} a \texttt{UnitSelector}. Jak můžeme vidět, instance komponent \texttt{WorldWalker} a \texttt{Shooter} si ukládáme pro budoucí ovládání v dalších metodách. Oproti tomu \texttt{MovingRangeTarget} neplánujeme měnit, proto pouze vytváříme jeho instanci na naší jednotce. Instanci \texttt{UnitSelctor} pak pouze nastavíme v této metodě a poté ji již nebudeme měnit.


Poslední požadovanou vlastností byl 3D animovaný model. Model samotný je k jednotce přidán platformou automaticky podle popisu v XML. Jedinou funkcí pluginu je následně spouštět a zastavovat animace modelu. K tomuto účelu poskytuje engine UrhoSharp komponentu \texttt{AnimationController}, kterou připojíme k naší jednotce. Také si uložíme referenci na tuto komponentu pro ovládání modelu v dále definovaných metodách.
\begin{lstlisting}
newChicken.animationController = 
unit.CreateComponent<AnimationController>();
\end{lstlisting}

Posledním krokem je registrace obsluh událostí. Platformou definované \texttt{DefaultComponenty} poskytují \texttt{eventy}, ke kterým si může plugin zaregistrovat obsluhu. V našem případě provedeme následující registraci:

\begin{lstlisting}
walker.MovementStarted += OnMovementStarted;
walker.MovementFinished += OnMovementFinished;
walker.MovementFailed += OnMovementFailed;
walker.MovementCanceled += OnMovementCanceled;

shooter.BeforeShotFired += BeforeShotFired;
shooter.TargetAutoAcquired += OnTargetAutoAcquired;
shooter.TargetDestroyed += OnTargetDestroyed;

selector.UnitSelected += OnUnitSelected;
\end{lstlisting}

Události komponenty \texttt{WorldWalker} nastávají v těchto situacích:
\begin{enumerate}
	\item \texttt{MovementStarted} event nastává při začátku pohybu;
	\item \texttt{MovementFailed} event nastává při úspěšném dokončení pohybu do cíle;
	\item \texttt{MovementFailed} event nastává při neúspěšném ukončení pohybu, například při zablokování cesty;
	\item \texttt{MovementCanceled} event nastává při explicitním přerušení pohybu.
\end{enumerate}

Události komponenty \texttt{Shooter} nastávají v těchto situacích:

\begin{enumerate}
	\item \texttt{BeforeShotFired} event nastává těsně před vystřelením projektilu;
	\item \texttt{TargetAutoAcquired} event nastává při nalezení cíle;
	\item \texttt{TargetDestroyed} event nastává při zničení cíle.
\end{enumerate}

Poslední událost \texttt{UnitSelected} nastává ve chvíli, kdy je jednotka označena hráčem.


Tímto je inicializace hotova. Nyní nám zbývá implementovat metody požadované předkem \texttt{UnitInstancePlugin}, požadované \texttt{DefaultComponenty} a obsluhující události.

\subsection{Metody pluginu}

Třída \texttt{UnitInstancePlugin} požaduje po svých potomcích implementaci sedmi metod. Těmito metodami jsou:
\begin{lstlisting}
public abstract void SaveState(PluginDataWrapper pluginData);
public abstract void LoadState(PluginDataWrapper pluginData);

public abstract void Dispose();

public abstract void TileHeightChanged(ITile tile);
public abstract void BuildingBuilt(IBuilding building, ITile tile);
public abstract void BuildingDestroyed(IBuilding building, ITile tile);
public abstract void OnHit(IEntity other, object userData);
\end{lstlisting}


\subsubsection{Ukládání a načítání}

Dvojice metod \texttt{SaveState} a \texttt{LoadState} umožňuje pluginu uložit aktuální stav v metodě \texttt{SaveState} a následně tento stav při načítání úrovně zpět načíst v metodě \texttt{LoadState}.

V naší jednotce je jedinou informací počet životů, které jednotce zbývají. Uložení této informace provedeme takto:
\begin{lstlisting}
public override void SaveState(
PluginDataWrapper pluginDataStorage)
{
var writer = 
pluginDataStorage.GetWriterForWrappedSequentialData();

writer.StoreNext(health);
}
\end{lstlisting}


Zpětné načtení jednotky bude potom vypadat následovně:

\begin{lstlisting}
public override void LoadState(PluginDataWrapper pluginData) {
Unit.AlwaysVertical = true;

Walker = Unit.GetDefaultComponent<WorldWalker>();
Shooter = Unit.GetDefaultComponent<Shooter>();
var selector = Unit.GetDefaultComponent<UnitSelector>();

RegisterEvents(Walker, Shooter, selector);

animationController =
Unit.CreateComponent<AnimationController>();

var reader = pluginData.GetReaderForWrappedSequentialData();
reader.GetNext(out health);
}
\end{lstlisting}

Jak můžeme vidět, při načítání jednotky je potřeba vykonat více práce než při jejím ukládání. První znovu nastavíme držení vertikální pozice jednotky. Tato vlastnost se během hry nemění, proto ji nepotřebujeme ukládat a pouze ji nastavíme při vytváření a načítání jednotky. 

Následuje získání referencí na \texttt{DefaultComponenty}. Jak bylo řečeno u vytváření nové instance, dokáží se tyto komponenty samy ukládat a načítat. Z tohoto důvodu nám zde stačí získat reference na již načtené komponenty. 

Dále zaregistrujeme obsluhy událostí stejně jako při načítání. Následuje vytvoření komponentu \texttt{AnimationController}. Jak bylo zmíněno v části o vytváření instance, není tato komponenta poskytována platformou MHUrho ale samotným enginem UrhoSharp, není tedy schopna sama se uložit. Plugin tedy musí jak při vytváření nové jednotky tak při načítání existující tuto komponentu vytvořit znovu. Jako poslední načteme uložený počet životů.

\subsubsection{Uvolnění zdrojů}

Uvolnění dat je prováděn pomocí metody \texttt{Dispose}. V naší implementaci pluginu nevytváříme žádné zdroje, které by bylo nutné uvolňovat pomocí \texttt{Dispose}, implementace této metody tedy bude prázdná.


\subsubsection{Události platformy}

Poslední čtyři metody požadované třídou \texttt{UnitInstancePlugin} jsou volány v těchto situacích:
\begin{itemize}
	\item \texttt{TileHeightChanged} při změně výšky dlaždice, na které se jednotka právě nachází,
	\item \texttt{BuildingBuilt} při stavbě budovy na dlaždici, na které se jednotka právě nachází
	\item \texttt{BuildingDestroyed} při zničení budovy na dlaždici, na které se jednotka právě nachází,
	\item \texttt{OnHit} pokud je jednotka zasažena projektilem či útokem jiné jednotky či budovy.
\end{itemize}


Pro implementaci \texttt{TileHeightChanged} je důležité vědět, že platforma sama udržuje všechny součásti nad úrovní terénu. Metoda tedy bude zavolána, ale při tomto volání bude vždy jednotka nad úrovní terénu. Naopak pokud je výška terénu snížena, platforma s jednotkou neprovádí žádné akce. Naše implementace \texttt{TileHeightChanged} tedy při změně výšky dlaždice přesune jednotku v ose \textit{Y}, tedy ve vertikální ose, na výšku terénu v daném bodě.


\begin{lstlisting}
public override void TileHeightChanged(ITile tile)
{
var newPosition = Unit.Position;
newPosition.Y = Level.Map
.GetHeightAt(newPosition.X,
newPosition.Z);
Unit.MoveTo(newPosition);
}
\end{lstlisting}

Property \texttt{Unit} a \texttt{Level} jsou poskytována předkem \texttt{UnitInstancePlugin}, kterému jsou předávány v konstruktoru, jak můžeme vidět v naší implementaci.


Metoda \texttt{BuildingBuilt} by neměla v naší ukázce nikdy být zavolána, protože vytvořené budovy nebudou dovolovat stavbu na dlaždicích obsahujících jednotky. Pro oznámení chyby při zavolání tedy bude metoda vyhazovat výjimku. Implementace tedy bude vypadat takto:

\begin{lstlisting}
public override void BuildingBuilt(IBuilding building, ITile tile)
{
throw new InvalidOperationException("Building building on top of units is not supported.");
}
\end{lstlisting}

Metoda \texttt{BuildingDestroyed} bude oproti předchozí metodě naší implementací využívána. Naše jednotka bude schopná chůze po budovách, při zničení budov bude tedy nutné jednotku přemístit zpět na úroveň terénu. Tuto funkcionalitu jsme již implementovali v metodě \texttt{TileHeightChanged}, implementace této metody bude tedy obdobná. Navíc při zničení budovy zastavíme pohyb jednotky. Implementace tedy bude vypadat takto:

\begin{lstlisting}
public override void BuildingDestroyed(IBuilding building, 
ITile tile)
{
var newPosition = Unit.Position;
newPosition.Y = Level.Map
.GetHeightAt(newPosition.X,
newPosition.Z);
Unit.MoveTo(newPosition);
Walker.Stop();
}
\end{lstlisting}

Poslední požadovaná metoda \texttt{OnHit} informuje jednotku o tom, že byla zasažena. Argumenty metody je \texttt{IEntity}, která nás zasáhla, tedy jednotka, budova či projektil, a \texttt{object}, který tato entita předala metodě \texttt{HitBy}, kterou zavolala na naší \texttt{Unit}. Tento \texttt{object} v naší hře představuje udělené poškození. Dále v naší hře nedovolujeme poškození přátelských jednotek. Implementace \texttt{OnHit} bude tedy následující:

\begin{lstlisting}
public override void OnHit(IEntity other, 
						   object userData)
{
	if (Unit.Player.IsFriend(other.Player)) {
		return;
	}

	int damage = (int)userData;
	health -= damage;

	if (health < 0) {
		animationController.PlayExclusive("Assets/Units/Chicken/Models/Dying.ani", 0, false);
		dying = true;
		Shooter.Enabled = false;
		Walker.Enabled = false;
	}
}
\end{lstlisting}

Pokud je udílející jednotka přátelská, k žádnému udělení poškození nedojde. Poté získáme udělené poškození z argumentu \texttt{userData} a odečteme ho od aktuálního počtu životů jednotky. Pokud počet životů klesne pod nulu, pak spustíme animaci umírání a zastavíme komponenty \texttt{Walker} a \texttt{Shooter}, čímž se jednotka přestane pohybovat a přestane střílet. Proměnná \texttt{dying} je použita v následující metodě pro změnu chování během sekvence umírání.

\subsubsection{OnUpdate}
Předek \texttt{UnitInstancePlugin} poskytuje k přetížení ještě jednu metodu, kterou je \texttt{OnUpdate}. Tato metoda je volána při každém výpočtu stavu hry a umožňuje periodicky kontrolovat aktuální stav a podle tohoto stavu provádět úkony.

Naše jednotka bude v této metodě provádět několik úkonu. Těmito úkony budou:
\begin{enumerate}
	\item odstranění jednotky z úrovně ve chvíli, kdy je dokončena animace umírání;
	\item střelbu na cíl explicitně zvolený hráčem;
	\item otočení proti aktuálnímu cíli střelby.
\end{enumerate}

Pro kontrolu, zda byla dokončena animace umírání, použijeme komponentu \texttt{AnimationController}, kterou jsme vytvořili při vytvoření či načtení pluginu a uložili si na ni referenci v proměnné \texttt{animationController}. Zároveň využijeme proměnnou \texttt{dying}, kterou jsme nastavili v metodě \texttt{OnHit}, pospané v předešlé části. Kód bude tedy vypadat následovně:

\begin{lstlisting}
if (dying && 
	animationController.IsAtEnd(
		"Assets/Units/Chicken/Models/Dying.ani")) 
{
	Unit.RemoveFromLevel();
	return;
}
\end{lstlisting}

Pro střelbu na cíl, který byl zvolený hráčem, potřebujeme mít uložen tento cíl. K tomuto účelu slouží proměnná třídy \texttt{explicitTarget}, nastavovaná v obsluze události popsané v části \ref{sec:eventhandlers}. V tuto chvíli pouze zkontrolujeme, zda máme nastaven cíl, tedy že daná proměnná nemá hodnotu \texttt{null}, a poté se pokusíme zaútočit.

Z důvodů šetření výpočetním výkonem není velká část kontrol a výpočtů prováděna při každém výpočtu stavu hry, tedy každém volání \texttt{OnUpdate}, ale mají přiřazený časovač, který s každým tímto voláním posouvají a ve chvíli, kdy časovač vyprší, provedou výpočet a časovač resetují.

Mezi tyto výpočty patří i střelba na cíl explicitně zvolený hráčem. Při této střelbě používáme komponenty \texttt{Shooter} a \texttt{WorldWalker}, kde první je používána pro kontrolu dostřelu a střelbu samotnou a druhá pro případ, kdy je cíl mimo dostřel a jednotka musí provést pohyb směrem k cíli.

Výsledná část kódu vypadá takto:
\begin{lstlisting}
if (explicitTarget != null) {
	shootTestTimer -= timeStep;
	if (shootTestTimer < 0) {
		shootTestTimer = timeBetweenTests;

		if (Shooter.CanShootAt(explicitTarget)) {
			Walker.Stop();
			Shooter.ShootAt(explicitTarget);
		}
		else {			
			Walker.GoTo(Level.Map
							 .PathFinding
							 .GetClosestNode(explicitTarget.CurrentPosition));
		}
	}
}
\end{lstlisting}

První řádek kontroluje, zda je nastaven explicitní cíl. Následující tři řádky implementují časovač. Poté se zeptáme komponenty \texttt{Shooter}, zda je cíl v dostřelu. Pokud cíl v dostřelu je, zastavíme pohyb jednotky a začneme střílet. Pokud ale není v dostřelu, pak se pokusíme o pohyb na pozici nejbližší cíli.

Poslední část zajišťuje při střelbě otočení jednotky směrem, kterým vypouští projektily. Je využita funkce jednotky \texttt{Unit.FaceTowards}, která otočí jednotku tak, aby mířila čelem k danému bodu v herním světě. Výsledný kód bude tedy vypadat následovně:

\begin{lstlisting}
if (Shooter.Target != null && 
	Walker.State != WorldWalkerState.Started) 
{
	var targetPos = Shooter.Target.CurrentPosition;
	Unit.FaceTowards(targetPos);
}
\end{lstlisting}

Kontroly před otočením jednotky zkoumají, zda aktuálně střílíme na cíl a nepohybujeme se po mapě.

\subsubsection{Metody požadované potomky DefaultComponent}
Při vytvoření instance \texttt{DefaultComponent} je požadováno předání instančního pluginu. Za využití generických metod jazyka C\# je navíc u některých komponent požadována implementace rozhraní \texttt{IUser} dané komponenty. Jak jsme zmínili na počátku tvorby instančního pluginu, definice třídy \texttt{Chicken} bude ještě změněna. Tato změna nastane právě zde. Pro použití komponentů \texttt{WorldWalker}, \texttt{MovingRangeTarget} a \texttt{UnitSelector} musí naše třída implementovat rozhraní \texttt{WorldWalker.IUser}, \texttt{MovingRangeTarget.IUser} a \texttt{UnitSelector.IUser}. Jak můžeme vidět, komponent \texttt{Shooter} nepožaduje implementaci žádného rozhraní, místo toho všechny metody poskytuje jako události. Konečná definice třídy bude tedy vypadat následovně:

\begin{lstlisting}
public class Chicken : UnitInstancePlugin, 
					   WorldWalker.IUser, 
					   MovingRangeTarget.IUser,
					   UnitSelector.IUser
\end{lstlisting}

Tato rozhraní vyžadují implementaci následujících metod:
\begin{lstlisting}
INodeDistCalculator GetNodeDistCalculator();

IEnumerable<Waypoint> GetFutureWaypoints(MovingRangeTarget target);

bool ExecuteOrder(Order order);
\end{lstlisting}

Metoda \texttt{GetNodeDistCalculator} vrací instanci třídy implementující rozhraní \texttt{INodeDistCalculator}. Toto rozhraní požaduje implementaci jediné metody, a tou je \texttt{GetTime}.

\begin{lstlisting}
bool GetTime(INode source, INode target, MovementType movementType, out float time);
\end{lstlisting}

Tato metoda dostává dva vrcholy z grafu algoritmu pro hledání nejkratší cesty v podobě \texttt{source}, tedy zdroje a \texttt{target}, tedy cíle. Tato dvojice představuje hranu v tomto grafu. Cílem metody je rozhodnout, zda tato hrana je průchozí pro danou jednotku, a v případě že ano, nastavit výstupní argument \texttt{time} na čas potřebný k průchodu touto hranou. Platforma definuje dva typy hran. Těmito typy jsou:
\begin{enumerate}
	\item \texttt{Linear}, tedy hrany přes které je pohyb spojitý,
	\item \texttt{Teleport}, tedy hrany, u kterých je jednotka teleportována ze zdroje na cíl po čase rovném hodnotě \texttt{time}.
\end{enumerate}

Podle návratových hodnot této funkce pak algoritmus provádí výpočet nejkratší cesty. 

Naše implementace využije naší volby používaného algoritmu, a využije třídu \texttt{NodeDistCalculator}, která tuto jednu metodu pomocí návrhového vzoru \textit{Visitor} rozděluje na více metod, kde každá dostává pouze určitý typ vrcholů. Příkladem takovýchto metod můžou být tyto metody:

\begin{lstlisting}
bool GetTime(ITileNode source, ITileNode target, MovementType movementType, out float time);
bool GetTime(ITileNode source, IBuildingNode target, MovementType movementType, out float time);
\*...*\
bool GetTime(IBuildingNode source, ITileNode target, MovementType movementType, out float time);
\end{lstlisting}

Celkem tedy \texttt{ChickenDistCalc}, který je potomkem \texttt{NodeDistCalculator}, bude implementovat 9 metod specifikujících možné přechody mezi všemi kombinacemi všech tří druhů vrcholů. Rozhodovat se bude podle načtených prostupných typů dlaždic pro \texttt{ITileNode} či podle typů budov pro \texttt{IBuildingNode}. Příkladem může být implementace výpočtu hrany mezi dvěma vrcholy \texttt{ITileNode}, tedy mezi dvěma dlaždicemi:

\begin{lstlisting}
bool GetTime(/*...*/) {
	if (chicken.typePlugin
		       .PassableTileTypes
		       .Contains(target.Tile.Type) &&
		target.Tile.Building == null) {
		time = Vector3.Distance(source.Position, target.Position) / speed;
		return true;
	}
	time = -1;
	return false;
}
\end{lstlisting}

Tato implementace ověří, zda má cílová dlaždice typ, kterým dokáže jednotka procházet, a že se na cílové dlaždici nevyskytuje žádná budova. Pokud jsou tyto podmínky splněny, vypočítá čas jako poměr vzdálenosti zdroje a cíle a vrátí \texttt{true} pro indikaci, že je hrana průchodná. Pokud nejsou podmínky splněny, vrací tato implementace \texttt{false} pro oznámení neprůchodnosti hrany.


Implementace zbývajících metod \texttt{GetFutureWaypoints} a \texttt{ExecuteOrder} je mnohem jednodušší. Metoda \texttt{GetFutureWaypoints} využije komponentu \texttt{WorldWalker}, která poskytuje právě tuto službu. Implementace tedy bude vypadat následovně:

\begin{lstlisting}
public IEnumerable<Waypoint> GetFutureWaypoints(MovingRangeTarget movingRangeTarget)
{
	return Walker.GetRestOfThePath(new Vector3(0, 0.5f, 0));
}
\end{lstlisting}

Protože cesty jsou vypočítávány v úrovni terénu, musíme specifikovat, jak vysoko nad danou pozicí má jednotka svůj střed, nebo jiný cíl, na který mají střelci mířit. Toto je specifikováno předávaným vektorem, kde v tomto případě říkáme, že mají střelci mířit půl jednotky rozměrů nad úroveň terénu.

Metoda \texttt{ExecuteOrder} podle přijatého rozkazu nastaví komponenty \texttt{Shooter} a \texttt{WorldWalker}. Potomek třídy \texttt{Order}, reprezentující odkaz, je předán této metodě jako argument. Podle typu rozkazu potom jednotka vykoná určitou akci. Dále \texttt{Order} obsahuje proměnnou \texttt{Executed}, která určuje, zda byl rozkaz vykonán, a je používána pro řízení šíření rozkazů a jako návratová hodnota metod vydávajících rozkazy. 

Naše implementace bude reagovat na rozkazy pro pohyb, reprezentované typem \texttt{MoveOrder}, a na rozkazy na útok, reprezentované typem \texttt{AttackOrder}. Pří rozkazu na pohyb je zastavena střelba, přerušeno vyhledávání cílů a spuštěn komponent \texttt{WorldWalker}. Příkaz pohybu bude zpracováván tímto způsobem:
\begin{lstlisting}
Shooter.StopShooting();
Shooter.SearchForTarget = false;
explicitTarget = null;
order.Executed = Walker.GoTo(moveOrder.Target);
\end{lstlisting}

Při zpracování rozkazu útoku je jako první zastaven útok na aktuální cíl. Dále ověříme, zda nový cíl je opravdu nepřátelská entita a že na tento cíl lze střílet. Pokud jsou tyto podmínky splněny, pokusíme se na tento cíl začít střílet. Pokud toto není možné, znamená to, že je cíl z dostřelu. V tuto chvíli tedy spustíme pohyb k pozici cíle. Implementace takto popsané akce bude zapsána takto:
\begin{lstlisting}
Shooter.StopShooting();

if (Unit.Player.IsEnemy(attackOrder.Target.Player) && 
	(SetExplicitTarget(attackOrder.Target) != null))
{
	if (Shooter.ShootAt(explicitTarget)) {
		order.Executed = true;
	}
	else {
		var targetNode = \*get target node*\ 
		order.Executed = Walker.GoTo(targetNode);
	} 
}

if (order.Executed)
{
	Shooter.SearchForTarget = false;
}
\end{lstlisting}

\subsubsection{Implementace obsluh událostí}
\label{sec:eventhandlers}

Jak jsme popsali v části o vytváření nové instance \ref{sec:instantiation}, provádí se při každé inicializaci instančního pluginu registrace obsluh událostí. V části \ref{sec:instantiation} byly také popsány situace, při kterých jsou tyto obsluhy volány. V této části popíšeme implementace těchto obsluh. Pro připomenutí zde uvádíme kód registrace obsluh událostí:
\begin{lstlisting}
walker.MovementStarted += OnMovementStarted;
walker.MovementFinished += OnMovementFinished;
walker.MovementFailed += OnMovementFailed;
walker.MovementCanceled += OnMovementCanceled;

shooter.BeforeShotFired += BeforeShotFired;
shooter.TargetAutoAcquired += OnTargetAutoAcquired;
shooter.TargetDestroyed += OnTargetDestroyed;

selector.UnitSelected += OnUnitSelected;
\end{lstlisting}

Jak můžeme vidět, jsou tyto události rozděleny do skupin podle komponenty, na které nastávají. První komponentou je \texttt{WorldWalker}. Implementace obsluh událostí této komponenty bude manipulovat s animacemi modelu a spouštět či zastavovat střelbu na cíl.

Implementace \texttt{OnMovementStarted} spustí animaci chůze a nastaví její rychlosti. Dále zastaví jakoukoli střelbu a vypne hledání cíle. Odpovídající kód je potom:
\begin{lstlisting}
void OnMovementStarted(WorldWalker walker)
{
	animationController.PlayExclusive("Assets/Units/Chicken/Models/Walk.ani", 0, true);
	animationController.SetSpeed("Assets/Units/Chicken/Models/Walk.ani", 2);

	Shooter.StopShooting();
	Shooter.SearchForTarget = false;
}
\end{lstlisting}

Implementace \texttt{OnMovementFinished}, \texttt{OnMovementFailed} a \texttt{OnMovementCanceled} bude pro všechny tři tyto metody identická. Tyto události zastaví přehrávání animace a spustí vyhledávání cíle pro střelbu.
\begin{lstlisting}
void OnMovementXXX(WorldWalker walker)
{
	animationController.Stop("Assets/Units/Chicken/Models/Walk.ani");
	Shooter.SearchForTarget = true;
}
\end{lstlisting}

Obsluhy událostí komponenty \texttt{Shooter} budou mít následující implementace. \texttt{BeforeShotFired} otočí jednotku tak, aby byla postavena čelem k cíli. Tento kód jsme viděli již při implementaci \texttt{OnUpdate} metody. Z určitého pohledu se tato duplikace může zdát přebytečná, ale ze sémantiky události dává toto otočení smysl. Výsledný kód bude:
\begin{lstlisting}
void BeforeShotFired(Shooter shooter) {
	var targetPos = shooter.Target.CurrentPosition;
	Unit.FaceTowards(targetPos);
}
\end{lstlisting}

Obsluha metody \texttt{OnTargetAutoAcquired} zastaví pohyb jednotky a také otočí jednotku čelem k cíli. Výsledný kód bude tedy:
\begin{lstlisting}
void OnTargetAutoAcquired(Shooter shooter) {
	Walker.Stop();
	/* turn towards target*/
}
\end{lstlisting}

Obsluha poslední metody \texttt{OnTargetDestroyed} vynuluje případný explicitní cíl nastavený rozkazem hráče. Kód tedy bude:
\begin{lstlisting}
void OnTargetDestroyed(Shooter shooter, IRangeTarget target)
{
	explicitTarget = null;	
}
\end{lstlisting}

\section{Tvorba pluginů ostatních druhů}
Jak jsme napsali na začátku této kapitoly, popíšeme v této části pouze odlišnosti oproti tvorbě pluginu jednotky. Tyto odlišnosti budou především v požadovaných metodách a poskytovaných komponentách.  

\subsection{Pluginy budov}
Typové pluginy budov sdílí s typovými pluginy jednotek metody \texttt{Initialize}, \texttt{CreateNewInstance} a \texttt{GetInstanceForLoading}. Oproti typovým pluginům jednotek vracejí metody pro tvorbu instančních pluginů potomky typu \texttt{BuildingInstancePlugin}. Obdobou metody \texttt{CanSpawnAt} pluginů typů jednotek je zde metoda \texttt{CanBuild}, která má mírně odlišnou signaturu:

\begin{lstlisting}
public abstract bool CanBuild(IntVector2 topLeftTileIndex, IPlayer owner, ILevelManager level);
\end{lstlisting}

Tato metoda, stejně jako metoda \texttt{CanSpawnAt} pro jednotky, je používána pro zjištění, zda je možné budovu vytvořit budovu na určené pozici v herním světě. Na rozdíl od jednotek ovšem budovy mohou zabírat více než jednu dlaždici, proto je zde udáván pouze levý horní roh, tedy roh s nejmenší souřadnicí X a Z, ze kterého lze pak podle velikosti budovy získat množinu dlaždic zabraných budovou na této pozici. 


Z pohledu pluginů instancí jsou odlišnosti viditelnější. Platforma omezuje počet budov na dlaždici na nejvýše jednu budovu, neexistuje tedy obdoba metod pluginu instancí jednotek \texttt{BuildingBuilt} a \texttt{BuildingDestroyed} oznamujících vytvoření a zničení budovy. Metody \texttt{TeleHeightChanged} a \texttt{OnHit} zůstávají beze změny. Instanční plugin budov naopak přidává metody \texttt{CanChangeTileHeight}, \texttt{GetHeightAt} a \texttt{GetFormationController} s těmito signaturami:

\begin{lstlisting}
public abstract bool CanChangeTileHeight(int x, int y);

public virtual float? GetHeightAt(float x, float y)
{
	return null;
}

public virtual IFormationController GetFormationController(Vector3 centerPosition)
{
	return null;
}
\end{lstlisting}

Metoda \texttt{CanChangeTileHeight} je volána při pokusu o změnu výšky dlaždice, na které stojí daná budova. Podle návratové hodnoty je pak změna vykonána nebo zamítnuta. Metoda \texttt{GetHeightAt} slouží pro získání výšky pro chůzi jednotek po budově. Tato metoda je volána v rámci \texttt{Map.GetHeightAt}. Při návratu hodnoty \texttt{null} je pak brána výška terénu. Metoda \texttt{GetFormationController} určuje rozmístění jednotek po budově při rozkazu na přesun jednotek na tuto budovu. Jednotky by měli být rozmístěny okolo bodu v parametru. Při návratu hodnoty \texttt{null} je budova brána jako neschůdná.


Dále jsou některé \texttt{DefaultComponent} omezeny na použití pouze pro jednotky či projektily. Na budovách lze tedy použít pouze tyto komponenty:
\begin{enumerate}
	\item \texttt{StaticRangeTarget},
	\item \texttt{StaticMeeleAttacker},
	\item \texttt{Shooter},
	\item \texttt{Clicker}.
\end{enumerate}

\subsection{Pluginy projektilů}
Typové pluginy projektilů sdílí, stejně jako typové pluginy budov, s typovými pluginy jednotek metody \texttt{Initialize}, \texttt{CreateNewInstance} a \texttt{GetInstanceForLoading} s odpovídající změnou návratového typu. Oproti budovám a jednotkám lze projektily vytvářet kdekoli v herním světě, neexistuje tedy obdoba metody \texttt{CanSpawnAt}. Oproti tomu definuje typový plugin projektilů metodu \texttt{IsInRange} s následující signaturou:

\begin{lstlisting}
public abstract bool IsInRange(Vector3 source, IRangeTarget target);
\end{lstlisting}

Návratová hodnota této metody značí, zda při výstřelu projektilu z pozice \texttt{source} v herním světě lze zasáhnout cíl \texttt{target}.




Umělá inteligence úrovně je určena pro kontrolu globálního stavu hry, jako například omezení na možnosti reliéfu mapy, omezení stavby budov v určitých částech mapy či změny vlastností některých jednotek. Rozdělení umělé inteligence mezi úroveň, hráče a entity je možné mnoha způsoby. Platforma žádný z těchto způsobů explicitně nepodporuje ani nezakazuje.


\chapter{Aplikace a~ukázková hra}
V~této části popíšeme instalaci aplikace, použití grafického rozhraní a~uvedeme návod pro hraní ukázkové hry. 

\section{Instalace}
Pro instalaci platformy spusťte soubor \texttt{setup.exe}, umístěný v~přílohách práce \ref{sec:appendix}. Po spuštění instalátoru budete vyzváni k~výběru umístění aplikace. Platforma vyžaduje instalaci do adresáře, ve kterém má každý uživatel spouštějící platformu možnost zápisu. Následně bude aplikace nainstalována spolu se všemi svými závislostmi. V~aktuální verzi je jedinou závislostí platformy Microsoft .NET Framework 4.7.2 (x86 and x64).

\section{Grafické rozhraní}
Grafické uživatelské rozhraní platformy je tvořeno několika obrazovkami, které můžeme vidět na diagramu \ref{fig:screen_structure2}. Přechody mezi obrazovkami jsou iniciovány za použití grafických prvků těchto obrazovek, či jsou řízeny průběhem akce na pozadí. V~této části popíšeme použití složitějších obrazovek uživatelského rozhraní.

\begin{figure}[h]
	\centering
	\includegraphics[width=\textwidth]{img/ScreenStructure.png}
	\caption{Obrazovky menu a~přechody mezi nimi.}
	\label{fig:screen_structure2}
\end{figure}

\subsection{Výběr balíčků}
Obrazovka pro výběr balíčků, na diagramu \ref{fig:screen_structure2} označena \texttt{Package picking}, je z~hlavního menu přístupná stisknutím tlačítka \texttt{Start}. Obrazovku můžeme vidět na obrázku \ref{fig:packagepicking}. Centrální část obrazovky zabírá seznam balíčků dostupných v~aktuální instalaci platformy. Každý z~balíčků je reprezentován jednou položkou tohoto seznamu. 

\begin{figure}[h]
	\centering
	\includegraphics[width=0.5\textwidth]{img/PackagePickingScreen.png}
	\caption{Obrazovky pro výběr balíčku.}
	\label{fig:packagepicking}
\end{figure}

Pro přidání nového balíčků do nainstalované instance platformy přemístěte adresář obsahující soubory balíčku do adresáře \texttt{\%AppData\%/MHUrho/Packages}. Následně stiskněte tlačítko \texttt{Add}, neboli \uv{Přidat}, na obrazovce pro vybírání balíčků. Toto tlačítko vás přesune na obrazovku pro procházení souborového systému. Na této obrazovce poté vyberte XML soubor definující přidávaný balíček. Při návratu na obrazovku pro vybírání balíčků by měl být přidaný balíček viditelný v~seznamu dostupných balíčků. d

Pro smazání balíčku označte položku balíčku v~seznamu a~stiskněte tlačítko \texttt{Remove}. Pro načtení balíčku  označte položku balíčku a~stisknute tlačítko \texttt{Select}. Pro návrat na obrazovku hlavního menu použijte tlačítko \texttt{Back}.

\subsection{Výběr úrovně}
Po vybrání balíčku budete přesunuti na obrazovku výběru úrovně, v~diagramu označenou jako \texttt{Level picking}. Tuto obrazovku můžete vidět na obrázku \ref{fig:levelpicking}. Tato obrazovka poskytuje následující funkce:

\begin{enumerate}
	\item vytváření a~editaci nových úrovní,
	\item editaci existujících úrovní,
	\item spuštění existujících úrovní,
	\item mazání úrovní v~balíčku.
\end{enumerate}

Pro vytvoření nové úrovně označte položku \texttt{Create new level} a~následně stiskněte tlačítko \texttt{Edit}.

\begin{figure}[h]
	\centering
	\includegraphics[width=0.5\textwidth]{img/LevelPickingScreen.png}
	\caption{Obrazovky pro výběr úrovně.}
	\label{fig:levelpicking}
\end{figure}

Pro akci s~existující úrovní označte tuto úroveň v~seznamu. Následně stisknutím tlačítka \texttt{Delete} smažete úroveň, stisknutím tlačítka \texttt{Edit} přejdete na obrazovku \texttt{Level creation} a~následně na editaci úrovně a~stisknutím tlačítka \texttt{Play} přejdete na obrazovku \texttt{Level settings} pro nastavení parametrů spuštění úrovně.

Pro návrat na obrazovku výběru balíčků použijte tlačítko \texttt{Back}.
\subsection{Vytváření úrovně}
Při vytváření úrovně či editaci existující úrovně budete přesunuti na obrazovku označenou v~diagramu \ref{fig:screen_structure2} jako \texttt{Level creation}. Vzhled této obrazovky můžete vidět na obrázku \ref{fig:levelcreation}. Jak můžete vidět, tato obrazovka umožňuje nastavit tyto vlastnosti úrovně:

\begin{enumerate}
	\item jméno,
	\item velikost mapy,
	\item plugin logiky,
	\item ikonu,
	\item popis.
\end{enumerate}

Při editaci existující úrovně nelze měnit její velikost a~plugin logiky.

\begin{figure}[h]
	\centering
	\includegraphics[width=0.5\textwidth]{img/LevelCreationScreen.png}
	\caption{Obrazovky pro nastavení vlastností vytvářené úrovně.}
	\label{fig:levelcreation}
\end{figure}

Pro návrat na obrazovku výběru úrovní použijte tlačítko \texttt{Back}.

\subsection{Nastavení úrovně}
Pro nastavení vlastností úrovně před samotným spuštěním slouží obrazovka \texttt{Level settings}, zobrazená na obrázku \ref{fig:levelsettings}. Obrazovka je rozdělena do čtyř částí:

\begin{enumerate}
	\item výběr pluginu hráčů úrovně a~nastavení jejich příslušenství do týmu;
	\item zobrazení ikony úrovně;
	\item zobrazení prvků grafického uživatelského rozhraní definovaných pluginem úrovně, používaných pro nastavení parametrů spouštěné úrovně;
	\item zobrazení popisu úrovně.
\end{enumerate}

Stisknutím tlačítka \texttt{Play} je spuštěno načítání úrovně a~následně hra. Stisknutím tlačítka \texttt{Back} dojde k~přesunu zpět na obrazovku vybírání úrovní.

\begin{figure}[h]
	\centering
	\includegraphics[width=0.5\textwidth]{img/LevelSettingsScreen.png}
	\caption{Obrazovky pro nastavení spuštěné hry.}
	\label{fig:levelsettings}
\end{figure}

\subsection{Přerušení hry}
Při pozastavení hry je zobrazeno tzv.~\texttt{PauseMenu}. Položky tohoto menu se liší podle toho, zda úroveň editujeme či úroveň hrajeme. Porovnání těchto dvou menu můžeme vidět na obrázku \ref{fig:pauseMenu}.

Při editaci umožňuje menu uložit aktuální stav úrovně do balíčkupod jménem nastaveným při vytváření úrovně pomocí tlačítka \texttt{Save} či pod novým jménem pomocí tlačítka \texttt{SaveAs}.

Při hraní hry umožňuje menu uložit aktuální stav hrané hry do adresáře platformy pomocí tlačítka \texttt{Save}, či načíst hranou hru uloženou v~tomto adresáři pomocí tlačítka \texttt{Load}.

\begin{figure}[h]
	\centering
	\includegraphics[width=0.5\textwidth]{img/PauseMenuComparison.png}
	\caption{Menu pozastavení hry.}
	\label{fig:pauseMenu}
\end{figure}


\subsection{Výběr souboru}
Obrazovku pro výběr souboru můžeme vidět na obrázku \ref{fig:filepicking}. Modře označená část výpisu obsahu adresáře představuje podadresáře, bílá část pak soubory. V~horní části můžeme vidět vyhledávací řádek, do kterého je možno napsat část jména pro filtrování zobrazených souborů či přímo celou cestu vybíraného souboru či adresáře. Pro výběr lze použít tlačítko \texttt{Select}, které vybere soubor či adresář podle cesty ve vyhledávacím řádku, či dvojklikem na záznam vybíraného souboru.

Obrazovky pro ukládání a~načítání hraných úrovní z~adresáře platformy poskytují navíc tlačítko \texttt{Delete}, kterým je možné uloženou hru smazat.

\begin{figure}[h]
	\centering
	\includegraphics[width=0.5\textwidth]{img/FilePickingScreen.png}
	\caption{Obrazovka pro výběr souboru.}
	\label{fig:filepicking}
\end{figure}

\section{Ukázková hra}
Ukázková hra je reprezentována balíčkem \texttt{ShowcasePackage}. Tento balíček obsahuje jednotky, budovy, projektily a~další součásti hry, které je možné využít pro tvorbu úrovní. Současně obsahuje několik již vytvořených úrovní demonstrujících schopnosti protihráčů, jednotek a~budov.

\subsection{Okno aplikace při hře}
\label{sec:appwindow}
Okno aplikace má při hře platformou definované základní uživatelské rozhraní. Toto rozhraní můžeme vidět na obrázku \ref{fig:UI}. 

Minimapa poskytuje hráči přehled o~stavu velké části mapy bez nutnosti pohybu kamerou. Minimapu lze přibližovat či oddalovat za použití kolečka myši při umístění kurzoru nad minimapu. Dále lze kliknutím přesunout kameru na odpovídající pozici v~herním světě. V~neposlední řadě lze minimapu použít k~rychlému přesunu kamery pomocí kliknutí a~držení levého tlačítka a~posunu myší. Pozice kliknutí se stává středem pomyslného joysticku, který ovládáme posunem myši odpovídajícím směrem.

Centrální lišta obsahuje tlačítka určená aktuálně zvoleným nástrojem. Při velkém počtu tlačítek je možno touto lištou posouvat za použití tlačítek označených šipkami na pravé a~levé straně lišty.

Nad lištou vidíme vpravo tlačítko pro výběr aktuálního hráče a~vlevo tlačítko pro výběr nástroje. Výběr hráče je možný pouze při editaci úrovně. Při hraní je toto tlačítko neaktivní a~pouze zobrazuje ikonu hráče reprezentujícího uživatele. Tlačítko pro výběr budovy, stejně jako tlačítko pro výběr hráče při editaci, při kliknutí zobrazí vysouvací lištu, označenou fialově. Tato lišta obsahuje seznam dostupných nástrojů či seznam dostupných hráčů.

Poslední součástí v~levém dolním rohu je \texttt{CustomWindow}. Obsah této části je určen aktuálním nástrojem. Nejčastěji je v~této části uváděn název nástroje či součásti nástroje, nastavní velikosti štětce, cena budov a~další.

Balíček může do uživatelského rozhraní dodat vlastní prvky, jak můžeme vidět ve vrchní části obrazovky, ve které balíček ukázkové hry umisťuje lištu pro zobrazení množství surovin vlastněných hráčem.

\begin{figure}[h]
	\centering
	\includegraphics[width=0.5\textwidth]{img/GameUI.png}
	\caption{Uživatelské rozhraní ukázkové hry.}
	\label{fig:UI}
\end{figure}

\subsection{Editor}
Každý z~balíčků, tedy i~ukázková hra, může použít nástroje poskytované platformou či definovat své vlastní nástroje pro editaci úrovně. V~této části popíšeme použití nástrojů dostupných v~editoru úrovní ukázkové hry.

\subsubsection{Nástroje}

Výběr nástroje lze provést stisknutím tlačítka označeného na obrázku \ref{fig:UI} jako \textit{Výběr nástroje}. Stisknutím tohoto tlačítka je zobrazena lišta se seznamem všech dostupných nástrojů, kterou můžete vidět na obrázku \ref{fig:toolselection}. V této liště můžeme vidět označený aktuálně vybraný nástroj, jehož vzhled přebírá také tlačítko pro výběr nástroje. 

Stisknutím tlačítka jiného nástroje než aktuálně vybraného je tento nástroj aktivován a je schována lišta pro výběr nástrojů.

Následuje seznam nástrojů dostupných v ukázkové hře. U každého z nástrojů je uvedeno, ve kterých módech je možné ho využít. Bližší popis jednotlivých nástrojů můžete najít následujících částech.

\medskip
\noindent{
	\begin{minipage}{0.15\textwidth}
		\includegraphics{TerrainManipulatorTool}
	\end{minipage} \hfill
	\begin{minipage}{0.8\textwidth}
		\textbf{Nástroj pro změnu výšky terénu} umožňuje změnu výšky jednotlivých rohů dlaždic, celých dlaždic uvnitř čtverce či vyhlazení rozdílů ve výšce dlaždic uvnitř čtverce. Tento nástroj je dostupný pouze v editačním módu.
	\end{minipage}	
}

\medskip
\noindent{
	\begin{minipage}{0.15\textwidth}
		\includegraphics{TileTypeTool}
	\end{minipage} \hfill
	\begin{minipage}{0.8\textwidth}
		\textbf{Nástroj pro změnu typů dlaždic} umožňuje změnu typu dlaždic. Tento nástroj je dostupný pouze v editačním módu.
	\end{minipage}
}

\medskip
\noindent{
	\begin{minipage}{0.15\textwidth}
		\includegraphics{UnitSelectorTool}
	\end{minipage} \hfill
	\begin{minipage}{0.8\textwidth}
		\textbf{Nástroj pro výběr a ovládání jednotek} umožňuje označit vybrat skupinu jednotek a následně této skupině vydávat rozkazy pro pohyb či pro útok. Tento nástroje je dostupný v editačním i herním módu.
	\end{minipage}
}

\medskip
\noindent{
	\begin{minipage}{0.15\textwidth}
		\includegraphics{UnitSpawningTool}
	\end{minipage} \hfill
	\begin{minipage}{0.8\textwidth}
		\textbf{Nástroj pro vytváření jednotek} umožňuje přidat do úrovně jednotky vlastněné aktuálně ovládaným hráčem. Tento nástroje je dostupný pouze v editačním módu.
	\end{minipage}
}

\medskip
\noindent{
	\begin{minipage}{0.15\textwidth}
		\includegraphics{BuildingBuilderTool}
	\end{minipage} \hfill
	\begin{minipage}{0.8\textwidth}
		\textbf{Nástroj pro stavbu budov} umožňuje přidat do úrovně budovy vlastněné aktuálně ovládaným hráčem. Tento nástroje je dostupný v editačním i herním módu.
	\end{minipage}
}

\subsubsection{Nástroj pro změnu výšky terénu}
Tento nástroj umožňuje měnit reliéf terénu aktuální úrovně. Nástroj je složen ze čtyř součástí. Těmito součástmi jsou:

\medskip
\noindent{
	\begin{minipage}{0.15\textwidth}
		\includegraphics{VertexSelection}
	\end{minipage} \hfill
	\begin{minipage}{0.8\textwidth}
		Výběr rohů dlaždic.
	\end{minipage}
}

\medskip
\noindent{
	\begin{minipage}{0.15\textwidth}
		\includegraphics{VertexMovement}
	\end{minipage} \hfill
	\begin{minipage}{0.8\textwidth}
		Změny výšky vybraných rohů dlaždic.
	\end{minipage}
}

\medskip
\noindent{
	\begin{minipage}{0.15\textwidth}
		\includegraphics{TileMovement}
	\end{minipage} \hfill
	\begin{minipage}{0.8\textwidth}
		Změna výšky dlaždic uvnitř zvýrazněného čtverce.
	\end{minipage}
}

\medskip
\noindent{
	\begin{minipage}{0.15\textwidth}
		\includegraphics{Smoothing}
	\end{minipage} \hfill
	\begin{minipage}{0.8\textwidth}
		Vyhlazen rozdílů výšky dlaždic uvnitř zvýrazněného čtverce.
	\end{minipage}
}
\bigskip


První dvě součásti jsou dohromady používány pro výběr jednotlivých rohů dlaždic v mapě a následnou úpravu jejich výšky. Výběr lze provést zvolením části pro výběr rohů a následným kliknutím na dlaždici obsahující daný roh. Je vybrán roh nejbližší pozici kliknutí. Zrušení výběru lze následně provést pokusem o výběr již vybraného rohu. Pro změnu výšky vybraných rohů zvolíme část pro změnu výšky vybraných rohů. Po jejím zvolení můžeme kliknutím a držením levého tlačítka a posouváním myši nahoru a dolu měnit výšku vybraných rohů.

Součást pro změnu výšky dlaždic umožňuje pohybem kurzoru po herním světě vybrat čtverec dlaždic okolo pozice kurzoru, jejichž výšku měníme pomocí stisknutí a držení levého tlačítka a pohybem myši stejně jako u předchozí části.

Poslední součást pro vyhlazení terénu umožňuje stisknutím a držením levého tlačítka myši a pohybem po mapě vyrovnávat rozdíly výšky terénu.

\subsubsection{Nástroj pro změnu typu dlaždic}
Při zvolení tohoto nástroje je centrální lišta vyplněna všemi dostupnými typy dlaždic v balíčku. V ukázkovém balíčku bude centrální lišta vyplněna těmito ikonami:

\medskip
\noindent{
	\begin{minipage}{0.1\textwidth}
		\includegraphics[scale=0.3]{SandIcon}
	\end{minipage} \hfill
	\begin{minipage}{0.85\textwidth}
		Ikona typu dlaždic \texttt{Sand}, neboli písek.
	\end{minipage}
}

\medskip
\noindent{
	\begin{minipage}{0.1\textwidth}
		\includegraphics[scale=0.3]{XamarinIcon}
	\end{minipage} \hfill
	\begin{minipage}{0.85\textwidth}
		Ikona typu dlaždic \texttt{Xamarin}.
	\end{minipage}
}

\medskip
\noindent{
	\begin{minipage}{0.1\textwidth}
		\includegraphics[scale=0.3]{GrassIcon}
	\end{minipage} \hfill
	\begin{minipage}{0.85\textwidth}
		Ikona typu dlaždic \texttt{Grass}, neboli tráva.
	\end{minipage}
}

\medskip
\noindent{
	\begin{minipage}{0.1\textwidth}
		\includegraphics[scale=0.3]{WaterIcon}
	\end{minipage} \hfill
	\begin{minipage}{0.85\textwidth}
		Ikona typu dlaždic \texttt{Water}, neboli voda.
	\end{minipage}
}

\medskip
\noindent{
	\begin{minipage}{0.1\textwidth}
		\includegraphics[scale=0.3]{StoneIcon}
	\end{minipage} \hfill
	\begin{minipage}{0.85\textwidth}
		Ikona typu dlaždic \texttt{Stone}, neboli kámen.
	\end{minipage}
}

\bigskip

Vybráním typu dlaždic z tohoto seznamu, umístěním kurzoru do herního světa a stisknutím levého tlačítka změníme typ dlaždic zvýrazněného čtverce na zvolený typ.

Pokud podržíme levé tlačítko a budeme pohybovat myší, budou změněny všechny dlaždice kterých se dotkně zvýrazněná část.

Velikost zvýrazněné části lze nastavit v okně nástrojů pomocí grafického prvku přidaného tímto nástrojem.

\subsubsection{Nástroj pro výběr a ovládání jednotek}
Tento nástroj umožňuje vybrat skupinu jednotek aktuálního hráče. Tento výběr uskutečníme stisknutím a držením levého tlačítka a následným tažením myši po herním světě. Mezi počáteční pozicí stisku a aktuální pozicí kurzoru se vytvoří obdélník označující oblast, ve které budou jednotky vybrány. Po uvolnění tlačítka myši budou všechny jednotky vlastněné aktuálním hráčem přidány do aktuálně vybrané skupiny jednotek. Jednotky lze označovat i jednotlivě kliknutím na jednotku vlastněnou aktuálním hráčem.

Následně lze vybraným jednotkám vydávat rozkazy. Kliknutím levým tlačítkem vydáme rozkaz k pohybu na aktuální pozici kurzoru. Kliknutím pravím tlačítkem na nepřátelskou jednotku či budovu pak vydáme rozkaz k útoku na tuto jednotku či budovu.

\subsubsection{Nástroj pro vytváření jednotek}
Tento nástroj umožňuje přidávat do aktuální úrovně nové jednotky či odebírat existující. Po zvolení tohoto nástroje je centrální lišta naplněna ikonami všech dostupných jednotek, které je možné manuálně přidávat do úrovně. Nejvíce vpravo je pak přidána ikona umožňující mazání existujících jednotek. V ukázkovém balíčku tedy bude centrální lišta vyplněna následujícími ikonami:

\medskip
\noindent{
	\begin{minipage}{0.1\textwidth}
		\includegraphics[scale=0.3]{ChickenIcon}
	\end{minipage} \hfill
	\begin{minipage}{0.85\textwidth}
		Ikona typu jednotek \texttt{Chicken}.
	\end{minipage}
}

\medskip
\noindent{
	\begin{minipage}{0.1\textwidth}
		\includegraphics[scale=0.3]{WolfIcon}
	\end{minipage} \hfill
	\begin{minipage}{0.85\textwidth}
		Ikona typu jednotek \texttt{Wolf}.
	\end{minipage}
}

\medskip
\noindent{
	\begin{minipage}{0.1\textwidth}
		\includegraphics[scale=0.3]{Deleter}
	\end{minipage} \hfill
	\begin{minipage}{0.85\textwidth}
		Ikona pro odstraňování jednotek.
	\end{minipage}
}


\bigskip


Pro vytvoření nové jednotky vybereme jeden z poskytovaných typů a následným kliknutím do herního světa vytvoříme na pozici kurzoru novou jednotku vybraného typu.

Pro odstranění jednotky vybereme ikonu pro odstraňování jednotek. Následným kliknutím na jednotku tuto jednotku odstraníme z úrovně.

Funkcionalita tohoto nástroje je při hraní úrovně nahrazena budovou tvrze, popsanou v~části \ref{sec:buildings}.


\subsubsection{Nástroj pro stavbu budov}
\label{sec:buildingbuilder}
Tento nástroj umožňuje v editačním i v herním módu přidávat do herního světa budovy vlastněné aktuálně ovládaným hráčem. Při zvolení tohoto nástroje je centrální lišta vyplněna ikonami všech budov dostupných v balíčku. V ukázkovém balíčku bude centrální lišta vyplněna těmito ikonami:

\medskip
\noindent{
	\begin{minipage}{0.1\textwidth}
		\includegraphics[scale=0.3]{KeepIcon}
	\end{minipage} \hfill
	\begin{minipage}{0.85\textwidth}
		Ikona typu budov \texttt{Keep}.
	\end{minipage}
}

\medskip
\noindent{
	\begin{minipage}{0.1\textwidth}
		\includegraphics[scale=0.3]{GateIcon}
	\end{minipage} \hfill
	\begin{minipage}{0.85\textwidth}
		Ikona typu budov \texttt{Gate}.
	\end{minipage}
}

\medskip
\noindent{
	\begin{minipage}{0.1\textwidth}
		\includegraphics[scale=0.3]{TowerIcon}
	\end{minipage} \hfill
	\begin{minipage}{0.85\textwidth}
		Ikona typu budov \texttt{Tower}.
	\end{minipage}
}

\medskip
\noindent{
	\begin{minipage}{0.1\textwidth}
		\includegraphics[scale=0.3]{WallIcon}
	\end{minipage} \hfill
	\begin{minipage}{0.85\textwidth}
		Ikona typu budov \texttt{Wall}.
	\end{minipage}
}

\medskip
\noindent{
	\begin{minipage}{0.1\textwidth}
		\includegraphics[scale=0.3]{TreeCutterIcon}
	\end{minipage} \hfill
	\begin{minipage}{0.85\textwidth}
		Ikona typu budov \texttt{TreeCutter}.
	\end{minipage}
}

\medskip
\noindent{
	\begin{minipage}{0.1\textwidth}
		\includegraphics[scale=0.3]{TreeIcon}
	\end{minipage} \hfill
	\begin{minipage}{0.85\textwidth}
		Ikona typu budov \texttt{Tree}.
	\end{minipage}
}

\medskip
\noindent{
	\begin{minipage}{0.1\textwidth}
		\includegraphics[scale=0.3]{Deleter}
	\end{minipage} \hfill
	\begin{minipage}{0.85\textwidth}
		Ikona pro odstraňování budov.
	\end{minipage}
}

\bigskip

Následně vybráním jedné z těchto budov a kliknutím na dlaždici je do herního světa umístěna nová budova se středem na dané dlaždici. Tento nástroj dále umožňuje ovládání a mazání existujících budov. 

Ovládání je umožněno při aktivaci nástroje bez vybrané budovy, což následně umožňuje kliknutím na budovu zobrazit rozhraní pro její ovládání.

Mazání je umožněno výběrem ikony červeného čtverce a následným kliknutím na existující budovu.

\subsection{Ovládání kamery}
Kamera je schopna pohybu v~několika módech. Těmito módy jsou RTS mód, \texttt{FreeFloat} mód a~sledování jednotky. Přepínání mezi RTS a \texttt{FreeFloat} módem je prováděno klávesou \textit{Shift}. Přepnutí na sledování jednotky je možné kliknutím pravého tlačítka na jednotku. Následná lze přejít zpět na RTS mód pokusem o~pohnutí kamerou, či na \texttt{FreeFloat} mód pomocí klávesy \textit{Shift}.

Pohyb v~módech RTS a \texttt{FreeFloat} lze ovládat pomocí klávesnice. Klávesy \textit{W}, \textit{S}, \textit{A}, \textit{D} umožňují pohyb kamery vpřed, vzad, vlevo a~vpravo. 

V~módu RTS lze také ovládat pohyb kamery pomocí umístění myší na okraj obrazovky, načež se kamera začne posouvat směrem k~tomuto okraji. 

Otáčení kamery lze v~RTS módu a~módu sledování jednotky provést klávesami \textit{Q} a \texttt{E} pro otáčení vlevo či vpravo, a~klávesami \textit{R} a \textit{F} pro otáčení vzhůru a~dolu. 

V~módu \texttt{FreeFloat} lze otáčet kamerou pouze pomocí myši.   

\subsection{Budovy}
\label{sec:buildings}
Ukázkový balíček obsahuje šest typů budov. První čtyři typy označujeme jako obranné budovy, sloužící především pro zamezení průchodu nepřátelských jednotek a~umožňující zvýšení dostřelu vlastních jednotek pomocí umístění jednotek na tyto budovy. Mezi tyto budovy patří \texttt{Keep}, neboli tvrz, \texttt{Gate}, neboli brána, \texttt{Tower}, neboli věž a \texttt{Wall}, neboli hradba. Zbylé dva typy slouží pro získávání dřeva. Těmito budovami jsou \texttt{Tree}, neboli strom, a \texttt{TreeCutter}, neboli dřevorubec. 

\medskip
\noindent{
	\begin{minipage}{0.15\textwidth}
		\includegraphics[scale=0.6]{Keep}
	\end{minipage} \hfill
	\begin{minipage}{0.8\textwidth}
		\textbf{Keep}, neboli tvrz. Každý hráč vlastní právě jednu tuto budovu, při jejímž zničení hráč prohrává. Cílem hry je tedy zničit protivníkovu tvrz bez ztráty své vlastní tvrze. Tvrz umožňuje pohyb jednotek po své střeše. Tvrz dále slouží pro tvorbu jednotek během hry. Tato funkce je zpřístupněna pomocí nástroje pro stavbu budov, popsaného v části \ref{sec:buildingbuilder}.
	\end{minipage}
}

\medskip
\noindent{
	\begin{minipage}{0.15\textwidth}
		\includegraphics[scale=0.6]{Gate}
	\end{minipage} \hfill
	\begin{minipage}{0.8\textwidth}
		\textbf{Gate}, neboli brána. Brána umožňuje pohyb jednotek jak po své střeše, tak skrz tunel uvnitř této budovy. Střecha této budovy je automaticky propojena se střechami bran, věží a hradeb na přímo sousedících dlaždicích. Dále umožňuje již podle svého názvu zavřít jeden konec tunelu a~tím znemožnit přístup do tunelu z~této strany. Ovládání brány je zpřístupněno pomocí nástroje pro stavbu budov, popsaného v části \ref{sec:buildingbuilder}.
	\end{minipage}
}

\medskip
\noindent{
	\begin{minipage}{0.15\textwidth}
		\includegraphics[scale=0.6]{Tower}
	\end{minipage} \hfill
	\begin{minipage}{0.8\textwidth}
		\textbf{Tower}, neboli věž. Tato budova umožňuje chůzi po své střeše a~díky své výšce zvyšuje dostřel jednotek. Její střecha je propojena se střechami bran, věží a hradeb na přímo sousedících dlaždicích.
	\end{minipage}
}

\medskip
\noindent{
	\begin{minipage}{0.15\textwidth}
		\includegraphics[scale=0.6]{Wall}
	\end{minipage} \hfill
	\begin{minipage}{0.8\textwidth}
		\textbf{Wall}, neboli hradba. Hlavním účelem této budovy je zablokování přístupu do hradu. Dále umožňuje chůzi po své střeše. Střecha je automaticky propojena se střechami bran, věží a hradeb přímo sousedících s touto budovou.
	\end{minipage}
}

\medskip
\noindent{
	\begin{minipage}{0.15\textwidth}
		\includegraphics[scale=0.6]{Tree}
	\end{minipage} \hfill
	\begin{minipage}{0.8\textwidth}
		\textbf{Tree}, neboli strom. Stavba stromů je omezena na editaci úrovně a~jejich vlastníkem může být pouze neutrální hráč, tedy hráč s~šedým štítem. Tento hráč jako jediný nemusí vlastnit tvrz, nelze ho zabít a~neútočí na ostatní hráče. Stromy podle typu dlaždice, na které se nacházejí, rostou různou rychlostí a~množí se s~různou pravděpodobností.
	\end{minipage}
}

\medskip
\noindent{
	\begin{minipage}{0.15\textwidth}
		\includegraphics[scale=0.6]{TreeCutter}
	\end{minipage} \hfill
	\begin{minipage}{0.8\textwidth}
		\texttt{TreeCutter}, neboli dřevorubec umožňuje hráči získávat ze stromů dřevo. Tato budova vytváří dvě jednotky, které následně pendlují mezi nejbližším stromem a~touto budovou, čímž získávají pro hráče dřevo.
	\end{minipage}
}

\subsection{Jednotky}
Balíček ukázkové hry definuje tři typy jednotek, z toho dvě ovladatelné hráčem a jednu plně automatickou.

\medskip
\noindent{
	\begin{minipage}{0.15\textwidth}
		\includegraphics[scale=0.6]{Chicken}
	\end{minipage} \hfill
	\begin{minipage}{0.8\textwidth}
		\textbf{Chicken} je jednotka útočící na dálku, používající vajíčka jako projektily. Dostřel této jednotky závisí na rozdílu její výšky od cíle, je tedy výhodné ji umisťovat na vyvýšená místa, jako například budovy. Oproti vlkům dokáže tento typ jednotek chodit přes vodu a~poškodit obranné budovy.
	\end{minipage}
}

\medskip
\noindent{
	\begin{minipage}{0.15\textwidth}
		\includegraphics[scale=0.6]{Wolf}
	\end{minipage} \hfill
	\begin{minipage}{0.8\textwidth}
		\textbf{Wolf}, neboli vlk, je jednotka útočící na blízko. Tato jednotka se pohybuje rychleji než \texttt{Chicken} nebo \texttt{Dog}. Tato jednotka nedokáže poškodit nepřátelské obranné budovy.
	\end{minipage}
}

\medskip
\noindent{
	\begin{minipage}{0.15\textwidth}
		\includegraphics[scale=0.6]{Dog}
	\end{minipage} \hfill
	\begin{minipage}{0.8\textwidth}
		\textbf{Dog}, neboli pes, je jednotka vytvářená budovou dřevorubce, která se pohybuje mezi budovou a nejbližším stromem a získává tak dřevo. Tuto jednotku nelze manuálně vytvářet ani ovládat, vše je plně automatizováno. 
	\end{minipage}
}

\subsection{Projektily}
Ukázková hra definuje pouze jeden projektilů, ze kterých je pouze jeden aktuálně využíván. Těmito typy jsou:

\begin{enumerate}
	\item \texttt{EggProjectile},
	\item \texttt{TestProjectile}.
\end{enumerate}

První typ je využíván jednotkou \texttt{Chicken}. Tento typ má jednoduchou logiku využívající komponenty \texttt{BallisticProjectile} pro implementaci svého pohybu. Drhý typ ukazuje možnosti složitějšího chování projektilu. Tento projektil také využívá pro implementaci svého pohybu komponentu \texttt{BallisticProjectile}, ale dále přidává dodatečné chování. Po určité době od výstřelu se tento projektil rozdělí na více projektilů stejného druhu, čímž vytvoří efekt brokovnice a~pokryje projektily okolí svého původního dopadu. Dále tento projektil ukazuje zpožděné odstranění z~úrovně, díky kterému zůstává určitou chvíli zaseknutý v~terénu.

\subsection{Umělé inteligence hráčů}
Ukázková hra poskytuje dvě umělé inteligence nepřátelských hráčů, kterými jsou:

\begin{enumerate}
	\item \texttt{LazyPlayer},
	\item \texttt{AggressivePlayer}.
\end{enumerate}

Lazy player je jednoduchá umělá inteligence, která nic nedělá. Jejím hlavním účelem umožnění uživately vyzkoušení herního režimu platformy a~možnost stavby svého hradu bez jakéhokoli ohrožení nepřítelem.

AggressivePlayer je aktivní umělá inteligence, která staví budovy pro získávání dřeva, jednotky pro obranu svého tvrze a~útok na nejbližšího nepřátelského hráče.

\subsection{Umělé inteligence úrovní}
Ukázková hra je příkladem balíčku, který všechnu svou logiku decentralizuje do jednotek, budov, projektilů a~hráčů. Z~tohoto důvodu obsahují dvě logiky definované ukázkovou hrou minimum herní logiky. Ukázková hra poskytuje tyto dvě logiky:

\begin{enumerate}
	\item \texttt{TwoPlayerLogic},
	\item \texttt{FourPlayerLogic}.
\end{enumerate}

Obě tyto logiky poskytují požadované metody informace platformě ve formě \texttt{ToolManageru} a \texttt{AStarFactory}. 

První logika již podle názvu definuje úrovně se dvěma hráči. Dále poskytuje možnost před prvním spuštěním úrovně nastavit počáteční množství surovin vlastněné hráči.
Druhá logiky definuje úrovně se čtyřmi hráči a~určuje pevné množství počátečních surovin.


\chapter*{Závěr}
\addcontentsline{toc}{chapter}{Závěr}
Na závěr práce shrneme implementaci naší platformy a~porovnáme ji s~našimi cíli, uvedenými v~části \ref{sec:cileprace}.

Výsledkem naší práce je platforma pro tvorbu 3D RTS her pro jednoho hráče, implementovaná pomocí jazyka C\# a~enginu UrhoSharp. Tato platforma umožňuje tvorbu balíčků, které lze distribuovat separátně od platformy a~přidávat i~do nainstalované instance platformy. Tyto balíčky mohou obsahovat nové typy jednotek, budov, projektilů, nepřátelských hráčů, nástrojů pro editaci map a~dalších herních prvků, které je možné využít pro tvorbu a~hraní map. 

Platforma umožňuje tvorbu jednotek schopných pohybu po herním světě, řízeného jak rozkazy hráče, tak umělou inteligencí jednotek. Jednotky je možné vytvářet během editace úrovně i~v~průběhu hry. Jednotky jsou schopné útočit na dálku, na blízko či oběma způsoby najednou na rozkaz hráče či z~rozhodnutí umělé inteligence. Při zásahu je jednotka o~této události informována, což umožňuje implementaci systému \textit{hit-pointů}.

Budovy lze v~platformě umisťovat do herního světa jak při editaci mapy, tak v~průběhu hry. Následně je možné v~průběhu hry budovy poškozovat a~při dostatečném poškození následně zničit. Dále dokáží budovy rozšiřovat prostor dostupný jednotkám mimo úroveň terénu. V~neposlední řadě platforma umožňuje vytvářet budovy produkující suroviny, vytvářející jednotky či stavějící další budovy.

Z~pohledu surovin platforma umožňuje přidání a~odebrání libovolného množství surovin v reakci na akci hráče, jednotky, budovy či uplynutí času, čímž je umožněna implementace aktivního i~pasivního získávání surovin.

Dostupnost jednotek a~budov pro tvorbu hráčem je plně v~rukou tvůrce pluginů, čímž platforma umožňuje implementaci systémů technologií a~postupného odemykání typů jednotek a~budov.

Implementace mapy je v~platformě rozdělena na čtvercové dlaždice, umožňující změnu výšek svých jednotlivých rohů dlaždic. Každé dlaždici je přiřazen typ, který je spolu s~přítomnými jednotkami a~budovami na dlaždici možné využít pro rozhodování v~implementaci pluginů.

Systém balíčků umožňuje vytvářet balíčky obsahující typy těchto herních prvků:
\begin{enumerate}
	\item jednotek,
	\item budov,
	\item projektilů,
	\item surovin,
	\item dlaždic,
	\item hráčů,
	\item úrovní.	
\end{enumerate} 
Pro typy s~grafickou reprezentací v~herním světě umožňuje balíček přidání modelů,textur a~animací. Pro typy, které umožňují autonomní chování, je umožněno přidat pluginy, které následně toto chování definují.

Z~pohledu koncového uživatele platforma poskytuje grafické rozhraní pro stolní počítače, umožňuje přidávání a~výběr balíčků a~následný výběr či tvorbu úrovní z~těchto balíčků. Dále platforma poskytuje ukládání a~načítání aktuálního stavu hry.

Ukázková hra následně demonstruje výše popsané vlastnosti pomocí několika typů jednotek, budov, surovin, projektilů, hráčů a~úrovní.


Z~popisu platformy tedy můžeme uznat cíle práce, uvedené v~části \ref{sec:cileprace}, za splněné.

\section{Možná rozšíření}
Přestože je aktuální verze platformy plně funkční a~splňuje všechny cíle naší práce, existuje několik oblastí, v~kterých by mohla být platforma rozšířena:

\begin{itemize}
	\item Implementace podpory pro systém Android, jejíž kostra je v~aktuální verzi připravena.
	\item Přidání dalších typů jednotek a~budov, poskytujících větší strategické možnosti při hraní ukázkové hry. Aktuální budovy a~jednotky v~ukázkové hře slouží pouze pro demonstraci schopností platformy a~nejsou navrženy s~ohledem na jejich použití hráčem.
	\item Množina umělých inteligencí hráčů je v~aktuální verzi omezena pouze na dvě možnosti, a~to hráče, který nedělá nic, a~hráče, který je agresivní a~útočí. V~budoucnu by bylo výhodné přidat hráče, kteří i~brání, či vytvářejí svoji strategii pomocí složitějších postupů.
	\item Verze platformy distribuovaná s~touto prací obsahuje pouze jeden balíček, obsahující ukázkovou hru. Pro budoucí verze platformy by bylo užitečné přidat další balíčky, poskytující jiné typy her.
	\item Vylepšení návrhu a~vzhledu uživatelského rozhraní platformy i~hry. Aktuální uživatelské rozhraní je navrženo především pro demonstraci schopností platformy a~nedbá příliš na estetiku či jednoduchost používání. 
	\item Přidání tzv.~\uv{Fog of War} funkcionality, která zakrývá části mapy vzdálené od jednotek a~budov hráče a~jeho spojenců. Tato funkcionalita je přítomná ve velké části RTS her a~její přítomnost by rozšířila množinu her implementovatelných v~naší platformě.
	\item I~když cílem naší práce byla tvorba platformy pro hry jednoho hráče, použitý herní engine podporuje tvorbu her pro více hráčů. Využitím těchto služeb by mělo být možné přidat mód pro více hráčů.
	\item Platforma umožňuje tvůrcům pluginů implementovat vlastní algoritmy pro hledání nejkratší cesty v~herním světě. V~aktuální verzi navíc platforma poskytuje jednu implementaci algoritmu A* pro tento účel. V~budoucnu by mohlo být výhodné přidat další algoritmy přímo do platformy a~poskytnout je tak tvůrcům pluginů.
	
\end{itemize}


%%% Seznam použité literatury
\include{literatura}

%%% Obrázky v bakalářské práci
%%% (pokud jich je malé množství, obvykle není třeba seznam uvádět)
%\listoffigures

%%% Tabulky v bakalářské práci (opět nemusí být nutné uvádět)
%%% U matematických prací může být lepší přemístit seznam tabulek na začátek práce.
%\listoftables

%%% Použité zkratky v bakalářské práci (opět nemusí být nutné uvádět)
%%% U matematických prací může být lepší přemístit seznam zkratek na začátek práce.
%\chapwithtoc{Seznam použitých zkratek}

%%% Přílohy k bakalářské práci, existují-li. Každá příloha musí být alespoň jednou
%%% odkazována z vlastního textu práce. Přílohy se číslují.
%%%
%%% Do tištěné verze se spíše hodí přílohy, které lze číst a prohlížet (dodatečné
%%% tabulky a grafy, různé textové doplňky, ukázky výstupů z počítačových programů,
%%% apod.). Do elektronické verze se hodí přílohy, které budou spíše používány
%%% v elektronické podobě než čteny (zdrojové kódy programů, datové soubory,
%%% interaktivní grafy apod.). Elektronické přílohy se nahrávají do SISu a lze
%%% je také do práce vložit na CD/DVD. Povolené formáty souborů specifikuje
%%% opatření rektora č. 72/2017.
\appendix
\chapter{Přílohy}
\label{sec:appendix}
Výsledný program této práce i jeho zdrojový kód jsou licencovány pod MIT licencí.
Přílohy práce:
\begin{itemize}
	\item Adresář \texttt{/src} obsahující \texttt{MHUrho} solution, tvořící implementaci platformy spolu s implementací ukázkové hry. 
	\item Adresář \texttt{/doc} obsahující dokumentaci vygenerovanou z dokumentačních komentářů kódu.
	\item Adresář \texttt{/lib} obsahující knihovnu \texttt{MHUrho.dll} pro tvorbu pluginů.
	\item Adresář \texttt{/install} obsahující instalátor platformy, umožňující koncovým uživatelům i tvůrcům balíčků její instalaci.
	\item Adresář \texttt{/schemas} obsahující schémata pro validaci XML souborů.
		\begin{itemize}
			\item Soubor \texttt{GamePack.xsd} obsahující schéma pro validaci XML balíčků.
		\end{itemize}
	\item Adresář \texttt{/templates} obsahující šablony pro tvobu balíčku.
		\begin{itemize}
			\item Soubor \texttt{GamePackTemplate.xml} obsahující šablonu XML souboru balíčku.
		\end{itemize}
	\item Soubor \texttt{/prace.pdf} obsahující tento text práce.
	\item Soubor \texttt{/LICENSE.txt} obsahující licenci platformy MHUrho.
\end{itemize}

\openright
\end{document}
